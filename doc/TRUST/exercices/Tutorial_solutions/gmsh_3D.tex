\begin{alltt}
// Variables definition
{\bf{lc = 0.02;}}
// First cell size (used when points are defined)
lc1 = lc * 8;
// Second cell size
lc2 = lc / 2;
// Circle diameter
H=2;
L=10;
// Points definition 
Point(1)=\{0,0,0,lc1\};
Point(2) = \{L,0,0,lc1\};
Point(3) = \{L,H,0,lc1\};
Point(4) = \{0,H,0,lc1\};
Point(10)= \{1,1,0,lc2\};
Point(11)=\{1.25,1,0,lc2\}; 
Point(12)=\{1,1.25,0,lc2\}; 
Point(13)=\{0.75,1,0,lc2\}; 
Point(14)=\{1,0.75,0,lc2\};
// Lines definition
Line(2) = \{1,2\};
Line(6) = \{3,2\};
Line(7) = \{3,4\};
Line(8) = \{4,1\};
// 1/4 Circle definition
// 3 points for the circle arc
// A circle arc is STRICTLY smaller than Pi
Circle(10)=\{11,10,12\}; 
Circle(11)=\{12,10,13\}; 
Circle(12)=\{13,10,14\}; 
Circle(13)=\{14,10,11\};
// Naming the boundaries
{\bf{//Physical Line("Outlet") = \{6\};}}
{\bf{//Physical Line("Wall") = \{2,7\};}}
{\bf{//Physical Line("Inlet") = \{8\};}}
{\bf{//Physical Line("Circle") = \{10,11,12,13\};}}
// A lineloop is a loop on several lines
// for defining/orienting a surface
// Use negative lines to reverse the
// orientation of the line
Line Loop(1) = \{2,-6,7,8\};
Line Loop(2) = \{10,11,12,13\};
/// The surface will use the lineloop
Plane Surface(1) = \{1,2\};
{\bf{Extrude \{0,0,1\} \{ Surface\{1\} ; \}}}
// Naming the domain
{\bf{Physical Surface("Inlet") = \{38\};}}
{\bf{Physical Surface("Outlet") = \{30\};}}
{\bf{Physical Surface("Wall") = \{1,26,34,55\};}}
{\bf{Physical Surface("Obstacle") = \{42,46,50,54\};}}
{\bf{Physical Volume("dom") = \{1\};}}
\end{alltt}
