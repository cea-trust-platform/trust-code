\begin{alltt}
TAILLES 3 203 43 4 
RIEN 0 meshing
MAIL_X_REG 4 xomin xomax nx-2 2 dx
MAIL_Y_REG 4 yomin yomax ny-2 2 dy
MAIL_Z_REG 4 zomin zomax nz-2 2 dz
RIEN 0 filling up of the domain
PARALAX 7 xomin xomax yomin yomax zomin zomax 1000 filling up of the domain
RIEN 0 boundaries domain definition
PARALAX 7 xomin-eps xomax+eps yomin-eps yomax+eps zomax zomax+eps 0 lateral
PARALAX 7 xomin-eps xomax+eps yomin-eps yomax+eps zomin-eps zomin 0 
PARALAX 7 xomin-eps xomin yomin-eps yomax+eps zomin-eps zomax+eps -1000 inlet
PARALAX 7 xomax xomax+eps yomin-eps yomax+eps zomin-eps zomax+eps -2000 outlet
PARALAX 7 xomin-eps xomax+eps yomin-eps yomin zomin-eps zomax+eps 0 
PARALAX 7 xomin-eps xomax+eps yomax yomax+eps zomin-eps zomax+eps 0 
FORT 0 do i=0,5
FORT 0 do j=0,2
CYLAX 7 x0+2*i*dx y0+2*j*dy radius zomin-eps zomax+eps 3 -9000 holes1
FORT 0 enddo
FORT 0 do j=0,1
CYLAX 7 x0+(2*i+1)*dx y0+(2*j+1)*dy radius zomin-eps zomax+eps 3 -9000 holes2
FORT 0 enddo
FORT 0 enddo
COUPE2D 3 0. 3 -5000 coupe_2D
file maillage
        real xomin,xomax,yomin,yomax,zomin,zomax,eps,radius,dx,dy, x0, y0
C       Extremal sizes of the meshing
        xomin=0.
        xomax=10.
        yomin=0.
        yomax=2.
        zomin=0.
        zomax=0.01
C       eps is useful for thickness of the blocks featuring boundaries
        eps=0.0001
        radius=0.2
        dx=0.8
        dy=0.3
        x0=0.6
        y0=0.4
        XM(1)=xomin-eps
        XM(nx)=xomax+eps
        YM(1)=yomin-eps
        YM(ny)=yomax+eps
        ZM(1)=zomin-eps
        ZM(nz)=zomax+eps
\end{alltt}
