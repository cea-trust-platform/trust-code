%%%%%%%%%%%%%%%%%%%%%%%%%%%%%%%%%%%%%%%%%
% Beamer Presentation
% LaTeX Template
% Version 1.0 (10/11/12)
%
% This template has been downloaded from:
% http://www.LaTeXTemplates.com
%
% License:
% CC BY-NC-SA 3.0 (http://creativecommons.org/licenses/by-nc-sa/3.0/)
%
%%%%%%%%%%%%%%%%%%%%%%%%%%%%%%%%%%%%%%%%%

%%%%%%%%%%%%%%%%%%%%%%%%%%%%%%%%%%%%%%%%%
%
% Themes and packages have been token from:
% $TRUST_ROOT/doc/TRUST/Tutorial/Baltik_tutorial.tex
%
%%%%%%%%%%%%%%%%%%%%%%%%%%%%%%%%%%%%%%%%%
%----------------------------------------------------------------------------------------
%    PACKAGES AND THEMES
%----------------------------------------------------------------------------------------
\documentclass[10pt, hyperref={unicode=true,pdfusetitle, bookmarks=true,bookmarksnumbered=false,bookmarksopen=false, breaklinks=false,pdfborder={0 0 1},backref=true,colorlinks=true,linkcolor=darkblue,pageanchor, urlcolor=darkblue}]{beamer}

\mode<presentation> {
\usetheme{CambridgeUS}
}

\usepackage{graphicx} % Allows including images
\definecolor{darkblue}{HTML}{3535B4}
\usepackage[T1]{fontenc}
\usepackage{alltt}

%----------------------------------------------------------------------------------------
%    TITLE PAGE
%----------------------------------------------------------------------------------------
%\title[Notes for Eclipse with TRUST (08/03/2018)]{Notes for Eclipse with TRUST \\ \footnotesize{LaTex version of \$TRUST\_ROOT/src/README\_ECLIPSE}}
\title[Notes for Eclipse with TRUST (08/03/2018)]{Notes for Eclipse with TRUST}
\institute[CEA/DEN/DANS/DM2S/STMF] 
{
CEA Saclay \\
\medskip
\textit{Support team: trust@cea.fr}
\medskip
}
\date{30/11/2018}

\begin{document}

%%%%%%%%%%%%%%%%%%%%%%%%%%%%%%%%%%%%%%%%%%%%%%%%%%%%%%%%%%%%%%%%%%%%%%%%
\begin{frame}
\titlepage % Print the title page as the first slide
\end{frame}
%%%%%%%%%%%%%%%%%%%%%%%%%%%%%%%%%%%%%%%%%%%%%%%%%%%%%%%%%%%%%%%%%%%%%%%%

%----------------------------------------------------------------------------------------
%    PRESENTATION SLIDES
%----------------------------------------------------------------------------------------
%%%%%%%%%%%%%%%%%%%%%%%%%%%%%%%%%%%%%%%%%%%%%%%%%%%%%%%%%%%%%%%%%%%%%%%%
\begin{frame}
\tableofcontents [hideallsubsections]
\end{frame}
%%%%%%%%%%%%%%%%%%%%%%%%%%%%%%%%%%%%%%%%%%%%%%%%%%%%%%%%%%%%%%%%%%%%%%%%


%%%%%%%%%%%%%%%%%%%%%%%%%%%%%%%%%%%%%%%%%%%%%%%%%%%%%%%%%%%%%%%%%%%%%%%%
%%%%%%%%%%%%%%%%%%%%%%%%%%%%%%%%%%%%%%%%%%%%%%%%%%%%%%%%%%%%%%%%%%%%%%%%
\section{{\bf{Download \& configure Eclipse}}}
\begin{frame}
\frametitle{Download \& configure Eclipse (I)}

\begin{block}{Download Eclipse}
\begin{itemize}
\item Go to the website of the Eclipse Foundation: \url{http://www.eclipse.org/downloads/eclipse-packages/}
\item Click on \textbf{Eclipse Neon (4.6)} on the menu \textbf{More downloads}.
\item Select \textbf{Eclipse IDE for C/C++ Developers} $\rightarrow$ \textbf{Linux 64-bits}
\item Download the \textbf{eclipse-cpp-neon-3-linux-gtk-x86\_64.tar.gz} package in your directory \texttt{Formation\_TRUST/yourname}
\item For OS older than CentOs7, Ubuntu16.04 and Fedora22, download \textbf{Eclipse Mars} version: \textbf{eclipse-cpp-mars-2-linux-gtk-x86\_64.tar.gz}
\end{itemize}
\end{block}

\begin{block}{Untar the downloaded Eclipse archive}
 \texttt{\$ cd Formation\_TRUST/yourname} \\
 \texttt{\$ tar xfz eclipse-*.tar.gz} \\
 \texttt{\$ cd eclipse}
\end{block}

\end{frame}
%%%%%%%%%%%%%%%%%%%%%%%%%%%%%%%%%%%%%%%%%%%%%%%%%%%%%%%%%%%%%%%%%%%%%%%%
\begin{frame}
\frametitle{Download \& configure Eclipse (II)}

\begin{block}{For Ubuntu16.04, Fedora22, CentOs 7 and recents OS}
Edit the \textit{eclipse.ini} file by deleting the last 2 lines (Xms and Xmx) and adding the following lines:\\
\textbf{Xms512m} \\
\textbf{Xmx2048m}
\end{block}

\begin{block}{For older OS}
Edit the \textit{eclipse.ini} file, by deleting the last 3 lines (MaxPermSize, Xms and Xmx) and adding the following ones: \\
\textbf{Xmn256m} \\
\textbf{Xss2m}\\
\textbf{server}\\
\textbf{Xms512m}\\
\textbf{Xmx2048m}
\end{block}

\end{frame}
%%%%%%%%%%%%%%%%%%%%%%%%%%%%%%%%%%%%%%%%%%%%%%%%%%%%%%%%%%%%%%%%%%%%%%%%


%%%%%%%%%%%%%%%%%%%%%%%%%%%%%%%%%%%%%%%%%%%%%%%%%%%%%%%%%%%%%%%%%%%%%%%%
%%%%%%%%%%%%%%%%%%%%%%%%%%%%%%%%%%%%%%%%%%%%%%%%%%%%%%%%%%%%%%%%%%%%%%%%
\section{{\bf{Create a TRUST platform project}}}
\begin{frame}
\tableofcontents[sections={1-5},currentsection, currentsubsection]
\end{frame}
%%%%%%%%%%%%%%%%%%%%%%%%%%%%%%%%%%%%%%%%%%%%%%%%%%%%%%%%%%%%%%%%%%%%%%%%
\begin{frame}
\frametitle{Create a TRUST platform project (I)}

\begin{block}{Initialize TRUST environnement}
\texttt{\$ source /home/triou/env\_TRUST\_X.Y.Z.sh} \\
\texttt{\$ echo \$TRUST\_ROOT/src} \\
\texttt{\$ echo \$exec\_debug} \\
\end{block}

\begin{block}{Launch Eclipse}
\texttt{\$ mkdir -p Formation\_TRUST/yourname/workspace}\\
\texttt{\$ cd Formation\_TRUST/yourname/eclipse}\\
\texttt{\$ ./eclipse \&}
%
\begin{itemize}
\item Workspace: Browse the directory \texttt{Formation\_TRUST/yourname/workspace} 
\item Welcome : close x button
\end{itemize}
%
\end{block}

\end{frame}
%%%%%%%%%%%%%%%%%%%%%%%%%%%%%%%%%%%%%%%%%%%%%%%%%%%%%%%%%%%%%%%%%%%%%%%%
\begin{frame}
\frametitle{Create a TRUST platform project (II)}

\begin{block}{Create the project}
\begin{itemize}
\item File $\rightarrow$ New  $\rightarrow$  C++ Project \\
 $\Rightarrow$ Project name: TRUST-X.Y.Z \\
 $\Rightarrow$ Project type: "Executable"  $\rightarrow$  "Empty Project" \\
 $\Rightarrow$ Toolchains: "Linux GCC" \\
 $\Rightarrow$ Finish
\end{itemize}
\end{block}

\begin{block}{Import source files into the already created project}
\begin{itemize}
\item From the "Project Explorer" tab, right click on TRUST-X-Y-Z  $\rightarrow$  "Import..." \\
  $\Rightarrow$ General  $\rightarrow$ File System $\rightarrow$ Next \\
  $\Rightarrow$ From directory: copy the string matching \texttt{\$TRUST\_ROOT/src/} \\
  $\Rightarrow$ Check "Select All" \\
  $\Rightarrow$ Into folder: TRUST-X.Y.Z \\
  $\Rightarrow$ Finish \\
  $\Rightarrow$ Wait to have 100\% at the bottom right corner of the window  (C/C++ indexer).
\end{itemize}
\end{block}

\end{frame}
%%%%%%%%%%%%%%%%%%%%%%%%%%%%%%%%%%%%%%%%%%%%%%%%%%%%%%%%%%%%%%%%%%%%%%%%
\begin{frame}
\frametitle{Create a TRUST platform project (III)}

\begin{block}{Configure the project and launch a computation}
\begin{itemize}
\item From the "Project Explorer" tab, right click on TRUST-X.Y.Z $\rightarrow$ Properties \\
  $\Rightarrow$ Builders: uncheck "CDT Builder" $\rightarrow$ OK $\rightarrow$ OK 
\item From the "Project Explorer" tab, right click on TRUST-X.Y.Z $\rightarrow$ "Debug As" $\rightarrow$ "Debug Configurations..."\\
  $\Rightarrow$ Right click on "C/C++ Application" $\rightarrow$ New 
  \begin{itemize}
  \item In the "Main" tab: \\
  $\Rightarrow$ Project: TRUST-X.Y.Z \\
  $\Rightarrow$ "C/C++ Application": copy the string matching \$exec\_debug \\
  $\Rightarrow$ "Apply" \\
  \item In the "Arguments" tab:\\
  $\Rightarrow$ "Program arguments" $\rightarrow$ specify the name of your datafile \\
  $\Rightarrow$ "Working directory" $\rightarrow$ uncheck "Use default" and select the directory with path containing the datafile \\
  $\Rightarrow$ "Apply"
  \end{itemize}
  $\Rightarrow$ "Debug" 
\end{itemize}
\end{block}

\end{frame}
%%%%%%%%%%%%%%%%%%%%%%%%%%%%%%%%%%%%%%%%%%%%%%%%%%%%%%%%%%%%%%%%%%%%%%%%


%%%%%%%%%%%%%%%%%%%%%%%%%%%%%%%%%%%%%%%%%%%%%%%%%%%%%%%%%%%%%%%%%%%%%%%%
%%%%%%%%%%%%%%%%%%%%%%%%%%%%%%%%%%%%%%%%%%%%%%%%%%%%%%%%%%%%%%%%%%%%%%%%
\section{{\bf{Create a basic BALTIK project without dependency}}}
\begin{frame}
\tableofcontents[sections={1-5},currentsection, currentsubsection]
\end{frame}
%%%%%%%%%%%%%%%%%%%%%%%%%%%%%%%%%%%%%%%%%%%%%%%%%%%%%%%%%%%%%%%%%%%%%%%%
\begin{frame}
\frametitle{Create a basic BALTIK project without dependency (I)}

\begin{block}{Initialize baltik environnement}
 \texttt{\$ source env\_baltik.sh } \\
 \texttt{\$ echo \$project\_directory/src }
\end{block}

\begin{block}{Launch Eclipse}
\texttt{\$ cd Formation\_TRUST/yourname/eclipse } \\
\texttt{\$ ./eclipse \& }
\end{block}

\begin{block}{Create the project}
\begin{itemize}
\item File $\rightarrow$ New $\rightarrow$ "Makefile Project with Existing Code" \\
  $\Rightarrow$ Project name: MY\_BALTIK \\
  $\Rightarrow$ Existing Code Location: copy string matching \texttt{\$project\_directory/src} \\
  $\Rightarrow$ Toolchain for Indexer Settings: "Linux GCC" \\
  $\Rightarrow$ Finish \\
  $\Rightarrow$ Wait to have 100\% at the bottom right corner of the window  (C/C++ indexer).
\end{itemize}
\end{block}

\end{frame}
%%%%%%%%%%%%%%%%%%%%%%%%%%%%%%%%%%%%%%%%%%%%%%%%%%%%%%%%%%%%%%%%%%%%%%%%
\begin{frame}
\frametitle{Create a basic BALTIK project without dependency (II)}

\begin{block}{Configure the BALTIK project and link it with TRUST}
\begin{itemize}
\item From the "Project Explorer" tab, right click on MY\_BALTIK $\rightarrow$ Properties \\
  $\Rightarrow$ Builders: check "CDT Builder" \\
  $\Rightarrow$ C/C++ Build : 
  \begin{itemize}
  \item Builder Settings: Build directory: \texttt{\$\{workspace\_loc:/MY\_BALTIK\}/../} or copy the string matching \texttt{\$project\_directory/} \\
  \item Behavior: check "Build (Incremental build)": debug optim (instead of all) 
  \end{itemize}
  $\Rightarrow$ Project References: check TRUST-X.Y.Z $\rightarrow$ OK
\end{itemize}
\end{block}
  
\begin{block}{Build the BALTIK project}
From the "Project Explorer" tab, right click MY\_BALTIK $\rightarrow$ Index $\rightarrow$ Rebuild \\
  $\Rightarrow$ Wait to have 100\% at the bottom right corner of the window  (C/C++ indexer). \\
  Right click MY\_BALTIK $\rightarrow$ Build Project \\
\end{block}

\end{frame}
%%%%%%%%%%%%%%%%%%%%%%%%%%%%%%%%%%%%%%%%%%%%%%%%%%%%%%%%%%%%%%%%%%%%%%%%
\begin{frame}
\frametitle{Create a basic BALTIK project without dependency (III)}

\begin{block}{Launch a computation}
\begin{itemize}
\item From the "Project Explorer" tab, right click MY\_BALTIK $\rightarrow$ "Debug As" $\rightarrow$ "Debug Configurations..." \\
  $\Rightarrow$ C/C++ Application $\rightarrow$ New 
  \begin{itemize}
  \item In the "Main" tab:\\
  $\Rightarrow$ Project: MY\_BALTIK \\
  $\Rightarrow$ C/C++ Application: \texttt{\$\{workspace\_loc:/MY\_BALTIK\}/../basic} or copy the string matching \texttt{\$project\_directory/basic} \\
  $\Rightarrow$ "Apply" 
  \item In the "Arguments" tab:\\
  $\Rightarrow$ Program arguments $\rightarrow$ specify the name of your datafile \\
  $\Rightarrow$ Working directory $\rightarrow$ uncheck "Use default" and select the  directory containing the datafile \\
  $\Rightarrow$ "Apply"
  \end{itemize}
  $\Rightarrow$ Debug
\end{itemize}
\end{block}

\end{frame}
%%%%%%%%%%%%%%%%%%%%%%%%%%%%%%%%%%%%%%%%%%%%%%%%%%%%%%%%%%%%%%%%%%%%%%%%


%%%%%%%%%%%%%%%%%%%%%%%%%%%%%%%%%%%%%%%%%%%%%%%%%%%%%%%%%%%%%%%%%%%%%%%%
%%%%%%%%%%%%%%%%%%%%%%%%%%%%%%%%%%%%%%%%%%%%%%%%%%%%%%%%%%%%%%%%%%%%%%%%
\section{{\bf{Useful shortcuts in sources}}}
\begin{frame}
\tableofcontents[sections={1-5},currentsection, currentsubsection]
\end{frame}
%%%%%%%%%%%%%%%%%%%%%%%%%%%%%%%%%%%%%%%%%%%%%%%%%%%%%%%%%%%%%%%%%%%%%%%%
\begin{frame}
\frametitle{Useful shortcuts in sources}

\begin{block}{Shortcuts}
\vspace{0.2cm}
\begin{itemize} 
\item Open a cpp file from Project Explorer tab: \\
  Double click on TRUST-X.Y.Z $\rightarrow$ Kernel $\rightarrow$ Framework $\rightarrow$ Probleme\_base.cpp \\ \vspace{0.2cm}
\item In the cpp file: Right click on method "initialize()" \\
  $\Rightarrow$ F3: Opens Declaration \\
  $\Rightarrow$ F4: Open Type Hierarchy \\
  $\Rightarrow$ Ctrl+Alt+H: Open Call Hierarchy \\
  $\Rightarrow$ "Alt+$\rightarrow$" and "Alt+$\leftarrow$": Move from a tab to another
\end{itemize}
\end{block}

\end{frame}

\end{document}
