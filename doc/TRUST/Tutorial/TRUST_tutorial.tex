%%%%%%%%%%%%%%%%%%%%%%%%%%%%%%%%%%%%%%%%%
% Beamer Presentation
% LaTeX Template
% Version 1.0 (10/11/12)
%
% This template has been downloaded from:
% http://www.LaTeXTemplates.com
%
% License:
% CC BY-NC-SA 3.0 (http://creativecommons.org/licenses/by-nc-sa/3.0/)
%
%%%%%%%%%%%%%%%%%%%%%%%%%%%%%%%%%%%%%%%%%

%----------------------------------------------------------------------------------------
%    PACKAGES AND THEMES
%----------------------------------------------------------------------------------------

%%TO COMPILE AND GENERATE THE INDEX
%pdflatex TRUST_tutorial.tex
%makeindex TRUST_tutorial.idx
%pdflatex TRUST_tutorial.tex
%pdflatex TRUST_tutorial.tex

%% TO IMPRESS 
% select scale : 180%

\documentclass[10pt, hyperref={unicode=true,pdfusetitle, bookmarks=true,bookmarksnumbered=false,bookmarksopen=false, breaklinks=false,pdfborder={0 0l 1},backref=true,colorlinks=true,linkcolor=darkblue,pageanchor}]{beamer}

\mode<presentation> {

% The Beamer class comes with a number of default slide themes
% which change the colors and layouts of slides. Below this is a list
% of all the themes, uncomment each in turn to see what they look like.

%\usetheme{default}
%\usetheme{AnnArbor}
%\usetheme{Antibes}
%\usetheme{Bergen}
%\usetheme{Berkeley}
%\usetheme{Berlin}
%\usetheme{Boadilla}
\usetheme{CambridgeUS}
%\usetheme{Copenhagen}
%\usetheme{Darmstadt}
%\usetheme{Dresden}
%\usetheme{Frankfurt}
%\usetheme{Goettingen}
%\usetheme{Hannover}
%\usetheme{Ilmenau}
%\usetheme{JuanLesPins}
%\usetheme{Luebeck}
%\usetheme{Madrid}
%\usetheme{Malmoe}
%\usetheme{Marburg}
%\usetheme{Montpellier}
%\usetheme{PaloAlto}
%\usetheme{Pittsburgh}
%\usetheme{Rochester}
%\usetheme{Singapore}
%\usetheme{Szeged}
%\usetheme{Warsaw}

% As well as themes, the Beamer class has a number of color themes
% for any slide theme. Uncomment each of these in turn to see how it
% changes the colors of your current slide theme.

%\usecolortheme{albatross} % bleubleu
%\usecolortheme{beaver} % rouge
%\usecolortheme{beetle} % bleu/gris
%\usecolortheme{crane}  % jaune
%\usecolortheme{dolphin} % bleu
%\usecolortheme{dove}  % blanc
%\usecolortheme{fly} % gris
%\usecolortheme{lily} % bleu
%\usecolortheme{orchid} % bleu
%\usecolortheme{rose} % bleu
%\usecolortheme{seagull} % gris
%\usecolortheme{seahorse} % bleu pale
%\usecolortheme{whale} % bleu
%\usecolortheme{wolverine} % jaune/bleu

%\setbeamertemplate{footline} % To remove the footer line in all slides uncomment this line
%\setbeamertemplate{footline}[page number] % To replace the footer line in all slides with a simple slide count uncomment this line

%\setbeamertemplate{navigation symbols}{} % To remove the navigation symbols from the bottom of all slides uncomment this line

}

\usepackage{graphicx} % Allows including images
%\usepackage{booktabs} % Allows the use of \toprule, \midrule and \bottomrule in tables
%\usepackage{multimedia} % for video
\definecolor{darkblue}{HTML}{3535B4}

\usepackage{multirow}
\usepackage{multicol}
\usepackage{makeidx}
    \makeindex


\newenvironment{theindex}
 {\let\item\par
  %definitions for subitem etc
  }{}
\newcommand\indexspace{}

%----------------------------------------------------------------------------------------
%    TITLE PAGE
%----------------------------------------------------------------------------------------
%\title[Short title]{Long title}
\title[TRUST Tutorial V1.9.6beta]{TRUST Tutorial V1.9.6beta} % The short title appears at the bottom of every slide, the full title is only on the title page

%\author{John Smith} % Your name
\institute[CEA/DES/ISAS/DM2S] % Your institution as it will appear on the bottom of every slide, may be shorthand to save space
{
CEA Saclay \\ % Your institution for the title page
\medskip
\textit{Support team: trust@cea.fr} % Your email address
\medskip
}
\date{\today} % Date, can be changed to a custom date


\begin{document}

%%%%%%%%%%%%%%%%%%%%%%%%%%%%%%%%%%%%%%%%%%%%%%%%%%%%%%%%%%%%%%%%%%%%%%%%
\begin{frame}
\titlepage % Print the title page as the first slide
\end{frame}
%%%%%%%%%%%%%%%%%%%%%%%%%%%%%%%%%%%%%%%%%%%%%%%%%%%%%%%%%%%%%%%%%%%%%%%%


%%%%%%%%%%%%%%%%%%%%%%%%%%%%%%%%%%%%%%%%%%%%%%%%%%%%%%%%%%%%%%%%%%%%%%%%
\begin{frame}
\frametitle{Overview} % Table of contents slide, comment this block out to remove it
%\tableofcontents[sectionstyle=show/show,subsectionstyle=hide/hide/hide ] 
\begin{small}
\begin{columns}[c] 
\column{.48\textwidth}
\vspace{-0.5cm}
\tableofcontents[sections={1-8}, hideallsubsections]
\column{.5\textwidth} 
\vspace{-0.5cm}
\tableofcontents[sections={9-16}, hideallsubsections]
\end{columns}
\end{small}
% Throughout your presentation, if you choose to use \section{} and \subsection{} commands, these will automatically be printed on this slide as an overview of your presentation
\end{frame}
%%%%%%%%%%%%%%%%%%%%%%%%%%%%%%%%%%%%%%%%%%%%%%%%%%%%%%%%%%%%%%%%%%%%%%%%

%----------------------------------------------------------------------------------------
%    PRESENTATION SLIDES
%----------------------------------------------------------------------------------------

\section{{\bf{Initialization}}}\label{init_trust}
%%%%%%%%%%%%%%%%%%%%%%%%%%%%%%%%%%%%%%%%%%%%%%%%%%%%%%%%%%%%%%%%%%%%%%%%
\begin{frame}
\begin{small}
\begin{columns}[c] 
\column{.45\textwidth}
\tableofcontents[sections={1-9},currentsection, currentsubsection]
\column{.5\textwidth} 
\tableofcontents[sections={10-16},currentsection, currentsubsection]
\end{columns}
\end{small}
\end{frame}
%%%%%%%%%%%%%%%%%%%%%%%%%%%%%%%%%%%%%%%%%%%%%%%%%%%%%%%%%%%%%%%%%%%%%%%%
%%%%%%%%%%%%%%%%%%%%%%%%%%%%%%%%%%%%%%%%%%%%%%%%%%%%%%%%%%%%%%%%%%%%%%%%
\begin{frame}
\frametitle{Initialization}
\begin{block}{}

To initialize TRUST environment:
\begin{itemize}
\item On CEA Saclay PCs, TRUST version is available with:\\
%\vspace{0.2cm}
\textbf{source  /home/trust\_trio-public/env\_TRUST-1.9.5.sh}
%\vspace{0.5cm}
\item On your own computer, download and install the latest version of TRUST in your local folder \$MyPathToTRUSTversion (unless this was already performed), then run on your terminal:\\
%\vspace{0.2cm}
\textbf{source  \$MyPathToTRUSTversion/env\_TRUST.sh}
%\vspace{0.5cm}
\end{itemize}


\end{block}
\end{frame}
%%%%%%%%%%%%%%%%%%%%%%%%%%%%%%%%%%%%%%%%%%%%%%%%%%%%%%%%%%%%%%%%%%%%%%%%
%%%%%%%%%%%%%%%%%%%%%%%%%%%%%%%%%%%%%%%%%%%%%%%%%%%%%%%%%%%%%%%%%%%%%%%%
\begin{frame}
\frametitle{Text Editor configuration}
\begin{block}{}
Several editors (\textbf{vim}, \textbf{emacs}, \textbf{nedit}, \textbf{gedit}) can be configured to highlight TRUST keywords when editing data files.

\begin{itemize}
 \item If you prefer using \textbf{nedit}, please do the following:
    \begin{itemize}
    \item [$\circ$] Run \textbf{nedit}, and select Preferences $\rightarrow$ Save Defaults.
    \item [$\circ$] Then run \textbf{trust -config\index{trust -config} nedit}, the message "nedit.rc updated" should appear.
    \end{itemize}

 \item If you prefer using \textbf{gedit}, run:
    \begin{itemize}
    \item [$\circ$] \textbf{trust -config\index{trust -config} gedit}
    \end{itemize}

\end{itemize}

\end{block}
\end{frame}
%%%%%%%%%%%%%%%%%%%%%%%%%%%%%%%%%%%%%%%%%%%%%%%%%%%%%%%%%%%%%%%%%%%%%%%%

\section{{\bf{Flow around an obstacle (2D, VDF)}}} \label{exo1}
\subsection{Sequential calculation}
%%%%%%%%%%%%%%%%%%%%%%%%%%%%%%%%%%%%%%%%%%%%%%%%%%%%%%%%%%%%%%%%%%%%%%%%
\begin{frame}
\begin{small}
\begin{columns}[c] 
\column{.45\textwidth}
\tableofcontents[sections={1-9},currentsection, currentsubsection]
\column{.5\textwidth} 
\tableofcontents[sections={10-16},currentsection, currentsubsection]
\end{columns}
\end{small}
\end{frame}
%%%%%%%%%%%%%%%%%%%%%%%%%%%%%%%%%%%%%%%%%%%%%%%%%%%%%%%%%%%%%%%%%%%%%%%%
%%%%%%%%%%%%%%%%%%%%%%%%%%%%%%%%%%%%%%%%%%%%%%%%%%%%%%%%%%%%%%%%%%%%%%%%
\begin{frame}
\frametitle{Geometry}
\begin{block}{}

\begin{figure}
\includegraphics[width=1\textwidth]{PICTURES/Obstacle.pdf}
%\caption{Obstacle}
\end{figure}

\begin{itemize}
\item Fluid: $\mu=3.7 \, 10^{-5} kg.m^{-1}.s^{-1}$, $\rho=2 kg.m^{-3}$ and $Re=\frac{U_0 H_{inlet} \rho}{\mu} = \frac{1 \times 0.22 \times 2}{3.7 \;10^{-5}} = 11 891$
\item Boundary conditions:\\
    \begin{itemize}
    \item [$\circ$] Inlet with uniform velocity: $U_0=1 m.s^{-1}$
    \item [$\circ$] Outlet with constant pressure: $P_0=0$
    \item [$\circ$] Square cylinder: No-slip wall
    \item [$\circ$] Upper and Lower walls: Symmetry
    \end{itemize}
\end{itemize}

\end{block}
\end{frame}
%%%%%%%%%%%%%%%%%%%%%%%%%%%%%%%%%%%%%%%%%%%%%%%%%%%%%%%%%%%%%%%%%%%%%%%%

%%%%%%%%%%%%%%%%%%%%%%%%%%%%%%%%%%%%%%%%%%%%%%%%%%%%%%%%%%%%%%%%%%%%%%%%
\begin{frame}
\frametitle{Create a study}
\begin{block}{}

\begin{itemize}
\item First, load TRUST environment.
\vspace{0.2cm}

\item Open a terminal and create a directory for this tutorial: \\
\textbf{mkdir -p \; Formation\_TRUST/yourname}\\
\textbf{cd \; Formation\_TRUST/yourname}
\vspace{0.2cm}

\item \label{method_copy} Copy a test case from TRUST's database with:\\
\textbf{trust -copy\index{trust -copy} Obstacle}\\
\textbf{cd Obstacle}
\vspace{0.2cm}

\item Ask for trust script options:\\
\textbf{trust -help\index{trust -help}}\\
\vspace{0.2cm}

\item Ask for help on the options of TRUST's executable:\\
\textbf{trust Obstacle -help\_trust\index{trust -help\_trust}}\\
\vspace{0.2cm}

\item Run the test case with:\\
\textbf{trust Obstacle}\\

\end{itemize}

\end{block}
\end{frame}
%%%%%%%%%%%%%%%%%%%%%%%%%%%%%%%%%%%%%%%%%%%%%%%%%%%%%%%%%%%%%%%%%%%%%%%%
%%%%%%%%%%%%%%%%%%%%%%%%%%%%%%%%%%%%%%%%%%%%%%%%%%%%%%%%%%%%%%%%%%%%%%%%
\begin{frame}
\frametitle{Probes and parameters}
\begin{block}{}

\begin{itemize}
\item Edit the data file Obstacle.data and set the time step to 0.004s:\\
\textbf{gedit Obstacle.data \&}\\

\item Replace the keyword "\textbf{format lml}" with "\textbf{format lata\index{format lata}}" inside the post-processing block in order to use VisIt post-processing tool during and/or at the end of calculation.
\end{itemize}

\end{block}
\end{frame}
%%%%%%%%%%%%%%%%%%%%%%%%%%%%%%%%%%%%%%%%%%%%%%%%%%%%%%%%%%%%%%%%%%%%%%%%
%%%%%%%%%%%%%%%%%%%%%%%%%%%%%%%%%%%%%%%%%%%%%%%%%%%%%%%%%%%%%%%%%%%%%%%%
\begin{frame}
\frametitle{Visualization during the calculation}
\begin{block}{}

\begin{itemize}

\item Launch the "trust -evol" tool with: \\
\textbf{trust -evol\index{trust -evol} Obstacle \&} \\
This tool allows to launch calculation and visualize results

\item To run the calculation, click on the button "Start computation!".

\item Visualization:
    \begin{itemize}
    \item [$\circ$] Select "PRESSION(X=0.13,Y=0.105)" in the left list  and click on "Plot" to draw the evolution of the pressure at the probe location.
    \item [$\circ$] Check the velocity profile behind the square cylinder by plotting "VITESSE\_X(X=0.14,Y=0.115)" and "VITESSE\_Y(X=0.14,Y=0.115)".
    \item [$\circ$] Visualize the equation residuals on the same plot, select "$ Ri = max \left| \frac{dV}{dt} \right|$" and "$residu = max \left| Ri \right|$" using the button "Plot on same" or select the two graphs with "Ctrl" button then "plot".
    \end{itemize}

%\item Check the convergence monitoring: plot 0 in the first menu to open the convergence menu, then open the 4 following plots:
%    \begin{itemize}
%    \item [$\circ$] Pressure linear system convergence at each time step
%    \item [$\circ$] Residual error $\left\Vert Ax-B \right\Vert$ of the pressure linear system at the last time step
%    \item [$\circ$] Time step evolution
%    \item [$\circ$] Flow rate error evolution
%    \end{itemize}
%Type "5" to quit this menu and return to the first one.
%\vspace{0.1cm}
%\item Open the menu "Surface fluxes characteristics" (plot1), visualize:
 %   \begin{itemize}
 %   \item [$\circ$] the drag exerted by the fluid on the square Cylinder (plot 1, then 4 for the square), close this menu with 7.
%    \item [$\circ$] the "Pressure drag + Friction drag" with 5 and select the Square boundary (4). Each drag is plotted together with the total drag also called viscous drag.
%    \end{itemize} 
\end{itemize}

\end{block}
\end{frame}
%%%%%%%%%%%%%%%%%%%%%%%%%%%%%%%%%%%%%%%%%%%%%%%%%%%%%%%%%%%%%%%%%%%%%%%%
%%%%%%%%%%%%%%%%%%%%%%%%%%%%%%%%%%%%%%%%%%%%%%%%%%%%%%%%%%%%%%%%%%%%%%%%
\begin{frame}
\frametitle{VisIt}
\begin{block}{}

\begin{itemize}
\item To quit this tool, close the GUI.

\item Once the calculation is finished, visualize the results with VisIt: \textbf{visit\index{visit} \&} \\
or using "trust -evol" tool: \textbf{trust -evol\index{trust -evol} Obstacle \&} and click on "Visualisation" on the right menu.

    \begin{itemize}
    \item [$\circ$] First, we are going to configure VisIt: {\footnotesize{In the menu File $\rightarrow$ Open file, select Off instead of Smart for File grouping option. For the Filter, specify *.lata to list only the lata files (results). Then save your choices, in the menu Options $\rightarrow$ Save Settings.}}

    \item [$\circ$] In the menu File $\rightarrow$ Open file, select the Obstacle.lata file.

    \item [$\circ$] Visualize the mesh in the "Plots" area with "Add $\rightarrow$ Mesh $\rightarrow$ dom" then click on the button "Draw". Zoom and move the mesh in the right window. You can un-zoom with right button (View $\rightarrow$ Reset view) or with a combination of "Ctrl" keypad and left button.

    \item [$\circ$] Visualize the pressure field ({\footnotesize{Plots area: "Add $\rightarrow$ Pseudocolor $\rightarrow$ PRESSION\_SOM\_dom" + Draw then select the last time on the Time slider}})
    \item [$\circ$] Suppress or hide the mesh (Select Mesh then click on Delete or Hide/Show).
    \end{itemize}
\end{itemize}

\end{block}
\end{frame}
%%%%%%%%%%%%%%%%%%%%%%%%%%%%%%%%%%%%%%%%%%%%%%%%%%%%%%%%%%%%%%%%%%%%%%%%
%%%%%%%%%%%%%%%%%%%%%%%%%%%%%%%%%%%%%%%%%%%%%%%%%%%%%%%%%%%%%%%%%%%%%%%%
\begin{frame}
\frametitle{VisIt}
\begin{block}{}

\hspace{1cm} $\color{darkblue}\circ$ {\small{Visualize the velocity field ({\footnotesize{Plots area: "Add $\rightarrow$ Vector $\rightarrow$ VITESSE\_SOM\_dom"}}\\
\hspace{1.3cm} {\footnotesize{ + Draw)}}. You can change each plot attributes:}}\\
\vspace{0.05cm}
\hspace{1.5cm} $\color{darkblue}\diamond$ {\footnotesize{click once onto the small arrow "$\blacktriangleright$" then }}\\
\hspace{1.5cm} $\color{darkblue}\diamond$ {\footnotesize{double click on the item Vector (cf the figure below). For example, change \\
\hspace{1.8cm} the number of vectors being plotted (by default 400, set it to 40000 \\
\hspace{1.8cm} then click the button "Make default" and save definitively this modification with \\
\hspace{1.8cm} the menu Options $\rightarrow$ Save Settings). You need to click "Apply" to update. Then \\
\hspace{1.8cm} click "Dismiss" to close the window.}}\\

\begin{figure}
\includegraphics[width=0.45\textwidth]{PICTURES/visit3.jpg}
%\caption{Obstacle}
\end{figure}

\hspace{1cm} $\color{darkblue}\circ$ {\small{Print your visualization (File $\rightarrow$ Save window): a PNG file is created into your \\
\hspace{1.3cm} working directory.}}

\end{block}
\end{frame}
%%%%%%%%%%%%%%%%%%%%%%%%%%%%%%%%%%%%%%%%%%%%%%%%%%%%%%%%%%%%%%%%%%%%%%%%
%%%%%%%%%%%%%%%%%%%%%%%%%%%%%%%%%%%%%%%%%%%%%%%%%%%%%%%%%%%%%%%%%%%%%%%%
\begin{frame}
\frametitle{VisIt}
\begin{block}{}

\hspace{1cm} $\color{darkblue}\circ$ {\small{Add a second screen with "Windows $\rightarrow$ Layouts $\rightarrow$ 1x2",}}\\

\hspace{1cm} $\color{darkblue}\circ$ {\small{Plot a pressure horizontal profile:\\}}
\hspace{1.5cm} $\color{darkblue}\diamond$ {\footnotesize{ select the pressure field,}} \\ 
\hspace{1.5cm} $\color{darkblue}\diamond$ {\footnotesize{ on the visualisation, use the right click and select "Mode $\rightarrow$ Line out",}}\\
\hspace{1.5cm} $\color{darkblue}\diamond$ {\footnotesize{ then define your profile with left button,}} \\
\hspace{1.5cm} $\color{darkblue}\diamond$ {\footnotesize{ click on the origin point, let the left button pushed, and release at the end point.}}\\
\hspace{1.5cm} $\color{darkblue}\diamond$ {\footnotesize{ The profile is shown on the second window.}}\\

\hspace{1cm} $\color{darkblue}\circ$ {\small{You notice that it is necessary to update (button Draw) the right window after \\
\hspace{1.3cm} adding a new plot or changing an option. It is possible to automatically \\
\hspace{1.3cm} update by activating "Auto apply" on the top right of the VisIt's GUI.}}\\

\hspace{1cm} $\color{darkblue}\circ$ {\small{You can create create new fields (expression) with "Controls $\rightarrow$ Expressions \\
\hspace{1.3cm} $\rightarrow$ New" by using existing variables and complex functions and visualize it.}}\\

\hspace{1cm} $\color{darkblue}\circ$ {\small{You can animate your visualization and/or create a movie (File $\rightarrow$ Save movie)}}\\

\hspace{1cm} $\color{darkblue}\circ$ {\small{You can operate calculations on variables with complex queries \\
\hspace{1.3cm} (Controls $\rightarrow$ Query),}}\\

\hspace{1cm} $\color{darkblue}\circ$ {\small{You can save a complex session (File $\rightarrow$ Save session) and reopen it during a \\
\hspace{1.3cm} next analyze with VisIt (File $\rightarrow$ Restore session),}}



\end{block}
\end{frame}
%%%%%%%%%%%%%%%%%%%%%%%%%%%%%%%%%%%%%%%%%%%%%%%%%%%%%%%%%%%%%%%%%%%%%%%%
%%%%%%%%%%%%%%%%%%%%%%%%%%%%%%%%%%%%%%%%%%%%%%%%%%%%%%%%%%%%%%%%%%%%%%%%
\begin{frame}
\frametitle{Outputs and resuming calculation}
\begin{block}{}

\hspace{1cm} $\color{darkblue}\circ$ {\small{During a 3D visualization, you will use one of the available Operators \\
\hspace{1.3cm} (In Plots, "Operators $\rightarrow$ Slicing $\rightarrow$ Slice") to create a 2D slice either in a 3D \\
\hspace{1.3cm} space, or projected to a 2D space.}}

\begin{itemize}
\item For more information on \textbf{VisIt}, you can refer to: 
    \begin{itemize}
    \item [$\circ$] the \textbf{VisIt} website and its manuals: \textcolor{blue}{https://wci.llnl.gov/simulation/computer-codes/visit/manuals}
    \item [$\circ$] the \textbf{VisIt} user community web site: \textcolor{blue}{http://visitusers.org}
    \item [$\circ$] or send an email to the VisIt software users community at: \textcolor{blue}{visit-users@elist.ornl.gov}
    \end{itemize}

\item Edit the different output (*.out) files to read the complete balances (mass, stress, energy, ...) on the whole domain or at the boundaries.

\item Now we want to edit the data file in order to resume the calculation. So, open it using "evol" tool: \textbf{trust -evol\index{trust -evol} Obstacle \&}.
\end{itemize}

\end{block}
\end{frame}
%%%%%%%%%%%%%%%%%%%%%%%%%%%%%%%%%%%%%%%%%%%%%%%%%%%%%%%%%%%%%%%%%%%%%%%%
%%%%%%%%%%%%%%%%%%%%%%%%%%%%%%%%%%%%%%%%%%%%%%%%%%%%%%%%%%%%%%%%%%%%%%%%
\begin{frame}
\frametitle{Outputs and resuming calculation}
\begin{block}{}

\begin{itemize}
\item Find the last backup time of the previous calculation in the .err file (or in the bottom right file in the "evol" tool if it is still running).

\item \label{save_restart} Edit your data file with "Edit data", then modify \textbf{tinit\index{tinit}}, \textbf{tmax\index{tmax}} values in the object "mon\_schema".

\item Add in the problem description block just before the last "\}":\\
\textbf{reprise binaire\index{reprise binaire} Obstacle\_pb.sauv}\\
(The file "Obstacle\_pb.sauv" must have been created during the first run.)

\item Save and close the window.

\item Resume the calculation again with "Start calculation!" button. You can see that values are added to the first probes during the new calculation.

\item [$\color{green}\Rightarrow$] \textit{Remark:} to resume your calculation, you can also use the keyword \textbf{resume\_last\_time\index{resume\_last\_time}} instead of \textbf{reprise} and only change the \textbf{tmax\index{tmax}} value (cf Reference Manual).
\end{itemize}

\end{block}
\end{frame}
%%%%%%%%%%%%%%%%%%%%%%%%%%%%%%%%%%%%%%%%%%%%%%%%%%%%%%%%%%%%%%%%%%%%%%%%

%%%%%%%%%%%%%%%%%%%%%%%%%%%%%%%%%%%%%%%%%%%%%%%%%%%%%%%%%%%%%%%%%%%%%%%%
%%%%%%%%%%%%%%%%%%%%%%%%%%%%%%%%%%%%%%%%%%%%%%%%%%%%%%%%%%%%%%%%%%%%%%%%
\begin{frame}
\frametitle{Probes and fields}
\begin{block}{}

\begin{itemize}
\item Edit the data file Obstacle.data:\\
\textbf{gedit Obstacle.data \&}\\

\item Add to the post-processing block of Obstacle.data the following elements:\\

    \begin{itemize}
    \item [$\circ$] A pressure probes segment (22 probes between points (0.01, 0.12) and (0.91, 0.12)).
    \vspace{0.1cm}

    \item [$\circ$] A velocity probes segment (22 probes between points (0.92, 0.00) and (0.92, 0.22)) to plot the velocity profile behind the square cylinder.
    \vspace{0.1cm}

    \item [$\circ$] Change fields post-processing period from 1s to 0.5s.
    \vspace{0.1cm}
    
     \item [$\circ$] Add the vorticity to the fields to the list of post-processed fields. To find the appropriate keyword, have a look to the Generic Guide:\\
    \textbf{trust -doc \&\index{trust -doc \&}}
    \vspace{0.1cm}
    \item [$\rightarrow$] You have access to useful resources in the TRUST index. Take few minutes to find test cases containing a particular keyword using the \underline{Keywords} link: \\
\textbf{trust -index \&\index{trust -index}}\\
\vspace{0.1cm}
    
    \end{itemize}

\end{itemize}

\end{block}
\end{frame}


\subsection{Parallel calculation} \label{exo_para_1}
%%%%%%%%%%%%%%%%%%%%%%%%%%%%%%%%%%%%%%%%%%%%%%%%%%%%%%%%%%%%%%%%%%%%%%%%
\begin{frame}
\begin{small}
\begin{columns}[c] 
\column{.45\textwidth}
\tableofcontents[sections={1-9},currentsection, currentsubsection]
\column{.5\textwidth} 
\tableofcontents[sections={10-16},currentsection, currentsubsection]
\end{columns}
\end{small}
\end{frame}
%%%%%%%%%%%%%%%%%%%%%%%%%%%%%%%%%%%%%%%%%%%%%%%%%%%%%%%%%%%%%%%%%%%%%%%%
%%%%%%%%%%%%%%%%%%%%%%%%%%%%%%%%%%%%%%%%%%%%%%%%%%%%%%%%%%%%%%%%%%%%%%%%
\begin{frame}
\frametitle{Parallel calculation}
\begin{block}{}

The goal of this exercise is to introduce parallelism in the data file of the previous exercise.

\begin{itemize}
\item Go to the previous study (should be done) and after you had suppressed the \textbf{reprise} keyword and set \textbf{tinit\index{tinit}} to 0 again in the \textit{Obstacle.data} file, create two new files:\\
\textbf{cd Formation\_TRUST/yourname/Obstacle} \\
\textbf{mkdir PARA1} \\
\textbf{cd PARA1} \\
\textbf{cp ../Obstacle.data DEC\_Obstacle.data} \\
\textbf{cp ../Obstacle.data PAR\_Obstacle.data} \\
\textbf{cp ../Obstacle.geo .}\\
\vspace{0.3cm}
\item Edit the first file (\textit{DEC\_Obstacle.data}) to create the partition of the mesh.
\end{itemize}

\end{block}
\end{frame}
%%%%%%%%%%%%%%%%%%%%%%%%%%%%%%%%%%%%%%%%%%%%%%%%%%%%%%%%%%%%%%%%%%%%%%%%
%%%%%%%%%%%%%%%%%%%%%%%%%%%%%%%%%%%%%%%%%%%%%%%%%%%%%%%%%%%%%%%%%%%%%%%%
\begin{frame}
\frametitle{Parallel calculation}
\begin{block}{}

\begin{itemize}
\item In this file, uncomment the block around the \textbf{Partition} keyword.
    \begin{itemize}
    \item [$\circ$] Here, the partitioning tool \textbf{Metis\index{Metis}} is used. We cut in \textbf{nb\_parts\index{nb\_parts}} blocks, here in 2.
    \item [$\circ$] The overlapping width \textbf{Larg\_joint\index{Larg\_joint}} between two parts of the partition should be defined according to the numerical scheme higher order, generally the convective scheme. Its value is generally 1 for a second-order scheme, and 2 for third- or fourth-order schemes such as Quick\index{quick} scheme.
    \item [$\circ$] In VEF, you should use \textbf{2} for \textbf{Larg\_joint\index{Larg\_joint}} except when partitioning a domain where only the conduction equation will be solved.
    \item [$\circ$] At least, the keyword \textbf{zones\_name\index{zones\_name}} is useful to define the name of the files containing the partitioned mesh and to write these files.
    \item [$\circ$] Notice the presence of the keyword \textbf{End} in the "Partition" block: the code will stop reading the data file at this line!
    \end{itemize}

\item Run the data file: \textbf{trust DEC\_Obstacle}

\item Check that the partitioned mesh files DOM\_0000.Zones and DOM\_0001.Zones are generated inside your working directory: \textbf{ls *.Zones}
\end{itemize}

\end{block}
\end{frame}
%%%%%%%%%%%%%%%%%%%%%%%%%%%%%%%%%%%%%%%%%%%%%%%%%%%%%%%%%%%%%%%%%%%%%%%%
%%%%%%%%%%%%%%%%%%%%%%%%%%%%%%%%%%%%%%%%%%%%%%%%%%%%%%%%%%%%%%%%%%%%%%%%
\begin{frame}
\frametitle{Parallel calculation}
\begin{block}{}

\begin{itemize}
\item Now, edit the file PAR\_Obstacle.data and comment the read of the mesh
(using \# tags of the 'BEGIN/END MESH' comments).

\item Uncomment the \textbf{Scatter} keyword which will read the partitioned mesh. 

%\item Visualize partitioning with:\\
%\textbf{trust -mesh\index{trust -mesh} PAR\_Obstacle}

\item Now, run a parallel calculation with TRUST:\\
\textbf{trust PAR\_Obstacle 2}

\item Post-processing step is identical in sequential or parallel modes. Probes are written into .son files and fields into .lata files. Run VisIt with:\\
\textbf{visit\index{visit} -o PAR\_Obstacle.lata \&}

\item Select the last time step and visualize the blocks (with Plots: Add $\rightarrow$ Subset $\rightarrow$ blocks)
which represent the parts of the domain partition, then the velocity fields. You can
also visualize a field only on a selected part (block) with the menu Control $\rightarrow$ Subset.

\item Visualize probes at the end of the calculation using:\\
\textbf{trust -evol\index{trust -evol} PAR\_Obstacle \&}

\end{itemize}

\end{block}
\end{frame}
%%%%%%%%%%%%%%%%%%%%%%%%%%%%%%%%%%%%%%%%%%%%%%%%%%%%%%%%%%%%%%%%%%%%%%%%
%%%%%%%%%%%%%%%%%%%%%%%%%%%%%%%%%%%%%%%%%%%%%%%%%%%%%%%%%%%%%%%%%%%%%%%%
%\begin{frame}
%\frametitle{Parallel calculation: 2nd method} \label{exo_para_2}
%\begin{block}{}
%
%\begin{itemize}
%
%\item The existing tool \textbf{trust -partition\index{trust -partition}} is useful on the data files \underline{which have} the marks MAILLAGE/MESH, DECOUPAGE/PARTITION and LECTURE/SCATTER. If you run the following commands: \\
%\vspace{0.2cm}
%
%\item First we will create a new working directory with our data files:\\
%\textbf{cd Formation\_TRUST/yourname/Obstacle} \\
%\textbf{mkdir PARA2} \\
%\textbf{cd PARA2} \\
%\textbf{cp ../Obstacle.data exemple.data}\\
%\textbf{cp ../Obstacle.geo .}\\
%\vspace{0.2cm}
%
%\item Then we run the command:\\
%\textbf{trust -partition\index{trust -partition} exemple 3}\\
%\textbf{ls}
%\end{itemize}
%
%\end{block}
%\end{frame}
%%%%%%%%%%%%%%%%%%%%%%%%%%%%%%%%%%%%%%%%%%%%%%%%%%%%%%%%%%%%%%%%%%%%%%%%%
%%%%%%%%%%%%%%%%%%%%%%%%%%%%%%%%%%%%%%%%%%%%%%%%%%%%%%%%%%%%%%%%%%%%%%%%%
%\begin{frame}
%\frametitle{Parallel calculation: 2nd method}
%\begin{block}{}
%
%\begin{itemize}
%\item It creates:
%    \begin{itemize} 
%    \item [$\circ$] a \underline{SEQ\_exemple.data} file which is a copy of the sequential data file exemple.data,
%    \item [$\circ$] a \underline{DEC\_exemple.data} file which is the first data file to be run. It is immediately run by the command line \textbf{trust -partition} to create a partition (with 3 sub zones here), located in the ***.Zones files. \\
%    \textbf{ls *.Zones}\\
%    Note that the TRUST code stops reading this file at the keyword "End" just before the "\# END PARTITION \#" block.
%    \item [$\circ$] a \underline{PAR\_exemple.data} file which is the data file for the parallel calculation. It uses the ***.Zones files to read the mesh through the line "Scatter DOM.Zones dom". Note that the meshing and cut of the mesh are commented here.
%    \end{itemize}
%
%\item Then you have to run the calculation by the usual command completed by the number of processors needed:\\
%\textbf{trust PAR\_exemple 3}
%\end{itemize}
%
%\end{block}
%\end{frame}
%%%%%%%%%%%%%%%%%%%%%%%%%%%%%%%%%%%%%%%%%%%%%%%%%%%%%%%%%%%%%%%%%%%%%%%%%
%%%%%%%%%%%%%%%%%%%%%%%%%%%%%%%%%%%%%%%%%%%%%%%%%%%%%%%%%%%%%%%%%%%%%%%%%
%\begin{frame}
%\frametitle{Parallel calculation: 2nd method}
%\begin{block}{Useful information:}
%
%\begin{itemize}
%\item Be careful when you want to modify your data file! You have two possibilities:
%    \begin{itemize} 
%    \item [$\circ$] you want to modify your mesh,
%    \item [$\circ$] you want to modify the calculation parameters.
%    \end{itemize}
%
%\item For the first one, you can modify:
%    \begin{itemize} 
%    \item [$\circ$] the file exemple.data and run \textbf{trust -partition}. But it will \underline{erase} the DEC\_exemple.data, SEQ\_exemple.data and PAR\_exemple.data files and create new zones. Then it will run the new DEC\_exemple.data file which gives your new ***.Zones files \underline{or},
%    \item [$\circ$] the meshing part of file DEC\_exemple.data and run it with:\\
%    \textbf{trust DEC\_exemple.data} \\
%    \end{itemize}
%Then run the parallel calculation normally, on the new ***.Zones files.\\
%\textbf{trust PAR\_exemple 3}
%
%\end{itemize}
%
%\end{block}
%\end{frame}
%%%%%%%%%%%%%%%%%%%%%%%%%%%%%%%%%%%%%%%%%%%%%%%%%%%%%%%%%%%%%%%%%%%%%%%%%
%%%%%%%%%%%%%%%%%%%%%%%%%%%%%%%%%%%%%%%%%%%%%%%%%%%%%%%%%%%%%%%%%%%%%%%%%
%\begin{frame}
%\frametitle{Parallel calculation: 2nd method}
%\begin{block}{}
%
%\begin{itemize}
%\item For the second possibility, you can modify:
%    \begin{itemize}
%    \item [$\circ$] the file exemple.data and run \textbf{trust -partition}. But it will \underline{erase} the DEC\_exemple.data, SEQ\_exemple.data and PAR\_exemple.data files and create new ones. Then it will run the new DEC\_exemple.data file. \underline{Note that in that} \underline{case, you don't need to re-create the mesh so you can use the second point} \underline{below:}
%    \item [$\circ$] modify the PAR\_exemple.data file \underline{without} running \textbf{trust -partition}.
%    \end{itemize}
%Then run the PAR\_exemple.data file with:\\
%\textbf{trust PAR\_exemple 3}
%
%\item \underline{Notice that if after a certain time, you want to reopen an old case} and understand what you did in it without any doubts, you can create two files manually:
%    \begin{itemize} 
%    \item [$\circ$] one "BuildMeshes.data" file only for the mesh and the cut of the mesh, and
%    \item [$\circ$] one "calculation.data" file for the parallel calculation.
%    \end{itemize}
%You will run it like:\\
%\textbf{trust BuildMeshes}\\
%\textbf{trust calculation nb\_procs}\\
%\end{itemize}
%
%\end{block}
%\end{frame}
%%%%%%%%%%%%%%%%%%%%%%%%%%%%%%%%%%%%%%%%%%%%%%%%%%%%%%%%%%%%%%%%%%%%%%%%



\subsection{Parallel calculation on a cluster} \label{exo_para_3}
%%%%%%%%%%%%%%%%%%%%%%%%%%%%%%%%%%%%%%%%%%%%%%%%%%%%%%%%%%%%%%%%%%%%%%%%
\begin{frame}
\begin{small}
\begin{columns}[c] 
\column{.45\textwidth}
\tableofcontents[sections={1-9},currentsection, currentsubsection]
\column{.5\textwidth} 
\tableofcontents[sections={10-16},currentsection, currentsubsection]
\end{columns}
\end{small}
\end{frame}
%%%%%%%%%%%%%%%%%%%%%%%%%%%%%%%%%%%%%%%%%%%%%%%%%%%%%%%%%%%%%%%%%%%%%%%%
%%%%%%%%%%%%%%%%%%%%%%%%%%%%%%%%%%%%%%%%%%%%%%%%%%%%%%%%%%%%%%%%%%%%%%%%
\begin{frame}
\frametitle{Parallel calculation on a cluster}
\begin{block}{}
NB: On CEA Clusters, TRUST is already installed and the procedure of launching calculation is described below. Out of CEA, your cluster administrator should install and configure TRUST. In addition, submission files and procedure depend on the cluster itself and could be different from those presented below.
\begin{itemize}
\item Login to the CEA cluster "orcus" and initialize the TRUST environment: \\
{\small{
\textbf{ssh -X yourlogin@orcusloginint1(.intra.cea.fr)} or\\
\textbf{ssh -X yourlogin@orcusloginamd1(.intra.cea.fr)} \\
\textbf{source /home/trust\_trio-public/env\_TRUST-1.9.5.sh}\\
}}

\item Copy the study Obstacle:\\
\textbf{cd \$SCRATCH}\\
\textbf{mkdir  -p  Formation\_TRUST/yourname} \\
\textbf{cd Formation\_TRUST/yourname} \\
\textbf{trust -copy\index{trust -copy} Obstacle} \\
\textbf{cd Obstacle} \\

\item Open Obstacle.data and set the \textbf{format} to \textbf{lata}\index{format lata} in the post-processing block.

\end{itemize}

\end{block}
\end{frame}
%%%%%%%%%%%%%%%%%%%%%%%%%%%%%%%%%%%%%%%%%%%%%%%%%%%%%%%%%%%%%%%%%%%%%%%
%%%%%%%%%%%%%%%%%%%%%%%%%%%%%%%%%%%%%%%%%%%%%%%%%%%%%%%%%%%%%%%%%%%%%%%%
\begin{frame}
\frametitle{Parallel calculation on a cluster}
\begin{block}{}

\begin{itemize}
\item Partition mesh and create a parallel data file with:\\
\textbf{trust -partition\index{trust -partition} Obstacle}

\item For clusters, you have to create a submission file:\\
\textbf{trust -create\_sub\_file\index{trust -create\_sub\_file} PAR\_Obstacle 2}

\item Open the file sub\_file and rename the job. Note that we will see only the first eight characters of the job name in the submitted jobs list.

\item Submit the job with: 
\textbf{sbatch sub\_file}

\item Check job status with: \textbf{"squeue"} or \textbf{"squeue -u yourlogin"}

\item To vizualize your results, use TurboVNC as described on Users training presentation.

\end{itemize}

\end{block}
\end{frame}
%%%%%%%%%%%%%%%%%%%%%%%%%%%%%%%%%%%%%%%%%%%%%%%%%%%%%%%%%%%%%%%%%%%%%%%%



\section{{\bf{Heat transfer (2D, VDF/VEF)}}} \label{docond_pb_couple}
%%%%%%%%%%%%%%%%%%%%%%%%%%%%%%%%%%%%%%%%%%%%%%%%%%%%%%%%%%%%%%%%%%%%%%%%
\begin{frame}
\begin{small}
\begin{columns}[c] 
\column{.45\textwidth}
\tableofcontents[sections={1-9},currentsection, currentsubsection]
\column{.5\textwidth} 
\tableofcontents[sections={10-16},currentsection, currentsubsection]
\end{columns}
\end{small}
\end{frame}
%%%%%%%%%%%%%%%%%%%%%%%%%%%%%%%%%%%%%%%%%%%%%%%%%%%%%%%%%%%%%%%%%%%%%%%%
%%%%%%%%%%%%%%%%%%%%%%%%%%%%%%%%%%%%%%%%%%%%%%%%%%%%%%%%%%%%%%%%%%%%%%%%
\begin{frame}
\frametitle{Heat transfer (2D, VDF/VEF)}
\begin{block}{}


\begin{columns}[c] 
\column{.55\textwidth}
%\begin{figure}
\begin{center}
\includegraphics[width=1.\textwidth]{PICTURES/heat_exchange.pdf}
\end{center}
%\caption{Obstacle}
%\end{figure}

\column{.4\textwidth} 
$$
\begin{array}{ll}
\mbox{\textbf{Fluid:}}      & Pr=\frac{\mu C_p}{\lambda}=1, \\
%                            & Ra=20000,\\
                            & T_{ref}=30^o C,\\
                            & \mu=2 \; 10^{-3} kg.m^{-1}.s^{-1},\\
                            & \rho=2 \; kg.m^{-3},\\
                            & \lambda=1. W.m^{-1}.K^{-1} \\
                            & Cp=500 J.kg^{-1}.K^{-1},\\
                            & \beta=1.10^{-4} K^{-1} \\
                            &  \\
\mbox{\textbf{Solid:}}      & \rho=1000 kg.m^{-3} \\
                            & \lambda=250 W.m^{-1}.K^{-1} \\
                            & Cp=100 J.kg^{-1}.K^{-1}
\end{array}
$$

\end{columns}

\end{block}
\end{frame}
%%%%%%%%%%%%%%%%%%%%%%%%%%%%%%%%%%%%%%%%%%%%%%%%%%%%%%%%%%%%%%%%%%%%%%%%
%%%%%%%%%%%%%%%%%%%%%%%%%%%%%%%%%%%%%%%%%%%%%%%%%%%%%%%%%%%%%%%%%%%%%%%%
\begin{frame}
\frametitle{Heat transfer (2D, VDF/VEF)}
\begin{block}{}

\begin{itemize}
\item Load TRUST environment as described on page \pageref{init_trust}
\item Create a new study Coupling\_VDF by copying the docond study:\\
\textbf{cd Formation\_TRUST/yourname} \\
\textbf{trust -copy\index{trust -copy} docond} \\
\textbf{mv docond Coupling\_VDF} \\
\textbf{cd Coupling\_VDF} \\

\item Check the fluid and solid characteristics inside the docond.data file.

\item This coupled problem is constituted by 2 domains of calculation with a mesh of 10x10 cells ($\Delta x = \Delta y = 0.1m$). 

\item Now open your data file with "evol" tool:\\
\textbf{trust -evol\index{trust -evol} docond  \&} \\

\item Click on "Edit data".

\item We want to modify the data file to have the 2 domains on a mesh of 40x40 cells ($\Delta x= \Delta y=0.025m$). 
\end{itemize}

\end{block}
\end{frame}
%%%%%%%%%%%%%%%%%%%%%%%%%%%%%%%%%%%%%%%%%%%%%%%%%%%%%%%%%%%%%%%%%%%%%%%%
%%%%%%%%%%%%%%%%%%%%%%%%%%%%%%%%%%%%%%%%%%%%%%%%%%%%%%%%%%%%%%%%%%%%%%%%
\begin{frame}
\frametitle{Heat transfer (2D, VDF/VEF)}
\begin{block}{}

\begin{itemize}
\item Change the number of nodes for each block like this:\\
First block (Cavite1): 4 11 $\rightarrow$ 13 41 \\
Second block (Cavite2): 8 4 $\rightarrow$ 29 13 \\ 
Third block (Cavite3): 8 8  $\rightarrow$ 29 29 \\

%\item Check your new mesh with:\\
%\textbf{trust -mesh\index{trust -mesh} docond } \\

\item Change \textbf{"format lml\index{format lml}"} to \textbf{"format lata\index{format lata}"} into the two problems definition
\item Click on "Save" and close the window.

\item Run the calculation with "Start computation!" and check the evolution.

\item Then post-process the temperature field with VisIt tool: "Visualization" button. A natural convection cell appears.

\item Change the color tables for the temperature to have the same one on the 2 domains. Close VisIt.

\item We are going to change the discretization of the test case: triangulate the domains with the keyword \textbf{Trianguler\_H\index{Trianguler\_H}} (refer to the Reference Manual).
\end{itemize}

\end{block}
\end{frame}
%%%%%%%%%%%%%%%%%%%%%%%%%%%%%%%%%%%%%%%%%%%%%%%%%%%%%%%%%%%%%%%%%%%%%%%%
%%%%%%%%%%%%%%%%%%%%%%%%%%%%%%%%%%%%%%%%%%%%%%%%%%%%%%%%%%%%%%%%%%%%%%%%
\begin{frame}
\frametitle{Heat transfer (2D, VDF/VEF)}
\begin{block}{}

\begin{itemize}
\item Then give an unstructured aspect to the 2 meshes using the following syntax: \\
\textbf{Transformer\index{Transformer} name\_of\_domain \; x*(1-0.5*y*y) \; y*(1+0.1*x*y)}

\item Substitute the discretization VDF\index{VDF} (pressure nodes at the element center) to VEFPreP1B\index{VEFPreP1B} (pressure nodes at the element's center and nodes).

\item Close the "evol" tool.

%\item Check the meshes with: \\
%\textbf{trust -mesh\index{trust -mesh} docond}

\item Run the calculation with:\\
\textbf{trust docond }

\item Open the IHM:\\
\textbf{trust -evol docond }

\item Select 'Ri=max\_pb1$|dT/dt|$', 'Ri=max\_pb2$|dT/dt|$', 'Ri=max\_pb2$|dV/dt|$', 'residu=max$|Ri|$' with "Ctrl" button and click on 'Plot on same'.

\item To see when convergence is reached, select a probe (for example temperature) and click on 'Plot'. 

\end{itemize}

\end{block}
\end{frame}
%%%%%%%%%%%%%%%%%%%%%%%%%%%%%%%%%%%%%%%%%%%%%%%%%%%%%%%%%%%%%%%%%%%%%%%%
%%%%%%%%%%%%%%%%%%%%%%%%%%%%%%%%%%%%%%%%%%%%%%%%%%%%%%%%%%%%%%%%%%%%%%%%
\begin{frame}
\frametitle{Heat transfer (2D, VDF/VEF)}
\begin{block}{}

\begin{itemize}
\item If the calculation is too long, open the docond.stop file, put a 1 instead the 0 and save.
The calculation will stop after the current time step and make post-process.

\item Post-process the results and compare the CPU performances with VDF\index{VDF} discretization: the VEF calculation is running $\approx$ 10 times slower (because more pressure unknowns and shorter time steps). Check the docond.out file to see the time steps for each equation (click on "Edit .out" at the upper right corner of the GUI).

\item Accelerate the calculation by impliciting the diffusive term of each equation with \textbf{diffusion\_implicite\index{diffusion\_implicite}} option in the explicit Euler scheme (check again the Generic Guide: \textbf{trust -doc \&\index{trust -doc \&}}). 

\item Run the calculation without any option: \\
\textbf{trust docond}

\item \label{schema_impl} Now, use a fully implicit scheme (suppress \textbf{diffusion\_implicite\index{diffusion\_implicite}}), by substituting \textbf{Scheme\_Euler\_Explicit} by \textbf{Scheme\_Euler\_implicit\index{scheme\_euler\_implicit}} and adding the Implicit solver "\textbf{solveur implicite}".

\end{itemize}

\end{block}
\end{frame}
%%%%%%%%%%%%%%%%%%%%%%%%%%%%%%%%%%%%%%%%%%%%%%%%%%%%%%%%%%%%%%%%%%%%%%%%
%%%%%%%%%%%%%%%%%%%%%%%%%%%%%%%%%%%%%%%%%%%%%%%%%%%%%%%%%%%%%%%%%%%%%%%%
\begin{frame}
\frametitle{Heat transfer (2D, VDF/VEF)}
\begin{block}{}

\begin{itemize}
\item Have a look at the Reference Manual for the \textbf{gmres\index{gmres}} options and define, according to the advice given on it, a value for \textbf{facsec\index{facsec}, facsec\_max\index{facsec\_max}}.

\item Your block will look like:\\
\textbf{Solveur Implicite\index{Solveur Implicite} \{ solveur gmres\index{gmres} \{ diag seuil 1e-30 nb\_it\_max 5 impr \}  seuil\_convergence\_implicite\index{seuil\_convergence\_implicite} 0.01  \} }

\item Run the calculation:\\
\textbf{trust -evol\index{trust -evol} docond  \&}
\end{itemize}

\end{block}
\end{frame}
%%%%%%%%%%%%%%%%%%%%%%%%%%%%%%%%%%%%%%%%%%%%%%%%%%%%%%%%%%%%%%%%%%%%%%%%



\section{{\bf{Dilatable flows (2D)}}}
\subsection{Quasi Compressible flow}
%%%%%%%%%%%%%%%%%%%%%%%%%%%%%%%%%%%%%%%%%%%%%%%%%%%%%%%%%%%%%%%%%%%%%%%%
\begin{frame}
\begin{small}
\begin{columns}[c] 
\column{.45\textwidth}
\tableofcontents[sections={1-9},currentsection, currentsubsection]
\column{.5\textwidth} 
\tableofcontents[sections={10-16},currentsection, currentsubsection]
\end{columns}
\end{small}
\end{frame}
%%%%%%%%%%%%%%%%%%%%%%%%%%%%%%%%%%%%%%%%%%%%%%%%%%%%%%%%%%%%%%%%%%%%%%%%
%%%%%%%%%%%%%%%%%%%%%%%%%%%%%%%%%%%%%%%%%%%%%%%%%%%%%%%%%%%%%%%%%%%%%%%%
\begin{frame}
\frametitle{Quasi Compressible flow (2D)}
\begin{block}{}

%\textbf{Add a description of the problem with an image showing domain, BC, ...}

\begin{itemize}
\item Open a terminal and Load TRUST environment as described on page \pageref{init_trust}
\item Copy the study \textbf{TP\_Temp\_QC\_VEF} (it is a 2D simulation of helium gas flow from left to right between two heated walls) as follows:\\
\textbf{mkdir -p Formation\_TRUST/yourname}\\
\textbf{cd Formation\_TRUST/yourname}\\
\textbf{trust -copy\index{trust -copy} TP\_Temp\_QC\_VEF}\\
\textbf{cd TP\_Temp\_QC\_VEF}

\item Open the Generic Guide with (it will be useful to search for keywords in this exercise): \textbf{trust -doc \&\index{trust -doc \&}}

\item Edit the data file with your favorite editor (\textbf{gedit} is recommended because it is configured to recognize the TRUST syntax):\\
\textbf{gedit TP\_Temp\_QC\_VEF.data \&} \\
or \\
\textbf{trust -evol\index{trust -evol} TP\_Temp\_QC\_VEF  \&} and "Edit data" button.
\end{itemize}

\end{block}
\end{frame}

%%%%%%%%%%%%%%%%%%%%%%%%%%%%%%%%%%%%%%%%%%%%%%%%%%%%%%%%%%%%%%%%%%%%%%%%
%%%%%%%%%%%%%%%%%%%%%%%%%%%%%%%%%%%%%%%%%%%%%%%%%%%%%%%%%%%%%%%%%%%%%%%%
\begin{frame}
\frametitle{Quasi Compressible flow (2D)}
\begin{block}{}

\begin{itemize}
\item Edit the data file in order to:
    \begin{itemize}
    \item [$\circ$] Modify the geometry and the mesh:
    \begin{figure}
    \includegraphics[width=0.5\textwidth]{PICTURES/low_mach.pdf}
    %\caption{Obstacle}
    \end{figure}
    \item [$\circ$] Add several probes (velocity, density, temperature) near the upper right corner of the geometry at location (x,y)=(4,1).
    \item [$\circ$] Add a probe \textbf{"segment"} (with 9 points) between the locations (x,y)=(4,0.05) and (x,y)=(4,0.95) for the temperature field.
    \item [$\circ$] Write the results on the \textbf{lata} format and change the \textbf{dt\_post\index{dt\_post}} period to 1s.
    \item [$\circ$] We are looking for the steady state, so suppress tmax\index{tmax} keyword and change the \textbf{seuil\_statio\index{seuil\_statio}} $\varepsilon$ value to 10 ($|dT/dt|<\varepsilon$ and $dt \sim 0.001s$ so $|dT|<0.01$).
    \item [$\circ$] Add the keyword \textbf{impr} into the pressure solver to print its convergence.
    \item [$\circ$] If you use "evol" tool, save and close the editor.
    \end{itemize}

\item Run the simulation with the TRUST command:\\
\textbf{trust -evol\index{trust -evol} TP\_Temp\_QC\_VEF \&} \\
\end{itemize}

\end{block}
\end{frame}
%%%%%%%%%%%%%%%%%%%%%%%%%%%%%%%%%%%%%%%%%%%%%%%%%%%%%%%%%%%%%%%%%%%%%%%%
%%%%%%%%%%%%%%%%%%%%%%%%%%%%%%%%%%%%%%%%%%%%%%%%%%%%%%%%%%%%%%%%%%%%%%%%
\begin{frame}
\frametitle{Quasi Compressible flow (2D)}
\begin{block}{}

\begin{itemize}
\item Click on "Start computation!".
\item Check mass flow rate (absolute and relative values) in the TP\_Temp\_QC\_VEF.out file: \\
\textbf{gedit TP\_Temp\_QC\_VEF.out \&} \\
or look at the upper small window on the right of the "evol" tool.

\item Once the calculation finishes, visualize the results by running \textbf{VisIt}:\\
\textbf{visit\index{visit} -o TP\_Temp\_QC\_VEF.lata \&}\\
or \\
"Visualization" button on "evol" tool.
\begin{itemize}
    \item [$\circ$] Show the mesh (Plots: "Add $\rightarrow$ Mesh $\rightarrow$ dom $\rightarrow$ Draw").
    \item [$\circ$] Visualize the temperature field (Select the last Time with the slicer, then Plots: "Add $\rightarrow$ Pseudo Color $\rightarrow$ TEMPERATURE\_SOM\_dom  $\rightarrow$ Draw").
    \item [$\circ$] Suppress or hide the mesh (Select "Mesh-dom" in the list of plots then "Delete" or "Hide/Show").
    \item [$\circ$] Visualize the velocity field (Add $\rightarrow$ Vector $\rightarrow$ VITESSE\_SOM\_dom $\rightarrow$ Draw).
\end{itemize}

\end{itemize}

\end{block}
\end{frame}
%%%%%%%%%%%%%%%%%%%%%%%%%%%%%%%%%%%%%%%%%%%%%%%%%%%%%%%%%%%%%%%%%%%%%%%%
%%%%%%%%%%%%%%%%%%%%%%%%%%%%%%%%%%%%%%%%%%%%%%%%%%%%%%%%%%%%%%%%%%%%%%%%
\begin{frame}
\frametitle{Quasi Compressible flow (2D)}
\begin{block}{}

\hspace{1cm} $\color{darkblue}\circ$ {\small{Select the Zoom mode with the right button of the mouse (Mode $\rightarrow$ Zoom) \\
\hspace{1.3cm}  then zoom by selecting an area on the plot. To un-zoom push "Ctrl" button \\
\hspace{1.3cm}  and select an area with the left button or with the right button select \\
\hspace{1.3cm}  "View $\rightarrow$ Reset view".}}\\
\hspace{1cm} $\color{darkblue}\circ$ {\small{Print your visualization (File $\rightarrow$ Set Save options $\rightarrow$ File type $\rightarrow$ Select a type \\
\hspace{1.3cm}  $\rightarrow$ Save): a file named visit*** is created into your working directory.}}

\hspace{1cm} $\color{darkblue}\circ$ {\small{Add a second screen with "Window $\rightarrow$ Layout $\rightarrow$ 1x2".}}

\hspace{1cm} $\color{darkblue}\circ$ {\small{Plot a horizontal profile of temperature  (Select the temperature field and \\
\hspace{1.3cm} thanks to the right button, select "Mode $\rightarrow$ Lineout", and define your \\
\hspace{1.3cm} profile with left button): the profile is shown on the second window.}}

\begin{itemize}
\item Substitute the time scheme by an implicit time scheme (like \textbf{scheme\_euler\_implicit\index{scheme\_euler\_implicit}}). 

\item Use the \textbf{implicite} solver and specify \textbf{facsec\index{facsec}} and \textbf{facsec\_max\index{facsec\_max}} parameters according to the advice given on the Reference Manual (search for the \textbf{scheme\_euler\_implicit\index{scheme\_euler\_implicit}} keyword). You can also see the instructions at the end of the Heat transfer VDF/VEF exercise on p.\pageref{schema_impl}.
\end{itemize}

\end{block}
\end{frame}
%%%%%%%%%%%%%%%%%%%%%%%%%%%%%%%%%%%%%%%%%%%%%%%%%%%%%%%%%%%%%%%%%%%%%%%%
%%%%%%%%%%%%%%%%%%%%%%%%%%%%%%%%%%%%%%%%%%%%%%%%%%%%%%%%%%%%%%%%%%%%%%%%
\begin{frame}
\frametitle{Quasi Compressible flow (2D)}
\begin{block}{}

\begin{itemize}
\item Run the calculation with this time scheme using the "evol" tool or: \\
{\footnotesize{\textbf{trust TP\_Temp\_QC\_VEF.data 1$>$TP\_Temp\_QC\_VEF.out 2$>$TP\_Temp\_QC\_VEF.err}}}

\item Edit the file containing information about dt (used time step), dt\_stab (stability time step), \textbf{facsec\index{facsec}} (dt=dt\_stab*facsec) and residuals evolution for each equation: \\
\textbf{gedit TP\_Temp\_QC\_VEF.dt\_ev \&}

\item  If everything is OK, try to enhance the convergence speed of the implicit solver with the value of \textbf{seuil\_convergence\_implicite\index{seuil\_convergence\_implicite}} keyword (look at the TP\_Temp\_QC\_VEF.out file, if the number of iterations for GMRES\index{gmres} is comprised between 3 and 5 then it is enough to converge quickly).

\item  In order to resume a calculation, you will have to change the \textbf{tinit\index{tinit}} value within the data file (pick up the last saved time in the \textbf{.err} file) and insert into the data file, in the problem definition block, the following keywords: \\
\textbf{reprise binaire\index{reprise binaire} TP\_Temp\_QC\_VEF\_pb.sauv}
\end{itemize}

\end{block}
\end{frame}
%%%%%%%%%%%%%%%%%%%%%%%%%%%%%%%%%%%%%%%%%%%%%%%%%%%%%%%%%%%%%%%%%%%%%%%%
%%%%%%%%%%%%%%%%%%%%%%%%%%%%%%%%%%%%%%%%%%%%%%%%%%%%%%%%%%%%%%%%%%%%%%%%
\begin{frame}
\frametitle{Quasi Compressible flow (2D)}
\begin{block}{}

\begin{itemize}

\item  Then run the calculation with: \\
{\footnotesize{\textbf{trust TP\_Temp\_QC\_VEF.data 1$>$TP\_Temp\_QC\_VEF.out 2$>$TP\_Temp\_QC\_VEF.err}}} \\
or \\
\textbf{trust -evol\index{trust -evol} TP\_Temp\_QC\_VEF.data \&} which automatically creates the .out file

\end{itemize}

\end{block}
\end{frame}
%%%%%%%%%%%%%%%%%%%%%%%%%%%%%%%%%%%%%%%%%%%%%%%%%%%%%%%%%%%%%%%%%%%%%%%%

\subsection{Weakly VS Quasi Compressible}
%%%%%%%%%%%%%%%%%%%%%%%%%%%%%%%%%%%%%%%%%%%%%%%%%%%%%%%%%%%%%%%%%%%%%%%%
\begin{frame}
\begin{small}
\begin{columns}[c] 
\column{.45\textwidth}
\tableofcontents[sections={1-9},currentsection, currentsubsection]
\column{.5\textwidth} 
\tableofcontents[sections={10-16},currentsection, currentsubsection]
\end{columns}
\end{small}
\end{frame}
%%%%%%%%%%%%%%%%%%%%%%%%%%%%%%%%%%%%%%%%%%%%%%%%%%%%%%%%%%%%%%%%%%%%%%%%
%%%%%%%%%%%%%%%%%%%%%%%%%%%%%%%%%%%%%%%%%%%%%%%%%%%%%%%%%%%%%%%%%%%%%%%%
\begin{frame}
\frametitle{Weakly Compressible}
\begin{block}{}
If you are interested in the comparison between Quasi Compressible and Weakly Compressible simulations, see the validation form:

\$TRUST\_ROOT/Validation/Rapports\_automatiques/Verification/\\ Verification\_codage/QC\_vs\_WC


\end{block}

\end{frame}

\section{{\bf{Periodic channel flow (3D)}}}
%%%%%%%%%%%%%%%%%%%%%%%%%%%%%%%%%%%%%%%%%%%%%%%%%%%%%%%%%%%%%%%%%%%%%%%%
\begin{frame}
\begin{small}
\begin{columns}[c] 
\column{.45\textwidth}
\tableofcontents[sections={1-9},currentsection, currentsubsection]
\column{.5\textwidth} 
\tableofcontents[sections={10-16},currentsection, currentsubsection]
\end{columns}
\end{small}
\end{frame}
%%%%%%%%%%%%%%%%%%%%%%%%%%%%%%%%%%%%%%%%%%%%%%%%%%%%%%%%%%%%%%%%%%%%%%%%
%%%%%%%%%%%%%%%%%%%%%%%%%%%%%%%%%%%%%%%%%%%%%%%%%%%%%%%%%%%%%%%%%%%%%%%%
\begin{frame}
\frametitle{Periodic channel flow (3D)}
\begin{block}{}

\begin{figure}
\includegraphics[width=0.45\textwidth]{PICTURES/periodic3D.png}
%\caption{Obstacle}
\end{figure}

\textbf{Fluid:} $Re=2000$, $\rho=2 kg.m^{-3}$, $\mu=0.01 kg.m^{-1}.s^{-1}$, initial velocity $V0=1m/s$, periodic boundary condition following the Z-direction.

\begin{itemize}
\item Copy the study named \textbf{P1toP1Bulle} as explained on page \pageref{method_copy}. It simulates a 3D incompressible laminar flow ($Re=2000$) with periodic boundary following the Z-direction only.

\item Open the P1toP1Bulle.data file and use \textbf{RegroupeBord\index{RegroupeBord}} keyword to merge Entree and Sortie boundaries into a single one named periox. 
\end{itemize}

\end{block}
\end{frame}
%%%%%%%%%%%%%%%%%%%%%%%%%%%%%%%%%%%%%%%%%%%%%%%%%%%%%%%%%%%%%%%%%%%%%%%%
%%%%%%%%%%%%%%%%%%%%%%%%%%%%%%%%%%%%%%%%%%%%%%%%%%%%%%%%%%%%%%%%%%%%%%%%
\begin{frame}
\frametitle{Periodic channel flow (3D)}
\begin{block}{}

\begin{itemize}
\item Modify boundary conditions to apply a periodic boundary on the new boundary. 

\item Change the velocity initial condition to $U_0=(1,0,0)$. 

\item Set the option \textbf{diffusion\_implicite\index{diffusion\_implicite}} to 1 into the Euler scheme to implicit the diffusive term in the Navier-Stokes equations.

\item You have now a 3D calculation with periodic boundary conditions on X- and Z-directions. Run the calculation for 30 time-steps (keyword \textbf{nb\_pas\_dt\_max\index{nb\_pas\_dt\_max}}).

\item Have a look at the P1toP1Bulle\_pb\_Debit.out file, check the flow rate on the periox boundary. Why does it decrease?

\item Add the \textbf{Canal\_perio} source term in the Navier-Stokes equations of the data file and run again the calculation to check the flow rate evolution on 30 time steps. 

\item Look at pressure and viscous forces applied on the cylinder inside the .out files.
\end{itemize}

\end{block}
\end{frame}
%%%%%%%%%%%%%%%%%%%%%%%%%%%%%%%%%%%%%%%%%%%%%%%%%%%%%%%%%%%%%%%%%%%%%%%%
%%%%%%%%%%%%%%%%%%%%%%%%%%%%%%%%%%%%%%%%%%%%%%%%%%%%%%%%%%%%%%%%%%%%%%%%
\begin{frame}
\frametitle{Periodic channel flow (3D)}
\begin{block}{}

\begin{itemize}
\item Now, the calculation domain is a rotating channel according to Z direction with a constant velocity $\Omega=1 rad/s$. 

\item Add the \textbf{Acceleration} source term in the Navier-Stokes equations. Suppress the \textbf{nb\_pas\_dt\_max\index{nb\_pas\_dt\_max}} keyword and set \textbf{tmax\index{tmax}} to 100s.

\item Add, if you wish, velocity or statistic calculation to the post-processing instructions. 

\item Run the calculation.

\item You can create a uniformly refined mesh using, for instance, the keyword \textbf{Raffiner\_Anisotrope\index{Raffiner\_Anisotrope}}.

\item Then improve the calculation speed on this mesh, you can use a coarse discretization \textbf{P1} (\textbf{Read dis \{ P1 \}}) with less pressure unknowns. On this latter, it runs 3 times faster than on P1Bulle discretization but it is less accurate: 8452 unknowns compared to 49221 unknowns.
\end{itemize}

\end{block}
\end{frame}
%%%%%%%%%%%%%%%%%%%%%%%%%%%%%%%%%%%%%%%%%%%%%%%%%%%%%%%%%%%%%%%%%%%%%%%%
%%%%%%%%%%%%%%%%%%%%%%%%%%%%%%%%%%%%%%%%%%%%%%%%%%%%%%%%%%%%%%%%%%%%%%%%
\begin{frame}
\frametitle{Periodic channel flow (3D)}
\begin{block}{}

\begin{itemize}
\item Then restart the calculation with \textbf{VEFPreP1B\index{VEFPreP1B}} discretization by reading the velocity field with \textbf{Champ\_fonc\_reprise\index{Champ\_fonc\_reprise}} keyword in the initial conditions for the velocity:\\
\textbf{vitesse champ\_fonc\_reprise\index{Champ\_fonc\_reprise} P1toP1Bulle\_pb.xyz  pb  vitesse  last\_time} \\
\underline{This will be useful to reach the quasi-stationary regime faster.}

\item You can also use implicit scheme (change the scheme to \textbf{Scheme\_Euler\_implicit\index{scheme\_euler\_implicit}} scheme and use an \textbf{Implicite} solver) \underline{only if you are looking for the stationary state}.\\ 
You can also see the instructions at the end of the Heat transfer VDF/VEF exercise on p. \pageref{schema_impl}.
\end{itemize}

\end{block}
\end{frame}
%%%%%%%%%%%%%%%%%%%%%%%%%%%%%%%%%%%%%%%%%%%%%%%%%%%%%%%%%%%%%%%%%%%%%%%%


\section{{\bf{Constituents \& turbulent flow}}}
%%%%%%%%%%%%%%%%%%%%%%%%%%%%%%%%%%%%%%%%%%%%%%%%%%%%%%%%%%%%%%%%%%%%%%%%
\begin{frame}
\begin{small}
\begin{columns}[c] 
\column{.45\textwidth}
\tableofcontents[sections={1-9},currentsection, currentsubsection]
\column{.5\textwidth} 
\tableofcontents[sections={10-16},currentsection, currentsubsection]
\end{columns}
\end{small}
\end{frame}
%%%%%%%%%%%%%%%%%%%%%%%%%%%%%%%%%%%%%%%%%%%%%%%%%%%%%%%%%%%%%%%%%%%%%%%%
%%%%%%%%%%%%%%%%%%%%%%%%%%%%%%%%%%%%%%%%%%%%%%%%%%%%%%%%%%%%%%%%%%%%%%%%
\begin{frame}
\frametitle{Constituents and turbulent flow}
\begin{block}{TrioCFD}

\begin{figure}
\includegraphics[width=0.75\textwidth]{PICTURES/marche.pdf}
%\caption{Obstacle}
\end{figure}

\textbf{Fluid:} $\mu=3.7 \, 10^{-5} kg.m^{-1}.s^{-1}, \rho=2 \, kg.m^{-3}, Re=\frac{U_0 H_{inlet} \rho}{\mu} =54054$

\textbf{Boundary conditions:} \\
Inlet with imposed velocity: $U_0=1 m.s^{-1}$ and constant values of $k=10^{-2}$ and $\varepsilon=10^{-3}$ (dimensionless values)\\
Outlet with constant pressure: $P_0=0$ and constant values of $k=0$ and $\varepsilon=0$ \\
Top and bottom walls: No-slip wall ($U=0$) and $k$ standard flux, $\varepsilon$ null.

\end{block}
\end{frame}
%%%%%%%%%%%%%%%%%%%%%%%%%%%%%%%%%%%%%%%%%%%%%%%%%%%%%%%%%%%%%%%%%%%%%%%%
%%%%%%%%%%%%%%%%%%%%%%%%%%%%%%%%%%%%%%%%%%%%%%%%%%%%%%%%%%%%%%%%%%%%%%%%
\begin{frame}
\frametitle{Constituents and turbulent flow}
\begin{block}{TrioCFD}

\begin{itemize}

\item Initialize TrioCFD full environment to get access to TRUST\&TrioCFD tests.\\
{\small{
\begin{itemize}
\item On CEA Saclay computers:\\
\textbf{source /home/trust\_trio-public/\textcolor{red}{full}\_env\_\textcolor{blue}{TrioCFD}-1.9.5.sh }

\item On your own computer:\\
\textbf{source PathToTrioCFD/\textcolor{red}{full}\_env\_\textcolor{blue}{TrioCFD}.sh}

\end{itemize}
\textbf{echo \$exec} \\
\textbf{echo \$project\_directory} \\
}}
\item Copy the study named \textbf{Marche} (\textcolor{blue}{TrioCFD}) using: \textbf{triocfd -copy Marche}\\
in the directory Formation\_TRUST/yourname. It is also possible to use \textbf{"trust"} script since both commands have the same options and use the same \$exec executable. This test case simulates a 2D incompressible turbulent flow in the above configuration using the k-$\varepsilon$ model.

\item We will add a source of constituent's diffusion, so copy the \textbf{Constituants\index{Constituants}} (2D incompressible laminar flow) study which uses constituents.




\end{itemize}

\end{block}
\end{frame}
%%%%%%%%%%%%%%%%%%%%%%%%%%%%%%%%%%%%%%%%%%%%%%%%%%%%%%%%%%%%%%%%%%%%%%%%
%%%%%%%%%%%%%%%%%%%%%%%%%%%%%%%%%%%%%%%%%%%%%%%%%%%%%%%%%%%%%%%%%%%%%%%%
\begin{frame}
\frametitle{Constituents and turbulent flow}
\begin{block}{TrioCFD}

\begin{itemize}

\item Edit your data file in the Marche directory. First, rename the problem in order to add concentration equations (look for the adequate keywords in the TrioCFD Reference Manual).\\
\textbf{triocfd -index\index{triocfd -index}} then click on \textbf{Reference manual}

\item Add 3 constituents of equal diffusivities ($\alpha=1m/s$) after the fluid definition (in the problem block).

\item Define the concentration equation into the problem (remember that concentrations will be a vector of 3 components) with correct initial ($C_1=0, C_2=0, C_3=0$) and boundary conditions.

\item Use the Schmidt model to close the turbulence model in the concentration equation.

\item Change the sources of the Navier-Stokes turbulence model to a \textbf{Source\_Transport\_K\_Eps\_aniso\_concen\index{Source\_Transport\_K\_Eps\_aniso\_concen} \{ C1\_eps 1.44 C2\_eps 1.92 C3\_eps 1. \} } to fit with the new concentration equation.

\end{itemize}

\end{block}
\end{frame}
%%%%%%%%%%%%%%%%%%%%%%%%%%%%%%%%%%%%%%%%%%%%%%%%%%%%%%%%%%%%%%%%%%%%%%%%
%%%%%%%%%%%%%%%%%%%%%%%%%%%%%%%%%%%%%%%%%%%%%%%%%%%%%%%%%%%%%%%%%%%%%%%%
\begin{frame}
\frametitle{Constituents and turbulent flow}
\begin{block}{TrioCFD}

\begin{itemize}

\item Add into the fluid definition, the volume expansion coefficient for the concentration: \textbf{beta\_co\index{beta\_co}} as a uniform field set to 0.

\item You have also to add a gravity field which can be initialized to 0.

\item Run the calculation to see if it is ok.

\item Define a sub domain (in grey on the previous picture) with the keyword \textbf{Sous\_Zone\index{Sous\_Zone}} (like in \textbf{PCR} data file (\textcolor{blue}{TRUST}) ).

\item Add a source term for the second constituent only ($S_2=1m^{-3}$) applied on the sub domain thanks to the keyword \textbf{Champ\_Uniforme\_Morceaux\index{Champ\_Uniforme\_Morceaux}}.

\item Add \textbf{format lata\index{format lata}} in the post-processing block.

\item Add the keyword \textbf{concentration0, concentration1, concentration2} in the \textbf{fields} of the post-processing block to write the 3 concentrations into the .lata file.

\item Run the calculation and check the results.
\end{itemize}

\end{block}
\end{frame}
%%%%%%%%%%%%%%%%%%%%%%%%%%%%%%%%%%%%%%%%%%%%%%%%%%%%%%%%%%%%%%%%%%%%%%%%



\section{{\bf{Turbulent flow in a curved pipe (3D)}}} \label{prm_para}
%%%%%%%%%%%%%%%%%%%%%%%%%%%%%%%%%%%%%%%%%%%%%%%%%%%%%%%%%%%%%%%%%%%%%%%%
\begin{frame}
\begin{small}
\begin{columns}[c] 
\column{.45\textwidth}
\tableofcontents[sections={1-9},currentsection, currentsubsection]
\column{.5\textwidth} 
\tableofcontents[sections={10-16},currentsection, currentsubsection]
\end{columns}
\end{small}
\end{frame}
%%%%%%%%%%%%%%%%%%%%%%%%%%%%%%%%%%%%%%%%%%%%%%%%%%%%%%%%%%%%%%%%%%%%%%%%
%%%%%%%%%%%%%%%%%%%%%%%%%%%%%%%%%%%%%%%%%%%%%%%%%%%%%%%%%%%%%%%%%%%%%%%%
\begin{frame}
\frametitle{Turbulent flow in a curved pipe (3D)}
\begin{block}{TrioCFD}

\begin{figure}
\includegraphics[width=0.25\textwidth]{PICTURES/pipe2.jpg}
\includegraphics[width=0.25\textwidth]{PICTURES/pipe1.jpg}
%\caption{Obstacle}
\end{figure}

Goals:
\begin{itemize}
\item Use of a RANS or LES model.
\item Use of a periodic box to initialize a fully developed turbulent flow.
\item Use of the TrioCFD parallel capabilities.
\end{itemize}

\end{block}
\end{frame}
%%%%%%%%%%%%%%%%%%%%%%%%%%%%%%%%%%%%%%%%%%%%%%%%%%%%%%%%%%%%%%%%%%%%%%%%
%%%%%%%%%%%%%%%%%%%%%%%%%%%%%%%%%%%%%%%%%%%%%%%%%%%%%%%%%%%%%%%%%%%%%%%%
\begin{frame}
\frametitle{Turbulent flow in a curved pipe (3D)}
\begin{block}{TrioCFD}

\begin{itemize}
\item First initialize TrioCFD environment:\\
{\small{
\begin{itemize}
\item On CEA Saclay computers:\\
\textbf{source /home/trust\_trio-public/\textcolor{red}{full}\_env\_\textcolor{blue}{TrioCFD}-1.9.5.sh }

\item On your own computer:\\
\textbf{source PathToTrioCFD/\textcolor{red}{full}\_env\_\textcolor{blue}{TrioCFD}.sh}

\end{itemize}
\textbf{echo \$exec} \\
\textbf{echo \$project\_directory} \\
}}

\item Create a new directory and copy some data:\\
{\small{
\textbf{mkdir -p Formation\_TRUST/yourname/PeriodicBox} \\
\textbf{cd Formation\_TRUST/yourname/PeriodicBox} \\
\textbf{cp \$project\_directory/share/Validation/Rapports\_automatiques/} \\
\hspace{4cm} \textbf{Turbulence/RANS/PeriodicBox/src/* \; .} \\
}}

This directory corresponds to an automated validation form. 
If you want to run it and generate the pdf report, see "Validation form" exercise on page \pageref{exo_validation}, but be aware that this case needs huge computational effort!
\end{itemize}

\end{block}
\end{frame}
%%%%%%%%%%%%%%%%%%%%%%%%%%%%%%%%%%%%%%%%%%%%%%%%%%%%%%%%%%%%%%%%%%%%%%%%
%%%%%%%%%%%%%%%%%%%%%%%%%%%%%%%%%%%%%%%%%%%%%%%%%%%%%%%%%%%%%%%%%%%%%%%%
\begin{frame}
\frametitle{Turbulent flow in a curved pipe (3D)}
\begin{block}{TrioCFD}

\begin{itemize}
\item There are several files:\\
    \begin{itemize}
    \item [$\circ$] \textit{BuildMeshes.data}: To build the meshes\\
    \item [$\circ$] \textit{PeriodicBoxRANS.data}: To run the flow in the box with RANS model\\
    \item [$\circ$] \textit{DomainFlowRANS.data}: To run the flow in the domain with inlet steady conditions from the box domain\\
    \item [$\circ$] \textit{PeriodicBoxLES.data}: To run the flow in the box with LES model\\
    \item [$\circ$] \textit{DomainFlowLES.data}: To run the flow in the domain with inlet unsteady conditions from the box domain
    \end{itemize}

\item First, edit and read the BuildMeshes.data file.

\item If you wish to run a RANS simulation, open the PeriodicBoxRANS.data and DomainFlowRANS.data files.

\item Or if you wish to run a LES simulation, open the PeriodicBoxLES.data and DomainFlowLES.data files.
\end{itemize}

\end{block}
\end{frame}
%%%%%%%%%%%%%%%%%%%%%%%%%%%%%%%%%%%%%%%%%%%%%%%%%%%%%%%%%%%%%%%%%%%%%%%%
%%%%%%%%%%%%%%%%%%%%%%%%%%%%%%%%%%%%%%%%%%%%%%%%%%%%%%%%%%%%%%%%%%%%%%%%
\begin{frame}
\frametitle{Turbulent flow in a curved pipe (3D)}
\begin{block}{TrioCFD}

\begin{itemize}
\item Then build the meshes:\\
\textbf{./prepare} \\
\textbf{triocfd BuildMeshes} \\
Notice that we use \textbf{"triocfd"} command lines, but it is also possible to use \textbf{"trust"} command because both scripts will use the variable \$exec which is the path to the TrioCFD executable.

%\item You can visualize the partitioned meshes with (MODEL=RANS or LES): \\
%\textbf{triocfd -mesh\index{triocfd -mesh} PeriodicBoxMODEL}

\item Then, set the max number of time steps in the time scheme using \textbf{nb\_pas\_dt\_max\index{nb\_pas\_dt\_max}} to 100 in the files PeriodicBoxRANS and PeriodicBoxLES and run a 2-cores parallel calculation, to initialize the turbulent flow in the box: \\
\textbf{triocfd PeriodicBoxRANS 2} \\
\textbf{triocfd PeriodicBoxLES 2} \\
(The full calculation takes approximately 1h in RANS and 10h in LES.)
\end{itemize}

\end{block}
\end{frame}
%%%%%%%%%%%%%%%%%%%%%%%%%%%%%%%%%%%%%%%%%%%%%%%%%%%%%%%%%%%%%%%%%%%%%%%%
%%%%%%%%%%%%%%%%%%%%%%%%%%%%%%%%%%%%%%%%%%%%%%%%%%%%%%%%%%%%%%%%%%%%%%%%
\begin{frame}
\frametitle{Turbulent flow in a curved pipe (3D)}
\begin{block}{TrioCFD}

\begin{itemize}
\item Open the file PeriodicBoxRANS.dt\_ev and PeriodicBoxLES.dt\_ev to read the last time of the two calculations after 100 time steps.

\item Once finished, open the file DomainFlowRANS.data and DomainFlowLES.data and change the maximal number of time steps to 10.

\item You can see that these data files are constituted by 2 problems, one for the box and one for the domain.
    \begin {itemize}
    \item We use the velocity and temperature fields of the last time step of the PeriodicBoxRANS (or LES) calculation as initial conditions for the pb\_box problem with the keyword "Champ\_fonc\_reprise\index{Champ\_fonc\_reprise}".
    \item In addition, the velocity and temperature fields of the pb\_box are used as boundary conditions for the pb\_dom through the keyword "champ\_front\_recyclage".
    \end{itemize}

\item Run a 6-cores parallel calculation over the domain (it will stop by default after 10 times steps):\\
\textbf{triocfd DomainFlowRANS 6}\\
\textbf{triocfd DomainFlowLES 6}

\end{itemize}

\end{block}
\end{frame}
%%%%%%%%%%%%%%%%%%%%%%%%%%%%%%%%%%%%%%%%%%%%%%%%%%%%%%%%%%%%%%%%%%%%%%%%


\section{{\bf{Turbulent flow over a backward-facing step (3D)}}}
%%%%%%%%%%%%%%%%%%%%%%%%%%%%%%%%%%%%%%%%%%%%%%%%%%%%%%%%%%%%%%%%%%%%%%%%
\begin{frame}
\begin{small}
\begin{columns}[c] 
\column{.45\textwidth}
\tableofcontents[sections={1-9},currentsection, currentsubsection]
\column{.5\textwidth} 
\tableofcontents[sections={10-16},currentsection, currentsubsection]
\end{columns}
\end{small}
\end{frame}
%%%%%%%%%%%%%%%%%%%%%%%%%%%%%%%%%%%%%%%%%%%%%%%%%%%%%%%%%%%%%%%%%%%%%%%%
%%%%%%%%%%%%%%%%%%%%%%%%%%%%%%%%%%%%%%%%%%%%%%%%%%%%%%%%%%%%%%%%%%%%%%%%
\begin{frame}
\frametitle{Turbulent flow over a backward-facing step}
\begin{block}{TrioCFD}

\begin{figure}
\includegraphics[width=0.75\textwidth]{PICTURES/marche3D.pdf}
%\caption{Obstacle}
\end{figure}

\textbf{Meshing}: 30x10x10 ($\Delta x=1m, \Delta y=0.2m, \Delta z=1m$)

\textbf{Fluid}: $\mu=5.10^{-5} kg.m^{-1}.s^{-1}, \rho=2 kg.m^{-3}$

\textbf{Boundary conditions}: with in entry $Re=\frac{U_0 H_{inlet} \rho}{\mu} = \frac{1 \times 1 \times 2}{5.10^{-5}} = 40000$\\
Inlet: $U_0=1 m.s^{-1}$\\
Outlet: $P_0=0$

\end{block}
\end{frame}
%%%%%%%%%%%%%%%%%%%%%%%%%%%%%%%%%%%%%%%%%%%%%%%%%%%%%%%%%%%%%%%%%%%%%%%%
%%%%%%%%%%%%%%%%%%%%%%%%%%%%%%%%%%%%%%%%%%%%%%%%%%%%%%%%%%%%%%%%%%%%%%%%
\begin{frame}
\frametitle{Turbulent flow over a backward-facing step}\label{init_triocfd}
\begin{block}{TrioCFD}

\begin{itemize}
\item First initialize TrioCFD environment:\\
{\small{
\begin{itemize}
\item On CEA Saclay computers:\\
\textbf{source /home/trust\_trio-public/env\_\textcolor{blue}{TrioCFD}-1.9.5.sh }

\item On your own computer:\\
\textbf{source PathToTrioCFD/env\_\textcolor{blue}{TrioCFD}.sh}

\end{itemize}
\textbf{echo \$exec} \\
\textbf{echo \$project\_directory} \\
}}

\item Copy the study named Marche3D: 
\textbf{triocfd -copy\index{triocfd -copy} Marche3D}

\item Edit the data file and: \\
    \begin{itemize}
    \item [$\circ$] Note that we use a "Pb\_Hydraulique\_Turbulent" problem with "Navier\_Stokes\_Turbulent" equations and a "modele\_turbulence\index{modele\_turbulence}" model.
    \item [$\circ$] Modify the fluid characteristics to perform a calculation at $Re=50000$. For example, impose $\rho = 1 kg.m^{-3}$ and $\mu=2.10^{-5} kg.m^{-1}.s^{-1}$.
%    \item [$\circ$] Define a turbulent hydraulic problem (see Reference Manual).
    \end{itemize}

\end{itemize}

\end{block}
\end{frame}
%%%%%%%%%%%%%%%%%%%%%%%%%%%%%%%%%%%%%%%%%%%%%%%%%%%%%%%%%%%%%%%%%%%%%%%%
%%%%%%%%%%%%%%%%%%%%%%%%%%%%%%%%%%%%%%%%%%%%%%%%%%%%%%%%%%%%%%%%%%%%%%%%
\begin{frame}
\frametitle{Turbulent flow over a backward-facing step}
\begin{block}{TrioCFD}

\begin{itemize}
\item Continue editing the data file
    \begin{itemize}
    \item [$\circ$] Select the sub-grid Smagorinsky turbulence model with standard wall law instead of the "sous\_maille" model (LES).
    \item [$\circ$] Select the Quick\index{quick} convection scheme.
    \item [$\circ$] Post-process of velocity, pressure, vorticity, turbulent viscosity at the nodes and elements.
    \end{itemize}
    
\item Run the calculation and post-process the main calculated fields.\\
\textbf{triocfd Marche3D} \\

\item Notice that we use \textbf{"triocfd"} command lines, but it is also possible to use \textbf{"trust"} script, since both scripts will use the variable \$exec which is the path to the TrioCFD executable.

\item Replace the sub-grid model by the standard k\_eps model (RANS).
\item Run the calculation and post-process of the velocity field to see the differences between the different turbulence models used.
\end{itemize}

\end{block}
\end{frame}
%%%%%%%%%%%%%%%%%%%%%%%%%%%%%%%%%%%%%%%%%%%%%%%%%%%%%%%%%%%%%%%%%%%%%%%%




\section{{\bf{Tank filling (2D, single-phase flow)}}}
%%%%%%%%%%%%%%%%%%%%%%%%%%%%%%%%%%%%%%%%%%%%%%%%%%%%%%%%%%%%%%%%%%%%%%%%
\begin{frame}
\begin{small}
\begin{columns}[c] 
\column{.45\textwidth}
\tableofcontents[sections={1-9},currentsection, currentsubsection]
\column{.5\textwidth} 
\tableofcontents[sections={10-16},currentsection, currentsubsection]
\end{columns}
\end{small}
\end{frame}
%%%%%%%%%%%%%%%%%%%%%%%%%%%%%%%%%%%%%%%%%%%%%%%%%%%%%%%%%%%%%%%%%%%%%%%%
%%%%%%%%%%%%%%%%%%%%%%%%%%%%%%%%%%%%%%%%%%%%%%%%%%%%%%%%%%%%%%%%%%%%%%%%
\begin{frame}
\frametitle{Tank filling (2D, single-phase flow)}
\begin{block}{We want to simulate the following flow:}

\begin{tabular}{cl}
\multirow{14}{*}{\includegraphics[width=0.3\textwidth]{PICTURES/tank2D.pdf}} & \textbf{Fluid}: Colored water \tabularnewline
 & diffusion $D=10^{-9}m^{2}.s^{-1}$, \tabularnewline
 & $\rho=1000kg.m^{-3}$, $\mu=10^{-3}kg.m^{-1}.s^{-1}$\tabularnewline

 & \tabularnewline

 & \textbf{Boundary conditions}:\tabularnewline

 & \textit{Inlet}: Velocity: $(V_{x},V_{y})=(V(t),0)$  \tabularnewline
 & with $V(t)=
    \begin{cases}
    1-(y-0.025/0.005)^{2} & ,\; t\leq0.5s\\
    0 & ,\; t>0.5s
    \end{cases}$ \tabularnewline

 & Concentration: $C=
    \begin{cases} 
    1 & ,\; t\leq0.5s\\
    0 & ,\; t>0.5s
    \end{cases}$ \tabularnewline
 & \textit{Outlet}: Pressure $P=0$\tabularnewline
 & \textit{Wall}: Velocity $V_{x}=0$, $V_{y}=0$ \tabularnewline

 & \tabularnewline

 & \textbf{Initial conditions}: Concentration $C_0=0$, \tabularnewline
 & Velocity $V=0$ \tabularnewline
\end{tabular}

\end{block}
\end{frame}
%%%%%%%%%%%%%%%%%%%%%%%%%%%%%%%%%%%%%%%%%%%%%%%%%%%%%%%%%%%%%%%%%%%%%%%%
%%%%%%%%%%%%%%%%%%%%%%%%%%%%%%%%%%%%%%%%%%%%%%%%%%%%%%%%%%%%%%%%%%%%%%%%
\begin{frame}
\frametitle{Tank filling (2D, single-phase flow)}
\begin{block}{}

\begin{itemize}
\item Load TRUST environment as described on page \pageref{init_trust}
\item Copy the study named \textbf{diagonale}. This test case deals with a 2D flow with Navier-Stokes and the equation for one constituent.
\item Edit the data file and modify the fluid characteristics to the previous ones ($\mu, \rho, D$).
\item We want to modify the geometry of this problem to the previous picture. So we want to create 3 blocks like:
\end{itemize}

\begin{figure}
\includegraphics[width=0.2\textwidth]{PICTURES/tank2D_2.pdf}
%\caption{Obstacle}
\end{figure}

\end{block}
\end{frame}
%%%%%%%%%%%%%%%%%%%%%%%%%%%%%%%%%%%%%%%%%%%%%%%%%%%%%%%%%%%%%%%%%%%%%%%%
%%%%%%%%%%%%%%%%%%%%%%%%%%%%%%%%%%%%%%%%%%%%%%%%%%%%%%%%%%%%%%%%%%%%%%%%
\begin{frame}
\frametitle{Tank filling (2D, single-phase flow)}
\begin{block}{}

\begin{itemize}
\item Create the corresponding mesh with 3 blocks (start with $dx=dy=0.2cm$ which gives a total nodes number $Nx=51$ and $Ny=121$). 
    \begin{itemize}
    \item [$\circ$] Create a first block "Block1" whose origin is (0, 0.03), $Nx=51$, $Ny=106$ (for $dx=dy=0.2cm$), $L=0.1 m$, $H=0.21 m$. Name the wall boundaries Left1, Outlet(=Top1) and Right1. (Don't forget the comma between blocks definitions.)
    \item [$\circ$] Create the second block "Block2" whose origin is (0, 0.02), $Nx=51$, $Ny=6$ (for $dx=dy=0.2cm$), $L=0.1 m$, $H=0.01 m$. Name the wall boundaries Inlet(=Left2) and Right2.
    \item [$\circ$] Create the third block "Block3" whose origin is (0, 0), $Nx=51$, $Ny=11$ (for $dx=dy=0.2cm$), $L=0.1 m$, $H=0.02 m$. Name the wall boundaries Left3, Bottom3 and Right3.
    \end{itemize}
\item Define the boundary wall, using the keyword \textbf{"RegroupeBord\index{RegroupeBord}"}.
\item You could also use facteurs and symx, symy keywords to define a refined mesh near the walls. 
%\item Check the mesh with: \textbf{trust -mesh\index{trust -mesh} diagonale} and correct the mesh errors if necessary.
\end{itemize}

\end{block}
\end{frame}
%%%%%%%%%%%%%%%%%%%%%%%%%%%%%%%%%%%%%%%%%%%%%%%%%%%%%%%%%%%%%%%%%%%%%%%%
%%%%%%%%%%%%%%%%%%%%%%%%%%%%%%%%%%%%%%%%%%%%%%%%%%%%%%%%%%%%%%%%%%%%%%%%
\begin{frame}
\frametitle{Tank filling (2D, single-phase flow)}
\begin{block}{}

\begin{itemize}
\item In the data file, change the values in the time scheme to stop the calculation at 1
second, and modify \textbf{dt\_min\index{dt\_min}} and \textbf{dt\_max\index{dt\_max}} values to let TRUST compute time step.

\item Change values for the gravity to $-9.81 m.s^{-2}$ following y-axis.

\item Note that the \textbf{beta\_co\index{beta\_co}} keyword may be useful in order to have a Boussinesq coupling between momentum and concentration equations ($\beta C_0 g(C-C_0$) source term added to the Navier-Stokes equations).

\item Change the initial and boundary conditions for Navier-Stokes equations:
    \begin{itemize}
    \item [$\circ$] for the Outlet boundary, you have to impose $P=0$,
    \item [$\circ$] for the Wall boundary, you have to impose $V_x=V_y=0$ with \textbf{"paroi\_fixe\index{paroi\_fixe}"} keyword.,
    \item [$\circ$] for the Inlet boundary, you have to impose $(V_{x},V_{y})=(V(t),0)$ with 
    $V(t)=  
    \begin{cases}  
    1-(y-0.025/0.005)^{2} & ,\; t\leq0.5s\\
    0 & ,\; t>0.5s
    \end{cases}
    $. You will use the \textbf{Champ\_Front\_Fonc\_txyz\index{Champ\_Front\_Fonc\_txyz}} keyword for the velocity, to write something like: \\
    \textbf{Champ\_Front\_Fonc\_txyz $2$ $(1-((y-0.025)/0.005)^2)*(t<0.5)$ $0.$}\\
    Note: Use ($t[0.5)$ syntax if you prefer ($t<=0.5$)
    \end{itemize}
\end{itemize}

\end{block}
\end{frame}
%%%%%%%%%%%%%%%%%%%%%%%%%%%%%%%%%%%%%%%%%%%%%%%%%%%%%%%%%%%%%%%%%%%%%%%%
%%%%%%%%%%%%%%%%%%%%%%%%%%%%%%%%%%%%%%%%%%%%%%%%%%%%%%%%%%%%%%%%%%%%%%%%
\begin{frame}
\frametitle{Tank filling (2D, single-phase flow)}
\begin{block}{}

\begin{itemize}
\item Change the initial and boundary conditions for the constituent equation. 
    \begin{itemize}
    \item [$\circ$] You will also use \textbf{Champ\_Front\_Fonc\_txyz\index{Champ\_Front\_Fonc\_txyz}} field for the Inlet boundary condition for concentration.
    \item [$\circ$] For the Outlet, use the following keywords to insure the external concentration is 0:
                    \textbf{Frontiere\_ouverte C\_ext Champ\_front\_uniforme 1 0.} \\
    \item [$\circ$] For the Wall, the keyword for impermeable boundary condition for concentration is \textbf{paroi\index{paroi}}.
    \end{itemize}
\item Check you have high-order schemes (i.e. \textbf{"Quick\index{quick}"} scheme) used in both equations to reduce numerical diffusion. 
\item Notice you could have suppressed diffusion term in concentration equation rather than using a small diffusion coefficient with:\\
\textbf{Diffusion \{ negligeable\index{negligeable} \}}
\item Add a concentration probe near the inlet (e.g.: at (0,0.025)).
\item Add a velocity segment probe (with 5 points between (0,0.021) and (0,0.029)) at the inlet boundary to see the time evolution of these two quantities (period 0.01s).
\end{itemize}

\end{block}
\end{frame}
%%%%%%%%%%%%%%%%%%%%%%%%%%%%%%%%%%%%%%%%%%%%%%%%%%%%%%%%%%%%%%%%%%%%%%%%
%%%%%%%%%%%%%%%%%%%%%%%%%%%%%%%%%%%%%%%%%%%%%%%%%%%%%%%%%%%%%%%%%%%%%%%%
\begin{frame}
\frametitle{Tank filling (2D, single-phase flow)}
\begin{block}{}

\begin{itemize}
%\item Change the period to 0.1s in the post-processing definition (keyword \textbf{dt\_post}).
\item Run the study and follow the time evolution with the probes:\\
\textbf{trust -evol\index{trust -evol} diagonale \&} \\
"Start computation!" button and "Plot" or "Plot on same" for probes.
\item Check the flow rate in inlet boundary in the diagonale\_pb\_Debit.out file (plotted on the right of the "evol" window). You should find a value near $6.8 \; 10^{-3} m^2.s^{-1}$.
\item Use VisIt to post-process the results at t=0.2, t=0.4s and t=0.7s. VisIt has some interesting feature for this study. It can give concentration histogram to check the numerical diffusion in the concentration equation: Add $\rightarrow$ Histogram $\rightarrow$ CONCENTRATION\_ELEM\_dom.
The volume of colored water (in $m^3$) is given by $Vol(t)= 6.66.10^{-3} t$ before $t=0.5s$ and $Vol(t)=3.33.10^{-3}$ after.
\end{itemize}

\end{block}
\end{frame}
%%%%%%%%%%%%%%%%%%%%%%%%%%%%%%%%%%%%%%%%%%%%%%%%%%%%%%%%%%%%%%%%%%%%%%%%
%%%%%%%%%%%%%%%%%%%%%%%%%%%%%%%%%%%%%%%%%%%%%%%%%%%%%%%%%%%%%%%%%%%%%%%%
\begin{frame}
\frametitle{Tank filling (2D, single-phase flow)}
\begin{block}{$\rightarrow$ VEF}

\begin{itemize}
\item Copy diagonale.data to diagonale\_VEF.data.
\item Triangulate your mesh (\textbf{trianguler\index{trianguler}} keyword).
\item In this new file, change the discretization (\textbf{VEFPreP1B\index{VEFPreP1B}} instead of \textbf{VDF\index{VDF}}).
\item Use \textbf{muscl\index{muscl}} instead of \textbf{quick\index{quick}} scheme.
\item And you can switch \textbf{GCP\index{GCP}} solver by \textbf{Cholesky\index{Cholesky}} solver of the Petsc library (direct method which may need large amount of RAM memory) to increase the speed resolution of the pressure linear system:\\
\textbf{GCP \{ precond ssor \{ omega 1.5 \} seuil 1.e-6 \} $\rightarrow$ Petsc Cholesky \{ \}}
\item Run the calculation. You must have an error, and TRUST stop the calculation. 
\end{itemize}

\end{block}
\end{frame}
%%%%%%%%%%%%%%%%%%%%%%%%%%%%%%%%%%%%%%%%%%%%%%%%%%%%%%%%%%%%%%%%%%%%%%%%
%%%%%%%%%%%%%%%%%%%%%%%%%%%%%%%%%%%%%%%%%%%%%%%%%%%%%%%%%%%%%%%%%%%%%%%%
\begin{frame}
\frametitle{Tank filling (2D, single-phase flow)}
\begin{block}{$\rightarrow$ VEF}

\begin{itemize}
\item As TRUST indicates, to avoid this problem, you can:
    \begin{itemize}
    \item [$\circ$] change the \textbf{trianguler\index{trianguler}} keyword to \textbf{trianguler\_h\index{Trianguler\_H}}, 
    \item [$\circ$] or use the \textbf{VerifierCoin\index{VerifierCoin}} keyword. For this, after this first error you must find a "diagonale\_VEF.decoupage\_som" file in your directory, so you can use it by adding:\\
    \textbf{VerifierCoin dom \{ read\_file diagonale\_VEF.decoupage\_som \} }\\
    just after "trianguler\index{trianguler} dom". This will subdivides inconsistent 2D/3D cells used with VEFPreP1B\index{VEFPreP1B} discretization (cf Reference Manual).
    \end{itemize}

\item Run the calculation and compare the results between \textbf{VDF/quick} and \textbf{VEFPreP1B/muscl} which must take much more time!
\end{itemize}

\end{block}
\end{frame}
%%%%%%%%%%%%%%%%%%%%%%%%%%%%%%%%%%%%%%%%%%%%%%%%%%%%%%%%%%%%%%%%%%%%%%%%



\section{{\bf{Tank filling (3D, two-phase flow)}}}
%%%%%%%%%%%%%%%%%%%%%%%%%%%%%%%%%%%%%%%%%%%%%%%%%%%%%%%%%%%%%%%%%%%%%%%%
\begin{frame}
\begin{small}
\begin{columns}[c] 
\column{.45\textwidth}
\tableofcontents[sections={1-9},currentsection, currentsubsection]
\column{.5\textwidth} 
\tableofcontents[sections={10-16},currentsection, currentsubsection]
\end{columns}
\end{small}
\end{frame}
%%%%%%%%%%%%%%%%%%%%%%%%%%%%%%%%%%%%%%%%%%%%%%%%%%%%%%%%%%%%%%%%%%%%%%%%
%%%%%%%%%%%%%%%%%%%%%%%%%%%%%%%%%%%%%%%%%%%%%%%%%%%%%%%%%%%%%%%%%%%%%%%%
\begin{frame}
\frametitle{Tank filling (3D, two-phase flow)}
\begin{block}{TrioCFD}

%\begin{figure}
%\includegraphics[width=0.4\textwidth]{PICTURES/tank3D.pdf}
%%\caption{Obstacle}
%\end{figure}

\begin{tabular}{cl}
\multirow{14}{*}{\includegraphics[width=0.35\textwidth]{PICTURES/tank3D}}
 & \textbf{Liquid}: $\rho=1000kg.m^{-3},$ \tabularnewline
 & $\mu=2,82.10^{-4}kg.m^{-1}.s^{-1},$ \tabularnewline
 & $\sigma=0.05N.m^{-1},\; D=10^{-6}m^{2}.s^{-1}$\tabularnewline
 & \tabularnewline
 & \textbf{Gas}: $\rho=100kg.m^{-3},$\tabularnewline
 & $\mu=2,82.10^{-4}kg.m^{-1}.s^{-1}$\tabularnewline
 & \tabularnewline
 & \textbf{Boundary conditions}:  \tabularnewline
 & Up: Free outlet, \hspace{0.5cm} Wall: $V=0$ \tabularnewline
 & Down: $V(x,y,z)=(0,0,10^{-3}m.s^{-1})$\tabularnewline
 & \tabularnewline
 & \textbf{Initial conditions}: $V=0$, \tabularnewline
 & $C=e^{(-((x-0.02)^{2}+(y-0.02)^{2}+(z-0.03)^{2})/0.03^{2})}$ \tabularnewline
 & \tabularnewline
 & N.B.: The interface between the air and the gas\tabularnewline
 & is a parabolic function.\tabularnewline
\end{tabular}

\end{block}
\end{frame}
%%%%%%%%%%%%%%%%%%%%%%%%%%%%%%%%%%%%%%%%%%%%%%%%%%%%%%%%%%%%%%%%%%%%%%%%
%%%%%%%%%%%%%%%%%%%%%%%%%%%%%%%%%%%%%%%%%%%%%%%%%%%%%%%%%%%%%%%%%%%%%%%%
\begin{frame}
\frametitle{Tank filling (3D, two-phase flow)}
\begin{block}{TrioCFD}

\begin{itemize}
\item First initialize TrioCFD environment:\\
{\small{
\begin{itemize}
\item On CEA Saclay computers:\\
\textbf{source /home/trust\_trio-public/env\_\textcolor{blue}{TrioCFD}-1.9.5.sh }

\item On your own computer:\\
\textbf{source PathToTrioCFD/env\_\textcolor{blue}{TrioCFD}.sh}

\end{itemize}
}}

\item Copy the study named \textbf{FTD\_all\_VDF}:\\
 \textbf{triocfd -copy\index{triocfd -copy} FTD\_all\_VDF}

\item This test case deals with a 3D two-phase flow in a tank with one initial interface between liquid and gas, a droplet, and a rotating solid in the liquid. The Discontinuous Front Tracking method is used with a 3D structured mesh.

\end{itemize}

\end{block}
\end{frame}
%%%%%%%%%%%%%%%%%%%%%%%%%%%%%%%%%%%%%%%%%%%%%%%%%%%%%%%%%%%%%%%%%%%%%%%%
%%%%%%%%%%%%%%%%%%%%%%%%%%%%%%%%%%%%%%%%%%%%%%%%%%%%%%%%%%%%%%%%%%%%%%%%
\begin{frame}
\frametitle{Tank filling (3D, two-phase flow)}
\begin{block}{TrioCFD}

\begin{itemize}


\item Notice that:
    \begin{itemize}
    \item [$\circ$] 2D Discontinuous Front Tracking method has not been intensively tested yet.
    \item [$\circ$] the type of the problem: \textbf{Probleme\_FT\_disc\_gen\index{Probleme\_FT\_disc\_gen}} in the data file. This refers to the Discontinuous Front Tracking method.
    \item [$\circ$] the keyword \textbf{modele\_turbulence\index{modele\_turbulence}}. Navier-Stokes equations of the Discontinuous Front Tracking problem needs the read of this keyword even if the flow is laminar. In this case, use the \textbf{nul\index{nul}} keyword just after \textbf{modele\_turbulence\index{modele\_turbulence}}. Else, specify the turbulence model to use.
    \end{itemize}
\end{itemize}

\end{block}
\end{frame}
%%%%%%%%%%%%%%%%%%%%%%%%%%%%%%%%%%%%%%%%%%%%%%%%%%%%%%%%%%%%%%%%%%%%%%%%
%%%%%%%%%%%%%%%%%%%%%%%%%%%%%%%%%%%%%%%%%%%%%%%%%%%%%%%%%%%%%%%%%%%%%%%%
\begin{frame}
\frametitle{Tank filling (3D, two-phase flow)}
\begin{block}{TrioCFD}

\begin{itemize}
\item Increase the height of the tank (from 0.06 to 0.12).

\item Add a second drop above the first one, at $z=0.08$ (keywords \textbf{ajout\_phase0\index{ajout\_phase0}} could be useful to add other interfaces, cf Reference Manual for \textbf{ajout\_phase0/ajout\_phase1\index{ajout\_phase1}} keywords). It is possible to access to the reference manual by typing \textbf{triocfd -index\index{triocfd -index}}. Don't forget the comma between the two definition of the drops.

\item Change the \textbf{dt\_post\index{dt\_post}} period of the 3 post-processing blocks (0.05 to 0.01). The first one (add \textbf{format lata\index{format lata}}) is the classical block for post-processing probes and fields. Here, we want to see the concentration field and the \textbf{"indicatrice\_interf"} field. Value of this field is 0 for liquid and 1 for gas, so the interface is located at "indicatrice" value 0.5.

\item Change the interpolation location of \textbf{indicatrice\_interf\index{indicatrice\_interf}} and the \textbf{concentration} fields in the first post-processing block, by adding the keyword \textbf{elem} just after the fields: the values in the post-processing tool will be plotted at the center of each element of the mesh.
\end{itemize}

\end{block}
\end{frame}
%%%%%%%%%%%%%%%%%%%%%%%%%%%%%%%%%%%%%%%%%%%%%%%%%%%%%%%%%%%%%%%%%%%%%%%%
%%%%%%%%%%%%%%%%%%%%%%%%%%%%%%%%%%%%%%%%%%%%%%%%%%%%%%%%%%%%%%%%%%%%%%%%
\begin{frame}
\frametitle{Tank filling (3D, two-phase flow)}
\begin{block}{TrioCFD}

\begin{itemize}
\item The second post-processing block is the new syntax to post-process interfaces moving meshes. You can visualize it with \textbf{VisIt}. 

\item On each interface, you can plot several fields i.e.: curvature with \textbf{courbure} keyword and velocity interface with \textbf{vitesse} keyword, \textbf{pe} field is for debugging purpose, it is useless here, you can suppress it) on several locations (on nodes with \textbf{sommets} keyword, on cells with \textbf{elements} keyword).

\item Run the calculation. Follow the time step evolution by having a look at the \textbf{dt\_ev} file. It contains on each line the physical time, the time step, security factor and residuals.

\item Post-process to visualize the interface and the concentration field.

\item You can increase the number of cells to have a finest simulation or also change to VEF discretization.

\end{itemize}

\end{block}
\end{frame}
%%%%%%%%%%%%%%%%%%%%%%%%%%%%%%%%%%%%%%%%%%%%%%%%%%%%%%%%%%%%%%%%%%%%%%%%



\section{{\bf{Salom\'e: 3D VEF mesh}}} \label{salome}
%\subsection{Salom\'e to create a 3D VEF mesh}
\subsection{Cylinder}
%\subsubsection{Cylinder}
%%%%%%%%%%%%%%%%%%%%%%%%%%%%%%%%%%%%%%%%%%%%%%%%%%%%%%%%%%%%%%%%%%%%%%%%
\begin{frame}
\begin{small}
\begin{columns}[c] 
\column{.45\textwidth}
\tableofcontents[sections={1-9},currentsection, currentsubsection]
\column{.5\textwidth} 
\tableofcontents[sections={10-16},currentsection, currentsubsection]
\end{columns}
\end{small}
\end{frame}
%%%%%%%%%%%%%%%%%%%%%%%%%%%%%%%%%%%%%%%%%%%%%%%%%%%%%%%%%%%%%%%%%%%%%%%%
%%%%%%%%%%%%%%%%%%%%%%%%%%%%%%%%%%%%%%%%%%%%%%%%%%%%%%%%%%%%%%%%%%%%%%%%
\begin{frame}
\frametitle{Salom\'e to create a 3D VEF mesh: Cylinder}
%\begin{block}{}

\begin{figure}
\includegraphics[width=0.7\textwidth]{PICTURES/salome1_2.jpg}
%\caption{Obstacle}
\end{figure}

%\end{block}
\end{frame}
%%%%%%%%%%%%%%%%%%%%%%%%%%%%%%%%%%%%%%%%%%%%%%%%%%%%%%%%%%%%%%%%%%%%%%%%
%%%%%%%%%%%%%%%%%%%%%%%%%%%%%%%%%%%%%%%%%%%%%%%%%%%%%%%%%%%%%%%%%%%%%%%%
\begin{frame}
\frametitle{Cylinder}
\begin{block}{Create a geometry}

\begin{itemize}
\item Create a new folder:\\
\texttt{\$ mkdir -p Formation\_TRUST/yourname/salome/exo1} \\
\texttt{\$ cd Formation\_TRUST/yourname/salome/exo1} \\

\item Launch Salom\'e (we suppose it is installed in \$PathToSalome):\\
\texttt{\$ \$PathToSalome/salome \&} 

\item Create a new study: File $\rightarrow$ New

\item Select the Geometry module into the SALOME drop-down menu (contains all the modules).

\item Save your study in hdf format (Salome format) frequently.

\item Create a first geometry with: New Entity $\rightarrow$ Primitives $\rightarrow$ Cylinder
\end{itemize}

\end{block}
\end{frame}
%%%%%%%%%%%%%%%%%%%%%%%%%%%%%%%%%%%%%%%%%%%%%%%%%%%%%%%%%%%%%%%%%%%%%%%%
%%%%%%%%%%%%%%%%%%%%%%%%%%%%%%%%%%%%%%%%%%%%%%%%%%%%%%%%%%%%%%%%%%%%%%%%
\begin{frame}
\frametitle{Cylinder}
\begin{block}{Create a geometry}

\begin{itemize}
\item Specify Radius $R=100$ and Height $H=300$ for the cylinder (the default values). Then Apply and Close.
\begin{figure}
\includegraphics[width=0.3\textwidth]{PICTURES/salome2.jpg}
%\caption{Obstacle}
\end{figure}

\item Rotate, zoom, move the geometry by switching to "Interaction style switch": Mouse icon.

\item Create groups for the geometry to define the top, the bottom and the lateral parts of the cylinder: New Entity $\rightarrow$ Group $\rightarrow$ Create Group

\item Select the good Shape Type ($\rightarrow \square$ surface).
\end{itemize}

\end{block}
\end{frame}
%%%%%%%%%%%%%%%%%%%%%%%%%%%%%%%%%%%%%%%%%%%%%%%%%%%%%%%%%%%%%%%%%%%%%%%%
%%%%%%%%%%%%%%%%%%%%%%%%%%%%%%%%%%%%%%%%%%%%%%%%%%%%%%%%%%%%%%%%%%%%%%%%
\begin{frame}
\frametitle{Cylinder}
\begin{block}{Create a geometry}

\begin{itemize}
\item Give a Group Name for the top: "Inlet".
\item Click on the arrow button of the Main Shape field and select the "Cylinder\_1" in the "Object browser" or in the visualization window.\\
\item Select the shape defining the top of the cylinder on the visualization window then Add $\rightarrow$ Apply.\\

\item Select the shape defining the part on the window then Add $\rightarrow$ Apply.\\
%\textbf{Warning}: Mouse icon should be unselected to select a shape.
\begin{figure}
\includegraphics[width=0.3\textwidth]{PICTURES/salome3.jpg}
%\caption{Obstacle}
\end{figure}
\end{itemize}

\end{block}
\end{frame}
%%%%%%%%%%%%%%%%%%%%%%%%%%%%%%%%%%%%%%%%%%%%%%%%%%%%%%%%%%%%%%%%%%%%%%%%
%%%%%%%%%%%%%%%%%%%%%%%%%%%%%%%%%%%%%%%%%%%%%%%%%%%%%%%%%%%%%%%%%%%%%%%%
\begin{frame}
\frametitle{Cylinder}

\begin{block}{Create a geometry}
\begin{itemize}
\item Do the same for the two other parts:
    \begin{itemize}
    \item [$\circ$] For the lateral: "Wall"
    \item [$\circ$] For the bottom: "Outlet" (you can rotate the cylinder to click on the bottom).
    \end{itemize}
\item Close the window once the 3 groups has been created. Check that they appear in the Object Browser (by clicking on the "$\triangleright$" in front of the "Cylinder\_1" object).
\end{itemize}
\end{block}

\begin{block}{Create a mesh}
\begin{itemize}
\item \label{salome_mesh} Now, switch to the Mesh module in the SALOME drop-down menu.
\item Select the Cylinder\_1 in the Object Browser and Right Click $\rightarrow$ 'Show' to visualize the geometry or click on the 'eye' next to the Cylindre\_1 object.
\item Create a mesh with: Mesh $\rightarrow$ Create Mesh
\item Select the Geometry used for the mesh if not selected by clicking on the Cylinder\_1 object in the Object Browser.
\end{itemize}
\end{block}

\end{frame}
%%%%%%%%%%%%%%%%%%%%%%%%%%%%%%%%%%%%%%%%%%%%%%%%%%%%%%%%%%%%%%%%%%%%%%%%
%%%%%%%%%%%%%%%%%%%%%%%%%%%%%%%%%%%%%%%%%%%%%%%%%%%%%%%%%%%%%%%%%%%%%%%%
\begin{frame}
\frametitle{Cylinder}
\begin{block}{Create a mesh}

\begin{itemize}
\item Choose Netgen 1D-2D-3D algorithm and click on "Apply and Close".
\item Select the object Mesh\_1 in the Object Browser and Right Click $\rightarrow$ Compute (or Mesh $\rightarrow$ Compute).
\item A tabular must appear with the number of triangles, quadrangles... Click on "Close".
\item Hide the geometry by selecting the Cylinder\_1 in the Object Browser and Right Click $\rightarrow$ Hide (or click on the eye).
\end{itemize}

\end{block}

\begin{center}
\includegraphics[width=0.39\textwidth]{PICTURES/salome4}
\end{center}

\end{frame}
%%%%%%%%%%%%%%%%%%%%%%%%%%%%%%%%%%%%%%%%%%%%%%%%%%%%%%%%%%%%%%%%%%%%%%%%
%%%%%%%%%%%%%%%%%%%%%%%%%%%%%%%%%%%%%%%%%%%%%%%%%%%%%%%%%%%%%%%%%%%%%%%%
\begin{frame}
\frametitle{Cylinder}
\begin{block}{Export your mesh in MED format}

\begin{itemize}
\item Check that the 3 boundaries have automatically been added in the "Group of Faces" of the \textbf{Mesh\_1} object in the Object Browser.
\item Export your mesh with the MED format:\\
Select the Mesh\_1 object then Right Click $\rightarrow$ Export $\rightarrow$ MED file (or File $\rightarrow$ Export $\rightarrow$ MED file).
\end{itemize}

\end{block}
\end{frame}
%%%%%%%%%%%%%%%%%%%%%%%%%%%%%%%%%%%%%%%%%%%%%%%%%%%%%%%%%%%%%%%%%%%%%%%%
%%%%%%%%%%%%%%%%%%%%%%%%%%%%%%%%%%%%%%%%%%%%%%%%%%%%%%%%%%%%%%%%%%%%%%%%
\begin{frame}
\frametitle{Cylinder}
\begin{block}{Read your mesh with TRUST}

\begin{itemize}
\item \label{read_mesh} Now build a data file named dom.data for TRUST:\\
\vspace{0.2cm}
\begin{tabular}{|l|}
\hline 
\textbf{dimension 3 }\tabularnewline
\textbf{domaine dom }\tabularnewline
\textbf{Read\_MED \{ domain dom file Mesh\_1.med \} }\tabularnewline
\textbf{Postraiter\_domaine\index{Postraiter\_domaine} \{ domaine dom fichier mesh format lata\index{format lata}
\}}\tabularnewline
\hline 
\end{tabular}

\item Open a new terminal then load TRUST environment as described on page \pageref{init_trust}
\item Run the data file and post-process the mesh with VisIt:\\
\textbf{trust dom} \\
\textbf{visit\index{visit} -o mesh.lata} \\
\vspace{0.2cm}
\textbf{Warning}: The more common error is to forget to define the boundaries with the groups for the mesh (and hence for Geometry). The error in TRUST is printed and detected during the discretization where all the faces of the mesh (in particular the boundary faces) are built.

\end{itemize}

\end{block}
\end{frame}
%%%%%%%%%%%%%%%%%%%%%%%%%%%%%%%%%%%%%%%%%%%%%%%%%%%%%%%%%%%%%%%%%%%%%%%%
%%%%%%%%%%%%%%%%%%%%%%%%%%%%%%%%%%%%%%%%%%%%%%%%%%%%%%%%%%%%%%%%%%%%%%%%
\begin{frame}
\frametitle{Cylinder}
\begin{block}{Refine your mesh and use viscous layers}

Goal: Improve the mesh for TRUST near the wall by using viscous layers.

\begin{itemize}
\item Create a new mesh named "Refined\_mesh" with: Mesh $\rightarrow$ Create Mesh

\item Select the Cylinder\_1 geometry in the Object Browser.

\item Select the "Netgen 3D" or "MG-Tetra" 3D algorithm.

\item Click on the wheel of "Add. Hypothesis" $\rightarrow$ "Viscous Layers" with:
    \begin{itemize}
    \item [$\circ$] Total thickness: 30
    \item [$\circ$] Number of layers: 3
    \item [$\circ$] Stretch factor: 1.1
    \item [$\circ$] Add to "Faces without layers" the 2 geometry groups "Inlet" and "Outlet" of Cylinder\_1 object (select or unselect the mouse icon).
    \item [$\circ$] Click OK
    \end{itemize}

\item Add a 2D algorithm: "Netgen 1D-2D" or "MG-CADSurf".
\end{itemize}

\end{block}
\end{frame}
%%%%%%%%%%%%%%%%%%%%%%%%%%%%%%%%%%%%%%%%%%%%%%%%%%%%%%%%%%%%%%%%%%%%%%%%
%%%%%%%%%%%%%%%%%%%%%%%%%%%%%%%%%%%%%%%%%%%%%%%%%%%%%%%%%%%%%%%%%%%%%%%%
\begin{frame}
\frametitle{Cylinder}
\begin{block}{Refine your mesh and use viscous layers}

\begin{itemize}
\item Click on the wheel of "Hypothesis" $\rightarrow$ "Netgen 2D parameters" or "MG-CADSurf parameters":
    \begin{itemize}
    \item For "Netgen 2D parameters":
        \begin{itemize}
        \item [$\circ$] Change "Fineness" from "Moderate" to "Very Fine".
        \item [$\circ$] Click OK.
        \end{itemize}
    \item For "MG-CADSurf parameters", change "User size" to 20. Click OK.
    \end{itemize}
\item "Apply and Close" the close mesh window.
\item Select the Refined\_Mesh object in the Object Browser and Right click $\rightarrow$ Compute
\end{itemize}

\end{block}
\end{frame}
%%%%%%%%%%%%%%%%%%%%%%%%%%%%%%%%%%%%%%%%%%%%%%%%%%%%%%%%%%%%%%%%%%%%%%%%
%%%%%%%%%%%%%%%%%%%%%%%%%%%%%%%%%%%%%%%%%%%%%%%%%%%%%%%%%%%%%%%%%%%%%%%%
\begin{frame}
\frametitle{Cylinder}
\begin{block}{Refine your mesh and use viscous layers}

\begin{itemize}
\item You should have a refined mesh with a mix of tetra, hexa, pyramid, prism elements for Netgen algorithms, and a mix of tetra and prisms for MG algorithms:

\begin{center}
\begin{tabular}{cc}
\includegraphics[width=0.28\textwidth]{PICTURES/salome6a.jpg}
 & 
\includegraphics[width=0.28\textwidth]{PICTURES/salome6b.jpg}
\tabularnewline
Netgen & MG
\end{tabular}
\end{center}

\item As TRUST accepted only tetras elements, you can quickly tetraedrize:
    \begin{itemize}
    \item [$\circ$] Select the Refined\_Mesh in the Object Browser.
    \item [$\circ$] "Modification" $\rightarrow$ "Split Volumes" and select "Tetrahedron".
    \item [$\circ$] Don't change the parameters, and click "Apply and Close".
    \end{itemize}
\end{itemize}

\end{block}
\end{frame}
%%%%%%%%%%%%%%%%%%%%%%%%%%%%%%%%%%%%%%%%%%%%%%%%%%%%%%%%%%%%%%%%%%%%%%%%
%%%%%%%%%%%%%%%%%%%%%%%%%%%%%%%%%%%%%%%%%%%%%%%%%%%%%%%%%%%%%%%%%%%%%%%%
\begin{frame}
\frametitle{Cylinder}

\begin{block}{Refine your mesh and use viscous layers}
\begin{itemize}
\item Check that the 3 boundaries have automatically been added in the "Group of Faces" of the \textbf{Refined\_Mesh} object in the Object Browser.
\item Export the mesh:
    \begin{itemize}
    \item [$\circ$] Select the Refined\_mesh, Right click $\rightarrow$ Export $\rightarrow$ MED file.
    \item [$\circ$] Save into a Refined\_Mesh.med file.
    \end{itemize}

\item Save your work in hdf format ("File" $\rightarrow$ "Save/Save As..."), and in python format with "File" $\rightarrow$ "Dump Study..."
\end{itemize}
\end{block}


\begin{block}{Run with TRUST}
\begin{itemize}
\item Edit your datafile or create a new one to read and visualize your refined mesh.

\item \textbf{N.B.}: The solutions of the exercise (mesh.py file for the first mesh and prism.py file for the second mesh) are located here: \$TRUST\_ROOT/doc/TRUST/exercices/salome.
\end{itemize}

\end{block}
\end{frame}
%%%%%%%%%%%%%%%%%%%%%%%%%%%%%%%%%%%%%%%%%%%%%%%%%%%%%%%%%%%%%%%%%%%%%%%%


\subsection{Revolution}
%\subsubsection{Revolution}
%%%%%%%%%%%%%%%%%%%%%%%%%%%%%%%%%%%%%%%%%%%%%%%%%%%%%%%%%%%%%%%%%%%%%%%%
\begin{frame}
\begin{small}
\begin{columns}[c] 
\column{.45\textwidth}
\tableofcontents[sections={1-9},currentsection, currentsubsection]
\column{.5\textwidth} 
\tableofcontents[sections={10-16},currentsection, currentsubsection]
\end{columns}
\end{small}
\end{frame}
%%%%%%%%%%%%%%%%%%%%%%%%%%%%%%%%%%%%%%%%%%%%%%%%%%%%%%%%%%%%%%%%%%%%%%%%
%%%%%%%%%%%%%%%%%%%%%%%%%%%%%%%%%%%%%%%%%%%%%%%%%%%%%%%%%%%%%%%%%%%%%%%%
\begin{frame}
\frametitle{Salom\'e to create a 3D VEF mesh: Revolution}
%\begin{block}{}

\begin{figure}
\includegraphics[width=0.7\textwidth]{PICTURES/salome9.jpg}
%\caption{Obstacle}
\end{figure}

%\end{block}
\end{frame}
%%%%%%%%%%%%%%%%%%%%%%%%%%%%%%%%%%%%%%%%%%%%%%%%%%%%%%%%%%%%%%%%%%%%%%%%
%%%%%%%%%%%%%%%%%%%%%%%%%%%%%%%%%%%%%%%%%%%%%%%%%%%%%%%%%%%%%%%%%%%%%%%%
\begin{frame}
\frametitle{Salom\'e to create a 3D VEF mesh: Revolution}
\begin{block}{Create a geometry}

\begin{itemize}
\item Create a directory and run Salom\'e (we suppose it is installed on \$PathToSalome):\\
\texttt{\$ mkdir -p Formation\_TRUST/yourname/salome/exo2} \\
\texttt{\$ cd Formation\_TRUST/yourname/salome/exo2} \\
\texttt{\$ \$PathToSalome/salome \&} 

\item Create a new study: File $\rightarrow$ New.

\item Select the Geometry module into the SALOME drop-down menu.

\item Create points: New Entity $\rightarrow$ Basic $\rightarrow$ Point\\
\begin{tabular}{lll}
Vertex\_1 (0,0,0)       &   Vertex\_2 (1,0,0)       &   Vertex\_3 (1,0,0.3) \tabularnewline
Vertex\_4 (0.75,0,0.3)  &   Vertex\_5 (0.375,0,1)   &   Vertex\_6 (0.75,0,1.6) \tabularnewline
Vertex\_7 (1,0,1.6)     &   Vertex\_8 (1,0,2)       &   Vertex\_9 (0,0,2) \tabularnewline
\end{tabular}

Then "Apply and Close"

\end{itemize}

\end{block}
\end{frame}
%%%%%%%%%%%%%%%%%%%%%%%%%%%%%%%%%%%%%%%%%%%%%%%%%%%%%%%%%%%%%%%%%%%%%%%%
%%%%%%%%%%%%%%%%%%%%%%%%%%%%%%%%%%%%%%%%%%%%%%%%%%%%%%%%%%%%%%%%%%%%%%%%
\begin{frame}
\frametitle{Revolution}
\begin{block}{Create a geometry}

\begin{figure}
\includegraphics[width=0.28\textwidth]{PICTURES/salome7.jpg}
%\caption{Obstacle}
\end{figure}

\begin{itemize}

\item Create edges:
    \begin{itemize}
    \item [$\circ$] New Entity $\rightarrow$ Basic $\rightarrow$ Line
        \begin{itemize}
        \item [$\diamond$] Line\_1 with Vertex\_1 and Vertex\_2
        \item [$\diamond$] Line\_2 with Vertex\_2 and Vertex\_3
        \item [$\diamond$] Line\_3 with Vertex\_3 and Vertex\_4
        \item [$\diamond$] Line\_4 with \textbf{Vertex\_6} and \textbf{Vertex\_7}
        \item [$\diamond$] Line\_5 with Vertex\_7 and Vertex\_8
        \item [$\diamond$] Line\_6 with Vertex\_8 and Vertex\_9
        \item [$\diamond$] Line\_7 with Vertex\_9 and Vertex\_1
        \item [$\diamond$] Then Apply and Close.
        \end{itemize}
    \end{itemize}
\end{itemize}

\end{block}
\end{frame}
%%%%%%%%%%%%%%%%%%%%%%%%%%%%%%%%%%%%%%%%%%%%%%%%%%%%%%%%%%%%%%%%%%%%%%%%
%%%%%%%%%%%%%%%%%%%%%%%%%%%%%%%%%%%%%%%%%%%%%%%%%%%%%%%%%%%%%%%%%%%%%%%%
\begin{frame}
\frametitle{Revolution}
\begin{block}{Create a geometry}

\begin{itemize}
\item Create edges:
    \begin{itemize}
    \item [$\circ$] New Entity $\rightarrow$ Basic $\rightarrow$ Arc
        \begin{itemize}
        \item [$\diamond$] Arc\_1 with Vertex\_4, Vertex\_5 and Vertex\_6.
        \item [$\diamond$] Then Apply and Close.
        \end{itemize}
    \end{itemize}

\item Create a wire: New Entity $\rightarrow$ Build $\rightarrow$ Wire
    \begin{itemize}
    \item [$\circ$] Wire\_1 on \underline{edges} with Line\_1,... , Line\_7 and Arc\_1 (with "Ctrl" button).
    \item [$\circ$] Then Apply and Close.
    \end{itemize}

\item Create a face: New Entity $\rightarrow$ Build $\rightarrow$ Face.
    \begin{itemize}
    \item [$\circ$] Face\_1 with Wire\_1 and "Apply and Close".
    \end{itemize}

\item Create a revolution cylinder: New Entity $\rightarrow$ Generation $\rightarrow$ Revolution.
    \begin{itemize}
    \item [$\circ$] named Cylinder\_1,
    \item [$\circ$] with Face\_1 in Objects,
    \item [$\circ$] click on the arrow button next "Axis" and select OZ in the Object Browser,
    \item [$\circ$] set the angle to $360^o$ and "Apply and Close".
    \end{itemize}
\end{itemize}

\end{block}
\end{frame}
%%%%%%%%%%%%%%%%%%%%%%%%%%%%%%%%%%%%%%%%%%%%%%%%%%%%%%%%%%%%%%%%%%%%%%%%
%%%%%%%%%%%%%%%%%%%%%%%%%%%%%%%%%%%%%%%%%%%%%%%%%%%%%%%%%%%%%%%%%%%%%%%%
\begin{frame}
\frametitle{Revolution}
\begin{block}{Create a geometry}

\begin{figure}
\includegraphics[width=0.28\textwidth]{PICTURES/salome8.jpg}
%\caption{Obstacle}
\end{figure}

\begin{itemize}
\item Create groups for the geometry to define the top, the bottom and the lateral parts of the cylinder: New Entity $\rightarrow$ Group $\rightarrow$ Create Group.
\item Save your study in hdf format ("File" $\rightarrow$ "Save/Save As..."), and in python format with "File" $\rightarrow$ "Dump Study..."
\item Now you can create the mesh in the same way than page \pageref{salome_mesh}.
\item \textbf{N.B.}: You can find the solutions of this exercise (revolution.py) in \$TRUST\_ROOT/doc/TRUST/exercices/salome.
\end{itemize}

\end{block}
\end{frame}
%%%%%%%%%%%%%%%%%%%%%%%%%%%%%%%%%%%%%%%%%%%%%%%%%%%%%%%%%%%%%%%%%%%%%%%%


\subsection{T-shape}
%\subsubsection{T-shape}
%%%%%%%%%%%%%%%%%%%%%%%%%%%%%%%%%%%%%%%%%%%%%%%%%%%%%%%%%%%%%%%%%%%%%%%%
\begin{frame}
\begin{small}
\begin{columns}[c] 
\column{.45\textwidth}
\tableofcontents[sections={1-9},currentsection, currentsubsection]
\column{.5\textwidth} 
\tableofcontents[sections={10-16},currentsection, currentsubsection]
\end{columns}
\end{small}
\end{frame}
%%%%%%%%%%%%%%%%%%%%%%%%%%%%%%%%%%%%%%%%%%%%%%%%%%%%%%%%%%%%%%%%%%%%%%%%
%%%%%%%%%%%%%%%%%%%%%%%%%%%%%%%%%%%%%%%%%%%%%%%%%%%%%%%%%%%%%%%%%%%%%%%%
\begin{frame}
\frametitle{Salom\'e to create a 3D VEF mesh: T-shape}
%\begin{block}{}

\begin{figure}
\includegraphics[width=0.7\textwidth]{PICTURES/salome17.jpg}
%\caption{Obstacle}
\end{figure}

%\end{block}
\end{frame}
%%%%%%%%%%%%%%%%%%%%%%%%%%%%%%%%%%%%%%%%%%%%%%%%%%%%%%%%%%%%%%%%%%%%%%%%
%%%%%%%%%%%%%%%%%%%%%%%%%%%%%%%%%%%%%%%%%%%%%%%%%%%%%%%%%%%%%%%%%%%%%%%%
\begin{frame}
\frametitle{Salom\'e to create a 3D VEF mesh: T-shape}
\begin{block}{Create a geometry}

\begin{itemize}
\item Create a directory and run Salom\'e (we suppose it is installed on \$PathToSalome):\\
\texttt{\$ mkdir -p Formation\_TRUST/yourname/salome/exo3} \\
\texttt{\$ cd Formation\_TRUST/yourname/salome/exo3} \\
\texttt{\$ \$PathToSalome/salome \&} 

\item Create a new study: File $\rightarrow$ New.

\item Select the Geometry module into the SALOME drop-down menu.

\item Create two cylinders: New Entity $\rightarrow$ Primitives $\rightarrow$ Cylinders:\\
Cylinder\_1: radius 0.5, height 5. Then "Apply".\\
Cylinder\_2: radius 0.3, height 3. Then "Apply and Close".
\item Save your study in hdf format (Salome format) frequently.
\end{itemize}

\end{block}
\end{frame}
%%%%%%%%%%%%%%%%%%%%%%%%%%%%%%%%%%%%%%%%%%%%%%%%%%%%%%%%%%%%%%%%%%%%%%%%
%%%%%%%%%%%%%%%%%%%%%%%%%%%%%%%%%%%%%%%%%%%%%%%%%%%%%%%%%%%%%%%%%%%%%%%%
\begin{frame}
\frametitle{T-shape}
\begin{block}{Create a geometry}

\begin{itemize}
\item Rotate Cylinder\_2: Operations $\rightarrow$ Transformation $\rightarrow$ Rotation\\
Name:Rotation\_1, Object: Cylinder\_2, Axis: 'OY', Angle: $90^o$ \\
Then "Apply and Close".

\item Translate Rotation\_1: Operations $\rightarrow$ Transformation $\rightarrow$ Translation\\
Name: Translation\_1, Object: Rotation\_1, Dx=Dy=0, Dz=1.5 \\
Then "Apply and Close".
\end{itemize}

\begin{center}
\begin{tabular}{ccc}
\includegraphics[width=0.25\textwidth]{PICTURES/salome11} & \includegraphics[width=0.25\textwidth]{PICTURES/salome12} & \includegraphics[width=0.28\textwidth]{PICTURES/salome13} \tabularnewline
Rotation & Translation & Fuse \tabularnewline
\end{tabular}
\end{center}

\end{block}
\end{frame}
%%%%%%%%%%%%%%%%%%%%%%%%%%%%%%%%%%%%%%%%%%%%%%%%%%%%%%%%%%%%%%%%%%%%%%%%
%%%%%%%%%%%%%%%%%%%%%%%%%%%%%%%%%%%%%%%%%%%%%%%%%%%%%%%%%%%%%%%%%%%%%%%%
\begin{frame}
\frametitle{T-shape}
\begin{block}{Create a geometry}

\begin{itemize}
\item Fuse Cylinder\_1 and Translation\_1: Operations $\rightarrow$ Boolean $\rightarrow$ Fuse\\
Name: Fuse\_1, Selected Objects: 2\_Objects (use "Ctrl" button to select Cylinder\_1 and Translation\_1 in the Object Browser). \\
Then "Apply and Close".

\begin{figure}
\includegraphics[width=0.5\textwidth]{PICTURES/salome14.jpg}
%\caption{Obstacle}
\end{figure}

\end{itemize}

\end{block}
\end{frame}
%%%%%%%%%%%%%%%%%%%%%%%%%%%%%%%%%%%%%%%%%%%%%%%%%%%%%%%%%%%%%%%%%%%%%%%%
%%%%%%%%%%%%%%%%%%%%%%%%%%%%%%%%%%%%%%%%%%%%%%%%%%%%%%%%%%%%%%%%%%%%%%%%
\begin{frame}
\frametitle{T-shape}
\begin{block}{Create a geometry}

\begin{itemize}
\item We are now going to create the boundaries: New Entity $\rightarrow$ Explode \\
    \begin{itemize}
    \item [$\circ$] Main Object: Fuse\_1, Sub-shape type: Face, select "Select sub-shape" and click on the surface Outlet and "Apply".
    \item [$\circ$] This will create a face named "Face\_1" in the Fuse\_1 object (click on the "$\blacktriangleright$"), rename it "Outlet" (by right-clicking and "rename").
    \end{itemize}

\begin{figure}
\includegraphics[width=0.48\textwidth]{PICTURES/salome15.jpg}
%\caption{Obstacle}
\end{figure}
\end{itemize}

\end{block}
\end{frame}
%%%%%%%%%%%%%%%%%%%%%%%%%%%%%%%%%%%%%%%%%%%%%%%%%%%%%%%%%%%%%%%%%%%%%%%%
%%%%%%%%%%%%%%%%%%%%%%%%%%%%%%%%%%%%%%%%%%%%%%%%%%%%%%%%%%%%%%%%%%%%%%%%
\begin{frame}
\frametitle{T-shape}
\begin{block}{Create a geometry}

\begin{itemize}
\item Do the same for "Inlet\_x" and "Inlet\_z".
\item We are now going to create the boundary Wall: New Entity $\rightarrow$ Group $\rightarrow$ Create group:\\
Shape Type: surface, Name: Wall, Main Shape: Fuse\_1.\\
Click on the surface of the Cylinder\_1 then "Add", click on the surface of Translation\_1 then "Add" and "Apply and Close".
\begin{figure}
\includegraphics[width=0.25\textwidth]{PICTURES/salome16.jpg}
%\caption{Obstacle}
\end{figure}
\end{itemize}

\end{block}
\end{frame}
%%%%%%%%%%%%%%%%%%%%%%%%%%%%%%%%%%%%%%%%%%%%%%%%%%%%%%%%%%%%%%%%%%%%%%%%
%%%%%%%%%%%%%%%%%%%%%%%%%%%%%%%%%%%%%%%%%%%%%%%%%%%%%%%%%%%%%%%%%%%%%%%%
\begin{frame}
\frametitle{T-shape}

\begin{block}{Create a geometry}
\begin{itemize}
\item We are now going to create the point "Corner": New Entity $\rightarrow$ Explode \\
    \begin{itemize}
    \item [$\circ$] Main Object: Fuse\_1, Sub-shape type: Vertex, select "Select sub-shape" and click on the chosen point and "Apply and Close".
    \item [$\circ$] This will create a vertex name "Vertex\_1", rename it "Corner" (by right-clicking and "rename").
    \end{itemize}
\end{itemize}
\end{block}

\begin{block}{Create a mesh}
\begin{itemize}
\item \label{salome_mesh} Now, switch to the Mesh module in the SALOME drop-down menu.
\item Select the Fuse\_1 in the Object Browser and Right Click $\rightarrow$ 'Show' to visualize the geometry or click on the 'eye' next to the Fuse\_1 object.
\item Create a mesh with: Mesh $\rightarrow$ Create Mesh.
\item Select the Geometry used for the mesh if not selected by clicking on the Fuse\_1 object in the Object Browser.
\end{itemize}
\end{block}

\end{frame}
%%%%%%%%%%%%%%%%%%%%%%%%%%%%%%%%%%%%%%%%%%%%%%%%%%%%%%%%%%%%%%%%%%%%%%%%
%%%%%%%%%%%%%%%%%%%%%%%%%%%%%%%%%%%%%%%%%%%%%%%%%%%%%%%%%%%%%%%%%%%%%%%%
\begin{frame}
\frametitle{T-shape}
\begin{block}{Create a mesh}

\begin{itemize}
\item Choose "Netgen 3D" for 3D algorithm.
\item Click on the wheel of "Add. Hypothesis" $\rightarrow$ "Viscous Layers" and set:
    \begin{itemize}
    \item [$\circ$] Total thickness: 0.05
    \item [$\circ$] Number of layers: 3
    \item [$\circ$] Stretch factor: 1.1
    \item [$\circ$] Extrusion method: Node Offset
    \item [$\circ$] Add to "Faces with layers (Wall)" the geometry group "Wall" of Fuse\_1 object in the Object Browser (select or unselect the mouse icon). Click on "Add".
    \item [$\circ$] Click "OK".
    \end{itemize}
\item Choose "Netgen 1D-2D" for 2D algorithm.
\item Click on the wheel of "Hypothesis" $\rightarrow$ "Netgen 2D parameters" and set for "Arguments" menu:
    \begin{itemize}
    \item [$\circ$] Max. Size: 0.6
    \item [$\circ$] Min. Size: 0
    \item [$\circ$] Finess: Custom
    \item [$\circ$] Growth rate: 0.1
    \end{itemize}

\end{itemize}

\end{block}
\end{frame}
%%%%%%%%%%%%%%%%%%%%%%%%%%%%%%%%%%%%%%%%%%%%%%%%%%%%%%%%%%%%%%%%%%%%%%%%
%%%%%%%%%%%%%%%%%%%%%%%%%%%%%%%%%%%%%%%%%%%%%%%%%%%%%%%%%%%%%%%%%%%%%%%%
\begin{frame}
\frametitle{T-shape}
\begin{block}{Create a mesh}

\hspace{1cm} $\color{darkblue}\circ$ {\small{Nb. segs per Edge: 2}}\\
\hspace{1cm} $\color{darkblue}\circ$ {\small{Nb. segs per Radius: 4}}\\
\hspace{1cm} $\color{darkblue}\circ$ {\small{Select "Limit size by Surface Curvature", "Optimize".}}\\
\hspace{1cm} $\color{darkblue}\circ$ {\small{Unselect "Allow Quadrangles".}}\\
\hspace{1cm} $\color{darkblue}\circ$ {\small{Unselect "Second Order".}}\\

\begin{itemize}
\item For "Local Size" menu:
    \begin{itemize}
    \item [$\circ$] Select "Corner" object in the Object Browser and click on "On Vertex" in the "Hypothesis Construction" window.
    \item [$\circ$] Double-click on the value in the table and set it to "0.01".
    \item [$\circ$] Click "OK".
    \end{itemize}
\item For "Advanced" menu:
    \begin{itemize}
    \item [$\circ$] Select "Fuse Coincident Nodes on Edges and Vertices".
    \end{itemize}
\item Click on "Apply and Close".
\item Select the Mesh\_1 object in the Object Browser and Right click $\rightarrow$ Compute.
\item You should have a mesh with a mix of tetra and prism elements.

\end{itemize}

\end{block}
\end{frame}
%%%%%%%%%%%%%%%%%%%%%%%%%%%%%%%%%%%%%%%%%%%%%%%%%%%%%%%%%%%%%%%%%%%%%%%%
%%%%%%%%%%%%%%%%%%%%%%%%%%%%%%%%%%%%%%%%%%%%%%%%%%%%%%%%%%%%%%%%%%%%%%%%
\begin{frame}
\frametitle{T-shape}
\begin{block}{Create a mesh}

\begin{itemize}
\item As TRUST accepted only tetras elements, you can quickly tetraedrize:
    \begin{itemize}
    \item [$\circ$] Select Mesh\_1 in the Object Browser.
    \item [$\circ$] "Modification" $\rightarrow$ "Split Volumes" and select "Tetrahedron".
    \item [$\circ$] Don't change the parameters, and click "Apply and Close".
    \end{itemize}

\item Check that the 4 boundaries have automatically been added in the "Group of Faces" of the \textbf{Mesh\_1} object in the Object Browser.

\item Export the mesh:
    \begin{itemize}
    \item [$\circ$] Select the Mesh\_1, Right click $\rightarrow$ Export $\rightarrow$ MED file.
    \item [$\circ$] Save into a Mesh\_1.med file.
    \end{itemize}

\item Save your study in hdf format ("File" $\rightarrow$ "Save/Save As..."), and in python format with "File" $\rightarrow$ "Dump Study..."

\item \textbf{N.B.}: You can find the solutions of this exercise (T\_shape.py) in \$TRUST\_ROOT/doc/TRUST/exercices/salome.
\end{itemize}

\end{block}
\end{frame}
%%%%%%%%%%%%%%%%%%%%%%%%%%%%%%%%%%%%%%%%%%%%%%%%%%%%%%%%%%%%%%%%%%%%%%%%
%%%%%%%%%%%%%%%%%%%%%%%%%%%%%%%%%%%%%%%%%%%%%%%%%%%%%%%%%%%%%%%%%%%%%%%%
\begin{frame}
\frametitle{T-shape}
\begin{block}{Run with TRUST}

\begin{itemize}
\item Copy the T\_shape.data file in your directory:\\
\textbf{cp \$TRUST\_ROOT/doc/TRUST/exercices/salome/T\_shape.data  .}

\item Run it with TRUST:\\
\textbf{trust T\_shape}\\
or in parallel with:\\
\textbf{trust -partition T\_shape}\\
\textbf{trust PAR\_T\_shape 4}\\

\item You can visualize the results with Visit or Salom\'e by opening the T\_shape\_0000.med file for sequential calculation or PAR\_T\_shape\_0000.med for parallel calculation.
\end{itemize}

\end{block}
\end{frame}
%%%%%%%%%%%%%%%%%%%%%%%%%%%%%%%%%%%%%%%%%%%%%%%%%%%%%%%%%%%%%%%%%%%%%%%%
%%%%%%%%%%%%%%%%%%%%%%%%%%%%%%%%%%%%%%%%%%%%%%%%%%%%%%%%%%%%%%%%%%%%%%%%
\begin{frame}
\frametitle{T-shape}
\begin{block}{Visu with Visit}

\begin{figure}
\includegraphics[width=0.5\textwidth]{PICTURES/salome18.jpg}
%\caption{Obstacle}
\end{figure}

\end{block}
\end{frame}
%%%%%%%%%%%%%%%%%%%%%%%%%%%%%%%%%%%%%%%%%%%%%%%%%%%%%%%%%%%%%%%%%%%%%%%%
\subsection{Mesh for coupled problem}
%%%%%%%%%%%%%%%%%%%%%%%%%%%%%%%%%%%%%%%%%%%%%%%%%%%%%%%%%%%%%%%%%%%%%%%%
\begin{frame}
\begin{small}
\begin{columns}[c] 
\column{.45\textwidth}
\tableofcontents[sections={1-9},currentsection, currentsubsection]
\column{.5\textwidth} 
\tableofcontents[sections={10-16},currentsection, currentsubsection]
\end{columns}
\end{small}
\end{frame}
%%%%%%%%%%%%%%%%%%%%%%%%%%%%%%%%%%%%%%%%%%%%%%%%%%%%%%%%%%%%%%%%%%%%%%%%
%%%%%%%%%%%%%%%%%%%%%%%%%%%%%%%%%%%%%%%%%%%%%%%%%%%%%%%%%%%%%%%%%%%%%%%%
\begin{frame}
\frametitle{Salom\'e: Create domains for a TRUST coupled problem\label{salomecoupledpb}}
\begin{block}{}
Consider that we want to simulate a coupled problem with TRUST on a complex geometry. Suppose that this latter is drawn by means of Salom\'e.

The main difficulty araises from the fact that the mesh elements should be connected on the interface between the two domains in order to be correctly read by TRUST. \\
%
In this exercise, you will learn how to:
\begin{itemize}
\item  Create two domains (domain 1 and domain 2) using Salom\'e
\item  Get a coherent mesh on the interface between the two domains. We recall that TRUST is able to treat only meshes with connected elements on the interface.
\item  Mesh both domains and export it into a single MED file.
\item  Read the MED file from TRUST datafile and simulate a coupled problem.
\end{itemize}
%
Note: for simple geometries, the internal TRUST mesher "Mailler" will be largely sufficient (see the exercise \ref{docond_pb_couple} for example).

\end{block}
\end{frame}
%%%%%%%%%%%%%%%%%%%%%%%%%%%%%%%%%%%%%%%%%%%%%%%%%%%%%%%%%%%%%%%%%%%%%%%%
%%%%%%%%%%%%%%%%%%%%%%%%%%%%%%%%%%%%%%%%%%%%%%%%%%%%%%%%%%%%%%%%%%%%%%%%
\begin{frame}
\frametitle{Mesh for coupled problem}
\begin{block}{Description of the problem}

Let us consider the cooling of a solid block by means of a fluid flowing inside circular cross-section channels. The channel is centered in the block of a square cross-section. The outer boundaries of the solid are adiabatic. Below is given a schematic description of the problem.
%
\begin{figure}
\includegraphics[width=0.55\textwidth]{PICTURES/Pb_couple.pdf}
\end{figure}
\vspace{-0.5cm}
In order to build meshes using Salom\'e for such a simulation, we should create two domains: the first domain will represent the solid block and the second domain the fluid.

\end{block}

\end{frame}
%%%%%%%%%%%%%%%%%%%%%%%%%%%%%%%%%%%%%%%%%%%%%%%%%%%%%%%%%%%%%%%%%%%%%%%%
%%%%%%%%%%%%%%%%%%%%%%%%%%%%%%%%%%%%%%%%%%%%%%%%%%%%%%%%%%%%%%%%%%%%%%%%
\begin{frame}
\frametitle{Mesh for coupled problem}

\begin{block}{In the Geometry module:}
\begin{itemize}

\item Create a new folder for this exercise and launch Salom\'e:\\
\texttt{\$ mkdir -p Formation\_TRUST/yourname/salome/exo4} \\
\texttt{\$ cd Formation\_TRUST/yourname/salome/exo4} \\
\texttt{\$ \$PathToSalome/salome \&} 

\item Create a new study: File $\rightarrow$ New

\item Select the Geometry module from drop-down menu of Salom\'e.

\item Save your study in hdf format (Salom\'e format) frequently.

\item Create the first geometry with: New Entity $\rightarrow$ Primitives $\rightarrow$ Box. Then, specify dimensions $Dx=200$,  $Dy=200$ and $Dz=400$. After that, Apply and Close.

\item Create a vertex with: New Entity $\rightarrow$ Basic $\rightarrow$ Point. Specify the vertex coordinates $X=100$, $Y=100$ and $Z=0$, then Apply and Close

\item Create the second geometry with: New Entity $\rightarrow$ Primitives $\rightarrow$ Cylinder. Then, specify the Base Point: Vertex\_1 and Vector OZ, Radius $R=40$ and Height $H=400$ for the cylinder. After that, Apply and Close.

\end{itemize}
\end{block}

\end{frame}
%%%%%%%%%%%%%%%%%%%%%%%%%%%%%%%%%%%%%%%%%%%%%%%%%%%%%%%%%%%%%%%%%%%%%%%%
%%%%%%%%%%%%%%%%%%%%%%%%%%%%%%%%%%%%%%%%%%%%%%%%%%%%%%%%%%%%%%%%%%%%%%%%
\begin{frame}
\frametitle{Mesh for coupled problem}

\begin{block}{In the Geometry module:}
\begin{itemize}

 \item Perform a cut with: Operations $\rightarrow$ Boolean $\rightarrow$ Cut. In the Main Object select: Box\_1 and in the Tool Objects select: Cylinder\_1 $\rightarrow$ Apply and Close.
 
 \item Create a partition with: Operations $\rightarrow$ Partition. In Objects select: Cylinder\_1 and Cut\_1. Then Apply and Close.

 \item Define 2 groups of volumes, one for each domain with: New entity $\rightarrow$ Group $\rightarrow$ Create Group.
 \begin{enumerate}
 \item  Shape Type: Volume. Name: Solid. Main Shape: Partition\_1. Select the hollow box then click on Add, after that on Apply.
 \item  Shape Type: Volume. Name: Fluid. Main Shape: Partition\_1. Select the cylindrical channel then click on Add, after that on Apply and Close.
 \end{enumerate}
 
\end{itemize}
 \end{block}

\end{frame}
%%%%%%%%%%%%%%%%%%%%%%%%%%%%%%%%%%%%%%%%%%%%%%%%%%%%%%%%%%%%%%%%%%%%%%%%
%%%%%%%%%%%%%%%%%%%%%%%%%%%%%%%%%%%%%%%%%%%%%%%%%%%%%%%%%%%%%%%%%%%%%%%%
\begin{frame}
\frametitle{Mesh for coupled problem}

\begin{block}{In the Geometry module:}
\begin{itemize}

 \item Define the groups of faces for external boundaries and the interface with: New entity $\rightarrow$ Group $\rightarrow$ Create Group.
 \begin{enumerate}
 \item Shape Type: Surface. Name: Fluid\_inlet. Main Shape: Partition\_1. Then select the bottom of the cylinder and Add. Click on Apply.
 \item Shape Type: Surface. Name: Fluid\_outlet. Main Shape: Partition\_1. Then select the top circular boundary of the cylinder then click on Add, then Apply.
 \item Shape Type: Surface. Name: Solid\_top. Main Shape: Partition\_1. Then select the top of the box then click on Add, then Apply.
 \item Shape Type: Surface. Name: Solid\_bottom. Main Shape: Partition\_1. Then select the bottom of the box then click on Add, then Apply.
 \item Shape Type: Surface. Name: Solid\_lateral\_walls. Main Shape: Partition\_1. Then select the remaining 4 lateral boundaries of the box then click on Add, then Apply.
 \item Shape Type: Surface. Name: Solid\_Fluid\_Interface. Main Shape: Partition\_1. Then select the top boundary of the box and click on Hide selected, then Click on a lateral boundary and click on Hide selected, then the lateral boundary of the cylinder will be visible. Select it and click on Add then Apply.
 \end{enumerate}
 
\end{itemize}
\end{block}

\end{frame}
%%%%%%%%%%%%%%%%%%%%%%%%%%%%%%%%%%%%%%%%%%%%%%%%%%%%%%%%%%%%%%%%%%%%%%%%
%%%%%%%%%%%%%%%%%%%%%%%%%%%%%%%%%%%%%%%%%%%%%%%%%%%%%%%%%%%%%%%%%%%%%%%%
\begin{frame}
\frametitle{Mesh for coupled problem}

\begin{block}{In the Mesh module:}
\begin{itemize} 

 \item Create a mesh based on the Partition\_1 with: Mesh $\rightarrow$ Create Mesh. Let the name be Mesh\_1 and in Geometry select Partition\_1. In the 3D algorithm, select NETGEN 1D-2D-3D. Click on the wheel of "Hypothesis" then on "NETGEN 3D Parameters". In Arguments, select the fineness "Fine" instead of "Moderate" then click on OK then Apply and Close.
 
 \item Right click on Mesh\_1, then Compute.
 
 \item Check that the 6 boundaries have automatically been added in the "Group of Faces" of the \textbf{Mesh\_1} object in the Object Browser.
 
 \item Check that the 2 volume groups have automatically been added in the "Group of Volumes" of the \textbf{Mesh\_1} object in the Object Browser.
 
 \item Export the mesh in med format (if possible, choose MED 3.2).

 \item Dump the study and the mesh in a python script with: File $\rightarrow$ Dump Study. \\
 We will need it on the next exercice. 
 
\end{itemize}
\end{block}

\end{frame}
%%%%%%%%%%%%%%%%%%%%%%%%%%%%%%%%%%%%%%%%%%%%%%%%%%%%%%%%%%%%%%%%%%%%%%%%
%%%%%%%%%%%%%%%%%%%%%%%%%%%%%%%%%%%%%%%%%%%%%%%%%%%%%%%%%%%%%%%%%%%%%%%%
\begin{frame}
\frametitle{Mesh for coupled problem}

\begin{block}{Launch the coupled problem datafile:}
\begin{itemize} 

 \item Load TRUST environment as described on page \pageref{init_trust}

 \item Copy the datafile: \\
 \texttt{\$ cp \$TRUST\_ROOT/doc/TRUST/exercices/salome/Coupled\_pb.data .}

 \item Run the test case using TRUST: \\
 \texttt{\$ trust Coupled\_pb.data}

 \item When the computation finishes, visualize results using VisIt:\\
 \texttt{\$ visit -o Coupled\_pb.lata}

 \item Draw the temperature profile on both domains and set the min and max on color bar to 300 and 400 respectively. When you visualize time evolution of temperature, you see that the solid is cooled and its temperature decreases. If we increase the time of the simulation, the temperature of the solid will be equal to that of the fluid at the steady state.
 
\end{itemize}
\end{block}

\end{frame}
%%%%%%%%%%%%%%%%%%%%%%%%%%%%%%%%%%%%%%%%%%%%%%%%%%%%%%%%%%%%%%%%%%%%%%%%
\subsection{Edit and build meshes with python script}
%%%%%%%%%%%%%%%%%%%%%%%%%%%%%%%%%%%%%%%%%%%%%%%%%%%%%%%%%%%%%%%%%%%%%%%%
\begin{frame}
\begin{small}
\begin{columns}[c] 
\column{.45\textwidth}
\tableofcontents[sections={1-9},currentsection, currentsubsection]
\column{.5\textwidth} 
\tableofcontents[sections={10-16},currentsection, currentsubsection]
\end{columns}
\end{small}
\end{frame}
%%%%%%%%%%%%%%%%%%%%%%%%%%%%%%%%%%%%%%%%%%%%%%%%%%%%%%%%%%%%%%%%%%%%%%%%
%%%%%%%%%%%%%%%%%%%%%%%%%%%%%%%%%%%%%%%%%%%%%%%%%%%%%%%%%%%%%%%%%%%%%%%%
\begin{frame}
\frametitle{Aim of this exercise}

\begin{block}{}
Consider you already created a mesh using Salome. You will be able to change mesh and geometry parameters without starting it from scratch if you saved your study in python script.
\newline

Salome offers the possibility to save all commands launched from the Graphical User Interface, either in HDF5 format, or to save the study as a python script. 
If you dump your study in a python script, you can later modify some parameters and run it to build the new mesh, without having to rebuild the geometry nor the mesh in Salome.
\newline

In this exercice, you will learn how to do so.

\end{block}

\end{frame}
%%%%%%%%%%%%%%%%%%%%%%%%%%%%%%%%%%%%%%%%%%%%%%%%%%%%%%%%%%%%%%%%%%%%%%%%
%%%%%%%%%%%%%%%%%%%%%%%%%%%%%%%%%%%%%%%%%%%%%%%%%%%%%%%%%%%%%%%%%%%%%%%%
\begin{frame}
\frametitle{Copy the python script}

\begin{block}{In your terminal}
\begin{itemize}
 \item Create a new directory\\
 \texttt{\$ mkdir -p Formation\_TRUST/yourname/salome/exo5}\\
 \texttt{\$ cd Formation\_TRUST/yourname/salome/exo5}

 \item Copy the python script you generated in the previous exercise (see page \pageref{salomecoupledpb}) and the data file\\
 \texttt{\$ cp ../exo4/Mesh\_1.py .}\\
 \texttt{\$ cp \$TRUST\_ROOT/doc/TRUST/exercices/salome/Coupled\_pb.data .}\\
N.B.: If you have not permformed the previous exercise, you can copy the python script as follows:\\
 \texttt{\$ path=\$TRUST\_ROOT/doc/TRUST/exercices/salome}\\
 \texttt{\$ cp \$path/Coupled\_pb.py Mesh\_1.py}\\
\end{itemize}
\end{block}

\end{frame}
%%%%%%%%%%%%%%%%%%%%%%%%%%%%%%%%%%%%%%%%%%%%%%%%%%%%%%%%%%%%%%%%%%%%%%%%
%%%%%%%%%%%%%%%%%%%%%%%%%%%%%%%%%%%%%%%%%%%%%%%%%%%%%%%%%%%%%%%%%%%%%%%%
\begin{frame}
\frametitle{Edit geometry and meshing parameters}

\begin{block}{Edit the Mesh\_1.py script in a text editor}
\begin{itemize}
 \item At the end of the \texttt{Mesh\_1.py} script, add the line:\\
 \texttt{   Mesh\_1.ExportMED("Mesh\_1.med",0)}\\
which allows to export the generated mesh on MED format.
 \item Change some parameters:
 \begin{enumerate}
  \item Change the height of the box and the cylinder: \texttt{400 $\rightarrow$ 300}

  \item Change the radius of the cylinder: \texttt{40 $\rightarrow$ 70}

  \item Change the cell's MaxSize in NETGEN\_3D\_Parameters\_1: \texttt{48.9898 $\rightarrow$ 9.}

  \item Change the cell's MinSize in NETGEN\_3D\_Parameters\_1: \texttt{6.97246 $\rightarrow$ 2.}

 \end{enumerate}

 \item Save and close\\
\end{itemize}
\end{block}

\end{frame}
%%%%%%%%%%%%%%%%%%%%%%%%%%%%%%%%%%%%%%%%%%%%%%%%%%%%%%%%%%%%%%%%%%%%%%%%
%%%%%%%%%%%%%%%%%%%%%%%%%%%%%%%%%%%%%%%%%%%%%%%%%%%%%%%%%%%%%%%%%%%%%%%%
\begin{frame}
\frametitle{Generate the mesh and vizualize it}

\begin{block}{In your terminal}
\begin{itemize}
 \item Run your python script using the \texttt{-t} option of Salome:\\
 \texttt{\$ \$PathToSalome/salome -t Mesh\_1.py}

 \item You should have now \texttt{Mesh\_1.med}  generated in your folder

 %\item Visualize the generated mesh to check that changes made:\\
 %\texttt{\$ trust -mesh Coupled\_pb}\\
 You should see that the box is smaller in the \texttt{z} direction, the cylinder is thicker, and the mesh is finer.

 \item You can now run the calculation on the new mesh:\\
 \texttt{\$ trust Coupled\_pb}
\end{itemize}
\end{block}

\end{frame}
%%%%%%%%%%%%%%%%%%%%%%%%%%%%%%%%%%%%%%%%%%%%%%%%%%%%%%%%%%%%%%%%%%%%%%%%
%%%%%%%%%%%%%%%%%%%%%%%%%%%%%%%%%%%%%%%%%%%%%%%%%%%%%%%%%%%%%%%%%%%%%%%%
\section{{\bf{Gmsh meshing tool}}} \label{gmsh}
\subsection{2D VEF mesh}
%\subsection{Gmsh}
%\subsubsection{2D VEF mesh}
%%%%%%%%%%%%%%%%%%%%%%%%%%%%%%%%%%%%%%%%%%%%%%%%%%%%%%%%%%%%%%%%%%%%%%%%
\begin{frame}
\begin{small}
\begin{columns}[c] 
\column{.45\textwidth}
\tableofcontents[sections={1-9},currentsection, currentsubsection]
\column{.5\textwidth} 
\tableofcontents[sections={10-16},currentsection, currentsubsection]
\end{columns}
\end{small}
\end{frame}
%%%%%%%%%%%%%%%%%%%%%%%%%%%%%%%%%%%%%%%%%%%%%%%%%%%%%%%%%%%%%%%%%%%%%%%%
%%%%%%%%%%%%%%%%%%%%%%%%%%%%%%%%%%%%%%%%%%%%%%%%%%%%%%%%%%%%%%%%%%%%%%%%
\begin{frame}
\frametitle{Gmsh to create a 2D VEF mesh}
\begin{block}{Geometry which will be created, based on a TrioCFD validation test case geometry}

\begin{figure}
\includegraphics[width=0.7\textwidth]{PICTURES/gmsh.pdf}
%\caption{Obstacle}
\end{figure}

\begin{itemize}
\item Create a directory and copy an example:\\
{\footnotesize{
\textbf{mkdir \, -p \, Formation\_TRUST/yourname/gmsh} \\
\textbf{cd \, Formation\_TRUST/yourname/gmsh}
\item Load TrioCFD environment as described on page \pageref{init_triocfd}.
\item Then copy file.geo as follows:
\textbf{dir=\$project\_directory/share/Validation/Rapports\_automatiques/Turbulence} \\
\textbf{cp \, \$dir/LES/Drag/src/shape.geo \, file.geo} \\
\textbf{gedit file.geo \& } \\
\textbf{gmsh file.geo \&} }}\\
\end{itemize}

\end{block}
\end{frame}
%%%%%%%%%%%%%%%%%%%%%%%%%%%%%%%%%%%%%%%%%%%%%%%%%%%%%%%%%%%%%%%%%%%%%%%%
%%%%%%%%%%%%%%%%%%%%%%%%%%%%%%%%%%%%%%%%%%%%%%%%%%%%%%%%%%%%%%%%%%%%%%%%
\begin{frame}
\frametitle{Gmsh to create a 2D VEF mesh}
\begin{block}{}

\begin{itemize}
\item First configure gmsh to show points, lines, and surface numbers of the geometry. In menu Tools $\rightarrow$ Options $\rightarrow$ Geometry $\rightarrow$ Visibility, select Lines, Surfaces, Point labels, Line labels, Surface labels, Volume labels close this window. 
\item Save definitively your choices with File $\rightarrow$ Save Options As default.

\item Now, look at the file.geo file, you can see the definition of parameters, points, lines... You can see the position of the points and lines with theirs numbers in gmsh.

\item Modify the file to suppress the obstacle. We want to keep only 4 points and 4 lines, so you have to suppress the points 5,6,7,8 and the lines 1,3,4,5. For this, you have to:
    \begin{itemize}
    \item [$\circ$] comment (with "//") the definition of the points 5,6,7 and 8.
    \item [$\circ$] comment the definition of the lines 1,3,4,5. Note that the "Circle" is a line so "Circle(1)=Line(1)".
    \item [$\circ$] modify the line(2), it will now links points 1 and 2.
    \item [$\circ$] comment the "Physical Line" named "Shape" which use the line(1) (= Circle(1)) and the line(3).
    \end{itemize}
\end{itemize}

\end{block}
\end{frame}
%%%%%%%%%%%%%%%%%%%%%%%%%%%%%%%%%%%%%%%%%%%%%%%%%%%%%%%%%%%%%%%%%%%%%%%%
%%%%%%%%%%%%%%%%%%%%%%%%%%%%%%%%%%%%%%%%%%%%%%%%%%%%%%%%%%%%%%%%%%%%%%%%
\begin{frame}
\frametitle{Gmsh to create a 2D VEF mesh}
\begin{block}{}

\hspace{1cm} $\color{darkblue}\circ$ {\small{ suppress the numbers 4 and 5 in the "Physical Line" "Axis". It refers to the \\
\hspace{1.4cm} lines 4 and 5 which does not exist anymore.}}\\
\hspace{1cm} $\color{darkblue}\circ$ {\small{ suppress the numbers 1,3,4 and 5 in the "Line Loop(1)".}}\\
\hspace{1cm} $\color{darkblue}\circ$ {\small{ set H to 2 and L to 10. (You can comment D, E, param and X definitions.)}}

\begin{itemize}
\item Press "Reload" in the gmsh GUI $\rightarrow$ Geometry to update the geometry visualization.
\item Now we will add the circle. Note that you can only create circle arcs with angle strictly smaller than $\pi$!
    \begin{itemize}
    \item [$\circ$] create the middle of the circle like "Point(10)=\{1,1,0,lc\};" where the triplet "1,1,0" are the coordinates of the point and "lc" the thickness of the cells next this point.
    \item [$\circ$] create 4 points around this point, which will correspond to the 4 arc of the circle: \\
Point(11)=\{1.25,1,0,lc2\};\\
Point(12)=\{1,1.25,0,lc2\};\\
Point(13)=\{0.75,1,0,lc2\};\\
Point(14)=\{1,0.75,0,lc2\};
    \end{itemize}

\end{itemize}

\end{block}
\end{frame}
%%%%%%%%%%%%%%%%%%%%%%%%%%%%%%%%%%%%%%%%%%%%%%%%%%%%%%%%%%%%%%%%%%%%%%%%
%%%%%%%%%%%%%%%%%%%%%%%%%%%%%%%%%%%%%%%%%%%%%%%%%%%%%%%%%%%%%%%%%%%%%%%%
\begin{frame}
\frametitle{Gmsh to create a 2D VEF mesh}
\begin{block}{}

\hspace{1cm} $\color{darkblue}\circ$ {\small{ define the 4 arcs of the circle with this points: \\
\hspace{1.4cm} Circle(10)=\{11,10,12\};\\
\hspace{1.4cm} Circle(11)=\{12,10,13\};\\
\hspace{1.4cm} Circle(12)=\{13,10,14\};\\
\hspace{1.4cm} Circle(13)=\{14,10,11\}; "}}\\
\hspace{1cm} $\color{darkblue}\circ$ {\small{ create a "Physical Line" for the circle: \\
\hspace{1.4cm} "Physical Line("Circle") = \{10,11,12,13\};"\\
\hspace{1.4cm} This name will be used in your TRUST data file as the name of your \\
\hspace{1.4cm} boundaries.}}\\
\hspace{1cm} $\color{darkblue}\circ$ {\small{create a line loop for the circle just after the first line loop:\\
\hspace{1.4cm} Line Loop(2) = \{10,11,12,13\}; \\}}
\hspace{1cm} $\color{darkblue}\circ$ {\small{Add the number of this line loop in the "Plane Surface(1)":\\
\hspace{1.4cm} Plane Surface(1) = \{1,2\}; }}


\begin{itemize}
\item Suppress the Physical lines "Axis" and "Top" and create a physical line named "Wall" which regroups the top and the bottom of this geometry (lines 2 and 7).
\end{itemize}

\end{block}
\end{frame}
%%%%%%%%%%%%%%%%%%%%%%%%%%%%%%%%%%%%%%%%%%%%%%%%%%%%%%%%%%%%%%%%%%%%%%%%


\subsection{3D VEF mesh}
%\subsubsection{3D VEF mesh}
%%%%%%%%%%%%%%%%%%%%%%%%%%%%%%%%%%%%%%%%%%%%%%%%%%%%%%%%%%%%%%%%%%%%%%%%
\begin{frame}
\begin{small}
\begin{columns}[c] 
\column{.45\textwidth}
\tableofcontents[sections={1-9},currentsection, currentsubsection]
\column{.5\textwidth} 
\tableofcontents[sections={10-16},currentsection, currentsubsection]
\end{columns}
\end{small}
\end{frame}
%%%%%%%%%%%%%%%%%%%%%%%%%%%%%%%%%%%%%%%%%%%%%%%%%%%%%%%%%%%%%%%%%%%%%%%%
%%%%%%%%%%%%%%%%%%%%%%%%%%%%%%%%%%%%%%%%%%%%%%%%%%%%%%%%%%%%%%%%%%%%%%%%
\begin{frame}
\frametitle{Gmsh to create a 3D VEF mesh}
\begin{block}{}

\begin{itemize}
\item Select "Mesh" in the drop-down menu of gmsh and mesh in 2D.
\item Export it to a MED file: "File" $\rightarrow$ "Save As..." and name the file file.med. (Keep the default options.) You can verify your mesh by opening it with gmsh: \textbf{gmsh file.med \&}.
\item Build a TRUST data file with the \textbf{Postraiter\_domaine\index{Postraiter\_domaine}} keyword, to read the mesh (like in the Salome exercise page \pageref{read_mesh}). Visualize the mesh with VisIt.
\item Then we will try to create a 3D mesh, by using the Extrusions feature of Gmsh. See more about extrusions in \textcolor{blue}{\underline{http://geuz.org/gmsh/doc/texinfo/gmsh.html}}.
\item Save your initial file in a new one named file3D.geo.
\item Comment your "Physical lines", they will not be used here.
\item Add the line "Extrude \{0,0,1\} \{ Surface\{1\} ; \}" just before the definition of the physical surface.
\end{itemize}

\end{block}
\end{frame}
%%%%%%%%%%%%%%%%%%%%%%%%%%%%%%%%%%%%%%%%%%%%%%%%%%%%%%%%%%%%%%%%%%%%%%%%
%%%%%%%%%%%%%%%%%%%%%%%%%%%%%%%%%%%%%%%%%%%%%%%%%%%%%%%%%%%%%%%%%%%%%%%%
\begin{frame}
\frametitle{Gmsh to create a 3D VEF mesh}
\begin{block}{}

\begin{itemize}
\item Comment the line "Physical Surface("domain") = \{1\};" in 3D we will have a "Physical Volume" which will be define at the end of the .geo file.
\item Define the "Physical Surface" which will be the boundaries of your geometry with the number of the surfaces which can be read on the geometry plotted by gmsh:\\
Physical Surface("Inlet") = \{38\};\\
Physical Surface("Outlet") = \{30\};\\
Physical Surface("Wall") = \{1,26,34,55\};\\
Physical Surface("Obstacle") = \{42,46,50,54\};\\
\item Define you physical volume, you can see its number in yellow in the window:\\
Physical Volume("dom") = \{1\};

%\item In 3D, you will use "Plane Surface" to define the boundaries of your domain and "Physical Surface" (instead of Physical Line in 2D) to name these boundaries: \\
%\begin{tabular}{ll}
%{\small{Line Loop(1) = \{15,6,1,14\};}}         & {\small{// Line loop of several lines}}   \tabularnewline
%{\small{Line Loop(2) = \{15,16,17,18\};}}       & {\small{// Line loop of several lines}}   \tabularnewline
%{\small{Plane Surface(1) = \{ 1,2 \} ;}}        & {\small{// Surface/boundary defined}}     \tabularnewline
%{\small{Physical Surface("Inlet") = \{1\};}}    & {\small{// Naming boundary is MANDATORY}} \tabularnewline
%\end{tabular}
%\item You will need to define a volume and a physical volume at the end of .geo file:
%\begin{tabular}{ll}
%{\small{Surface Loop(1) = { 1,2,8,12,16,20 } ;}}    & {\small{ // Surface Loop of Plane Surfaces }}     \tabularnewline
%{\small{Volume(1) = { 1 } ;}}                       & {\small{ // Volume defined }}                     \tabularnewline
%{\small{Physical Volume("domain") = 1;}}            & {\small{ // Naming domain is MANDATORY }}         \tabularnewline
%\end{tabular}

\item Select the "Mesh" tool in the drop-down menu pf gmsh and mesh in 3D your geometry. It takes a few minutes, to reduce this time, increase the size of your cells by changing the values of lc.
\end{itemize}

\end{block}
\end{frame}
%%%%%%%%%%%%%%%%%%%%%%%%%%%%%%%%%%%%%%%%%%%%%%%%%%%%%%%%%%%%%%%%%%%%%%%%
%%%%%%%%%%%%%%%%%%%%%%%%%%%%%%%%%%%%%%%%%%%%%%%%%%%%%%%%%%%%%%%%%%%%%%%%
\begin{frame}
\frametitle{Gmsh to create a 3D VEF mesh}
\begin{block}{}

\begin{itemize}
\item You can use the "Optimize 3D" algorithm to optimize your mesh.

\item Export your mesh to a MED file. Run gmsh again on this exported MED file to check everything is defined:\\
\textbf{gmsh file.med \&}

\item Now, use your mesh in a TRUST calculation, for example:
        \begin{itemize}
        \item [$\circ$] copy the data file of the first exercise into a Obstacle\_VEF.data file, \\
        \item [$\circ$] read the MED file:\\
                        \textbf{Read\_MED \{ domain dom file file.med \} } \\
                        Notice that by default with Gmsh, the mesh name is the name of the file!
        \item [$\circ$] change the discretization type,
        \item [$\circ$] be careful to the choice of the convection scheme for your VEF calculation,
        \item [$\circ$] and run the simulation on the unstructured mesh.
        \end{itemize}
\end{itemize}

\end{block}
\end{frame}
%%%%%%%%%%%%%%%%%%%%%%%%%%%%%%%%%%%%%%%%%%%%%%%%%%%%%%%%%%%%%%%%%%%%%%%%

\section{{\bf{Validation form}}}\label{exo_validation}
%%%%%%%%%%%%%%%%%%%%%%%%%%%%%%%%%%%%%%%%%%%%%%%%%%%%%%%%%%%%%%%%%%%%%%%%
\begin{frame}
\begin{small}
\begin{columns}[c] 
\column{.45\textwidth}
\tableofcontents[sections={1-9},currentsection, currentsubsection]
\column{.5\textwidth} 
\tableofcontents[sections={10-16},currentsection, currentsubsection]
\end{columns}
\end{small}
\end{frame}
%%%%%%%%%%%%%%%%%%%%%%%%%%%%%%%%%%%%%%%%%%%%%%%%%%%%%%%%%%%%%%%%%%%%%%%%
%%%%%%%%%%%%%%%%%%%%%%%%%%%%%%%%%%%%%%%%%%%%%%%%%%%%%%%%%%%%%%%%%%%%%%%%
\begin{frame}
\frametitle{Validation form}
\begin{block}{}

The prefered route to build a validation form in TRUST and its baltiks is to build a Jupyter notebook. \\
\vspace{0.5cm}

See an example of notebook in \\
\small{\$TRUST\_ROOT/Validation/Rapports\_automatiques/Verification/SampleFormJupyter/} \\
or run on terminal:\\
Run\_fiche -doc
\end{block}
\end{frame}
%%%%%%%%%%%%%%%%%%%%%%%%%%%%%%%%%%%%%%%%%%%%%%%%%%%%%%%%%%%%%%%%%%%%%%%%
%%%%%%%%%%%%%%%%%%%%%%%%%%%%%%%%%%%%%%%%%%%%%%%%%%%%%%%%%%%%%%%%%%%%%%%%
\begin{frame}
\frametitle{Validation form}
\begin{block}{}

\begin{itemize}
\item First copy the validation form named Source\_canal\_perio:\\
{\small{
\textbf{mkdir -p Formation\_TRUST/yourname/validation} \\
\textbf{cd  Formation\_TRUST/yourname/validation} \\
\textbf{VERIF=\$TRUST\_ROOT/Validation/Rapports\_automatiques/Verification} \\
\textbf{cp -r \$VERIF/Verification\_codage/Source\_canal\_perio \, .} \\
\textbf{cd Source\_canal\_perio}
 }}\\

\item Display Run\_fiche script options:\\
\textbf{Run\_fiche -help\index{Run\_fiche -help}}

\item Build the PDF report with:\\
\textbf{Run\_fiche -export\_pdf}\\
then, open the report with:\\
\textbf{evince build/rapport.pdf}
\end{itemize}

\end{block}
\end{frame}
%%%%%%%%%%%%%%%%%%%%%%%%%%%%%%%%%%%%%%%%%%%%%%%%%%%%%%%%%%%%%%%%%%%%%%%%%
%%%%%%%%%%%%%%%%%%%%%%%%%%%%%%%%%%%%%%%%%%%%%%%%%%%%%%%%%%%%%%%%%%%%%%%%%
\begin{frame}
\frametitle{Validation form}
\begin{block}{}

\begin{itemize}
\item Now, we are going to change the validation form (Examples are given in SampleFormJupyter Validation form):\\
\textbf{Run\_fiche \&}
    \begin{itemize}
    \item [$\circ$] Add the mesh plot in the report. For this, at end of the notebook, add a new Markdown cell for a title:\\
    \textbf{\#\#Additional information\\
    \#\#\#Mesh visualization } \\
    \item Add a new code cell to plot mesh \\
    \small{fig=visit.Show("./std.lata", "Mesh", "dom", plotmesh=True,title="Mesh")}\\
    \small{fig.plot()}
    \end{itemize}
\end{itemize}

\end{block}
\end{frame}
%%%%%%%%%%%%%%%%%%%%%%%%%%%%%%%%%%%%%%%%%%%%%%%%%%%%%%%%%%%%%%%%%%%%%%%%%
%%%%%%%%%%%%%%%%%%%%%%%%%%%%%%%%%%%%%%%%%%%%%%%%%%%%%%%%%%%%%%%%%%%%%%%%%
\begin{frame}
\frametitle{Validation form}
\begin{block}{}

\begin{itemize}
\item Add the evolution of residuals in the report in log scale (see .dt\_ev file). For this, introduce a new Jupyter text cell and write:\\
\textbf{\#\# Residual plot}\\
then, on another cell plot residual using:\\
\textbf{Graph = plot.Graph("Residual plot")} \\
\textbf{Graph.addResidu("std.dt\_ev",label="Residu")} \\
\textbf{Graph.scale(yscale='log')}
\end{itemize}

\end{block}
\end{frame}
%%%%%%%%%%%%%%%%%%%%%%%%%%%%%%%%%%%%%%%%%%%%%%%%%%%%%%%%%%%%%%%%%%%%%%%%%
%%%%%%%%%%%%%%%%%%%%%%%%%%%%%%%%%%%%%%%%%%%%%%%%%%%%%%%%%%%%%%%%%%%%%%%%%
\begin{frame}
\frametitle{Validation form}
\begin{block}{}

\hspace{1cm} $\color{darkblue}\circ$ {Visualize the pressure field at the last time: complete the section "Additional information" with a new cell "Visualizing fields" }

\hspace{1cm} $\color{darkblue}\circ$ Then try to find how to display that field on \textbf{SampleFormJupyter} available with \textbf{Run\_fiche -doc}

\end{block}
\end{frame}
%%%%%%%%%%%%%%%%%%%%%%%%%%%%%%%%%%%%%%%%%%%%%%%%%%%%%%%%%%%%%%%%%%%%%%%%%
%%%%%%%%%%%%%%%%%%%%%%%%%%%%%%%%%%%%%%%%%%%%%%%%%%%%%%%%%%%%%%%%%%%%%%%%%
\begin{frame}
\frametitle{Validation form}
\begin{block}{}

\begin{itemize}
\item Now, we are going to extract the number of cells and the simulation final time from three .err files and write it in .dat files. Extraction script (extraction.sh) is already available in the src directory of this validation form (you can have a look at it). 

    \begin{itemize}
    \item [$\circ$] from your validation form, run this script using "executeScript" from run module of trustutils (see SampleFormJupyter for help). 

    \item [$\circ$] Add a table to display the results of .dat files: complete the chapter "Additional information" by introducing new cells.\\
    Tip: look on the next slide for an example of table plot.
    \end{itemize}

\end{itemize}

\end{block}
\end{frame}
%%%%%%%%%%%%%%%%%%%%%%%%%%%%%%%%%%%%%%%%%%%%%%%%%%%%%%%%%%%%%%%%%%%%%%%%%
%%%%%%%%%%%%%%%%%%%%%%%%%%%%%%%%%%%%%%%%%%%%%%%%%%%%%%%%%%%%%%%%%%%%%%%%%
\begin{frame}
\frametitle{Validation form}
\begin{block}{}

Here is an example of a Jupyter cell for displaying a table.

\begin{figure}
 \includegraphics[width=1.\textwidth]{PICTURES/jupyter_table.png}
\end{figure}


\end{block}
\end{frame}
%%%%%%%%%%%%%%%%%%%%%%%%%%%%%%%%%%%%%%%%%%%%%%%%%%%%%%%%%%%%%%%%%%%%%%%%%
%%%%%%%%%%%%%%%%%%%%%%%%%%%%%%%%%%%%%%%%%%%%%%%%%%%%%%%%%%%%%%%%%%%%%%%%%
\begin{frame}
\frametitle{Validation form}
\begin{block}{}

\begin{itemize}
\item Now we are going to add a fourth test case: "debit4"
    \begin{itemize}
    \item [$\circ$] "debit4" corresponds to "std" test case with zero initial velocity and imposed flow rate to $2 m^3/s$ on "periox" boundary.
    \item [$\circ$] Add the test case using "substitute" and "addCase"
    \end{itemize}

\item We are going to rerun the validation form.
    \begin{itemize}
    \item [$\circ$] Re-build the whole validation form by clicking on the icone corresponding to the \"restart of the Jupyter kernel and run of the whole notebook\":

    \item [$\circ$] build the pdf report using: \\

    \textbf{Run\_fiche\index{Run\_fiche} -export\_pdf}
    \item [$\circ$] NB: You can add the results of this test case to your "visualization" and to the "table".
    \end{itemize}

\end{itemize}

\end{block}
\end{frame}
%%%%%%%%%%%%%%%%%%%%%%%%%%%%%%%%%%%%%%%%%%%%%%%%%%%%%%%%%%%%%%%%%%%%%%%


\section{{\bf{Annex: Unix Quick Reference}}}
%%%%%%%%%%%%%%%%%%%%%%%%%%%%%%%%%%%%%%%%%%%%%%%%%%%%%%%%%%%%%%%%%%%%%%%%
\begin{frame}
\begin{small}
\begin{columns}[c] 
\column{.45\textwidth}
\tableofcontents[sections={1-9},currentsection, currentsubsection]
\column{.5\textwidth} 
\tableofcontents[sections={10-16},currentsection, currentsubsection]
\end{columns}
\end{small}
\end{frame}
%%%%%%%%%%%%%%%%%%%%%%%%%%%%%%%%%%%%%%%%%%%%%%%%%%%%%%%%%%%%%%%%%%%%%%%%
%%%%%%%%%%%%%%%%%%%%%%%%%%%%%%%%%%%%%%%%%%%%%%%%%%%%%%%%%%%%%%%%%%%%%%%%
\begin{frame}
\frametitle{Unix Quick Reference}
\begin{block}{File Commands}

\begin{tabular}{ll}
\textbf{ls}              & Directory listing                                                     \tabularnewline
\textbf{ls -al}          & Formatted listing with hidden files                                   \tabularnewline
\textbf{ls -lt}          & Sorting the Formatted listing by time modification                    \tabularnewline
\textbf{cd dir}          & Change directory to dir                                               \tabularnewline
\textbf{cd}              & Change to home directory                                              \tabularnewline
\textbf{pwd}             & Show current working directory                                        \tabularnewline
\textbf{mkdir dir}       & Creating a directory dir                                              \tabularnewline
\textbf{cat $>$file}     & Places the standard input into the file                               \tabularnewline
\textbf{more file}       & Output the contents of the file                                       \tabularnewline
\textbf{head file}       & Output the first 10 lines of the file                                 \tabularnewline
\textbf{tail file}       & Output the last 10 lines of the file                                  \tabularnewline
\textbf{tail -f file}    & Output the contents of file as it grows,starting with the last 10 lines \tabularnewline
\textbf{touch file}      & Create or update file                                                 \tabularnewline
\textbf{rm file}         & Deleting the file                                                     \tabularnewline
\textbf{rm -r dir}       & Deleting the directory                                                \tabularnewline
\end{tabular}

\end{block}
\end{frame}
%%%%%%%%%%%%%%%%%%%%%%%%%%%%%%%%%%%%%%%%%%%%%%%%%%%%%%%%%%%%%%%%%%%%%%%%
%%%%%%%%%%%%%%%%%%%%%%%%%%%%%%%%%%%%%%%%%%%%%%%%%%%%%%%%%%%%%%%%%%%%%%%%
\begin{frame}
\frametitle{Unix Quick Reference}
\begin{block}{File Commands}

\begin{tabular}{ll}
\textbf{rm -f file}      & Force to remove the file                                              \tabularnewline
\textbf{rm -rf dir}      & Force to remove the directory dir                                     \tabularnewline
\textbf{cp file1 file2}  & Copy the contents of file1 to file2                                   \tabularnewline
\textbf{cp -r dir1 dir2} & Copy dir1 to dir2;create dir2 if not present                          \tabularnewline
\textbf{mv file1 file2}  & Rename or move file1 to file2,if file2 is an existing directory       \tabularnewline
\textbf{ln -s file link} & Create symbolic link link to file                                     \tabularnewline
\end{tabular}

\end{block}
\end{frame}
%%%%%%%%%%%%%%%%%%%%%%%%%%%%%%%%%%%%%%%%%%%%%%%%%%%%%%%%%%%%%%%%%%%%%%%%
%%%%%%%%%%%%%%%%%%%%%%%%%%%%%%%%%%%%%%%%%%%%%%%%%%%%%%%%%%%%%%%%%%%%%%%%
\begin{frame}
\frametitle{Unix Quick Reference}

\begin{block}{Process management}
\begin{tabular}{ll}
\textbf{ps}             & To display the currently working processes                              \tabularnewline
\textbf{top}            & Display all running process                                             \tabularnewline
\textbf{kill pid}       & Kill the process with given pid                                         \tabularnewline
\textbf{killall proc}   & Kill all the process named proc                                         \tabularnewline
\textbf{pkill pattern}  & Will kill all processes matching the pattern                            \tabularnewline
\textbf{bg}             & List stopped or background jobs,resume a stopped job                    \tabularnewline
                        & in the background                                                       \tabularnewline
\textbf{fg}             & Brings the most recent job to foreground                                \tabularnewline
\textbf{fg n}           & Brings job n to the foreground                                          \tabularnewline
\end{tabular}
\end{block}

\begin{block}{File permission}
\begin{tabular}{ll}
\textbf{chmod octal file}      & Change the permission of file to octal, which can be found \tabularnewline
                               & separately for user, group, world by adding:               \tabularnewline
 & 4-read(r)    \tabularnewline
 & 2-write(w)   \tabularnewline
 & 1-execute(x) \tabularnewline
\end{tabular}
\end{block}

\end{frame}
%%%%%%%%%%%%%%%%%%%%%%%%%%%%%%%%%%%%%%%%%%%%%%%%%%%%%%%%%%%%%%%%%%%%%%%%
%%%%%%%%%%%%%%%%%%%%%%%%%%%%%%%%%%%%%%%%%%%%%%%%%%%%%%%%%%%%%%%%%%%%%%%%
\begin{frame}
\frametitle{Unix Quick Reference}
\begin{block}{Searching}

\begin{tabular}{ll}
\textbf{grep pattern file}     & Search for pattern in file                                  \tabularnewline
\textbf{grep -r pattern dir}   & Search recursively for pattern in dir                       \tabularnewline
\textbf{command $|$ grep}        & Search pattern in the output of a command pattern         \tabularnewline
\textbf{locate file}           & Find all instances of file                                  \tabularnewline
\textbf{find . -name filename} & Searches in the current directory (represented by           \tabularnewline
                               & a period) and below it, for files and directories           \tabularnewline
                               & with names starting with filename                           \tabularnewline
\textbf{pgrep pattern}         & Searches for all the named processes , that matches         \tabularnewline
                               & with the pattern and, by default, returns their ID          \tabularnewline
\end{tabular}

\end{block}
\end{frame}
%%%%%%%%%%%%%%%%%%%%%%%%%%%%%%%%%%%%%%%%%%%%%%%%%%%%%%%%%%%%%%%%%%%%%%%%
%%%%%%%%%%%%%%%%%%%%%%%%%%%%%%%%%%%%%%%%%%%%%%%%%%%%%%%%%%%%%%%%%%%%%%%%
\begin{frame}
\frametitle{Unix Quick Reference}
\begin{block}{System Info}

\begin{tabular}{ll}
\textbf{date}   & Show the current date and time                \tabularnewline
\textbf{cal}    & Show this month's calender                    \tabularnewline
\textbf{uptime} & Show current uptime                           \tabularnewline
\textbf{w}      & Display who is on line                        \tabularnewline
\textbf{whoami} & Who you are logged in as                      \tabularnewline
\textbf{finger user}      & Display information about user                      \tabularnewline
\textbf{uname -a}      & Show kernel information                                \tabularnewline
\textbf{cat /proc/cpuinfo}      & Cpu information                               \tabularnewline
\textbf{cat proc/meminfo}      & Memory information                             \tabularnewline
\textbf{man command}      & Show the manual for command                         \tabularnewline
\textbf{df}      & Show the disk usage                                          \tabularnewline
\textbf{du}      & Show directory space usage                                   \tabularnewline
\textbf{free}      & Show memory and swap usage                                 \tabularnewline
\textbf{whereis app}      & Show possible locations of app                      \tabularnewline
\textbf{which app}      & Show which applications will be run by default        \tabularnewline
\end{tabular}

\end{block}
\end{frame}
%%%%%%%%%%%%%%%%%%%%%%%%%%%%%%%%%%%%%%%%%%%%%%%%%%%%%%%%%%%%%%%%%%%%%%%%
%%%%%%%%%%%%%%%%%%%%%%%%%%%%%%%%%%%%%%%%%%%%%%%%%%%%%%%%%%%%%%%%%%%%%%%%
\begin{frame}
\frametitle{Unix Quick Reference}

\begin{block}{Compression}
\begin{tabular}{ll}
\textbf{tar cf file.tar file}           & Create tar named file.tar containing file     \tabularnewline
\textbf{tar xf file.tar}                & Extract the files from file.tar               \tabularnewline
\textbf{tar cf file.tar file}           & Create tar named file.tar containing file     \tabularnewline
\textbf{tar xf file.tar}                & Extract the files from file.tar               \tabularnewline
\textbf{tar czf file.tar.gz files}      & Create a tar with Gzip compression            \tabularnewline
\textbf{tar xzf file.tar.gz}            & Extract a tar using Gzip                      \tabularnewline
\textbf{tar cjf file.tar.bz2}           & Create tar with Bzip2 compression             \tabularnewline
\textbf{tar xjf file.tar.bz2}           & Extract a tar using Bzip2                     \tabularnewline
\textbf{gzip file}                      & Compresses file and renames it to file.gz     \tabularnewline
\textbf{gzip -d file.gz}                & Decompresses file.gz back to file             \tabularnewline
\end{tabular}
\end{block}

\end{frame}
%%%%%%%%%%%%%%%%%%%%%%%%%%%%%%%%%%%%%%%%%%%%%%%%%%%%%%%%%%%%%%%%%%%%%%%%
%%%%%%%%%%%%%%%%%%%%%%%%%%%%%%%%%%%%%%%%%%%%%%%%%%%%%%%%%%%%%%%%%%%%%%%%
\begin{frame}
\frametitle{Unix Quick Reference}
\begin{block}{Network}

\begin{tabular}{ll}
\textbf{ping host}      & Ping host and output results                  \tabularnewline
\textbf{whois domain}   & Get whois information for domains             \tabularnewline
\textbf{dig domain}     & Get DNS information for domain                \tabularnewline
\textbf{dig -x host}    & Reverse lookup host                           \tabularnewline
\textbf{wget file}      & Download file                                 \tabularnewline
\textbf{wget -c file}   & Continue a stopped download                   \tabularnewline
\end{tabular}

\end{block}
\end{frame}
%%%%%%%%%%%%%%%%%%%%%%%%%%%%%%%%%%%%%%%%%%%%%%%%%%%%%%%%%%%%%%%%%%%%%%%%
%%%%%%%%%%%%%%%%%%%%%%%%%%%%%%%%%%%%%%%%%%%%%%%%%%%%%%%%%%%%%%%%%%%%%%%%
\begin{frame}
\frametitle{Unix Quick Reference}
\begin{block}{Shortcuts}

\begin{tabular}{ll}
\textbf{"Ctrl"+c} & Halts the current command                                     \tabularnewline
\textbf{"Ctrl"+z} & Stops the current command, resume with fg in the foreground   \tabularnewline
                & or bg in the background                                       \tabularnewline
\textbf{"Ctrl"+d} & Logout the current session, similar to exit                   \tabularnewline
\textbf{"Ctrl"+w} & Erases one word in the current line                           \tabularnewline
\textbf{"Ctrl"+u} & Erases the whole line                                         \tabularnewline
\textbf{"Ctrl"+r} & Type to bring up a recent command                             \tabularnewline
\textbf{!!}     & Repeats the last command                                      \tabularnewline
\textbf{exit}   & Logout the current session                                    \tabularnewline
\end{tabular}

\end{block}
\end{frame}
%%%%%%%%%%%%%%%%%%%%%%%%%%%%%%%%%%%%%%%%%%%%%%%%%%%%%%%%%%%%%%%%%%%%%%%%


\section{{\bf{Index}}}
%%%%%%%%%%%%%%%%%%%%%%%%%%%%%%%%%%%%%%%%%%%%%%%%%%%%%%%%%%%%%%%%%%%%%%%%
\begin{frame}
\begin{small}
\begin{columns}[c] 
\column{.45\textwidth}
\tableofcontents[sections={1-9},currentsection, currentsubsection]
\column{.5\textwidth} 
\tableofcontents[sections={10-16},currentsection, currentsubsection]
\end{columns}
\end{small}
\end{frame}
%%%%%%%%%%%%%%%%%%%%%%%%%%%%%%%%%%%%%%%%%%%%%%%%%%%%%%%%%%%%%%%%%%%%%%%%
%%%%%%%%%%%%%%%%%%%%%%%%%%%%%%%%%%%%%%%%%%%%%%%%%%%%%%%%%%%%%%%%%%%%%%%%
\begin{frame}[allowframebreaks]
\frametitle{Index}
\printindex
\end{frame}
%%%%%%%%%%%%%%%%%%%%%%%%%%%%%%%%%%%%%%%%%%%%%%%%%%%%%%%%%%%%%%%%%%%%%%%%


%%%%%%%%%%%%%%%%%%%%%%%%%%%%%%%%%%%%%%%%%%%%%%%%%%%%%%%%%%%%%%%%%%%%%%%%
\begin{frame}
\begin{center}
{\Huge{End}}\\
\vspace{2cm}
\textbf{Tutorial solutions}: \textit{\$TRUST\_ROOT/doc/TRUST/exercices/Tutorial\_solutions.pdf}
\end{center}
\end{frame}
%%%%%%%%%%%%%%%%%%%%%%%%%%%%%%%%%%%%%%%%%%%%%%%%%%%%%%%%%%%%%%%%%%%%%%%%

\end{document} 
