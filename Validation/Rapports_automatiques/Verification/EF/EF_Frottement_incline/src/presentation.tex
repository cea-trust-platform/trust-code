On consid�re un �coulement dans un faisceau inclin� d'un angle
$\alpha$ par rapport � la direction verticale tel que repr�sent� sur la Fig.~\ref{fig:schema}.

\begin{figure}[!hbt]
\begin{center}
\input{\orig/figure_frottement_incline.pstex_t}
\end{center}
\caption{Sch�matisation de la configuration trait� avec repr�sentation du faisceau inclin� d'un angle $\alpha$.}
\label{fig:schema}
\end{figure}

On suppose que, dans le r�f�rentiel \emph{local}, le tenseur de
frottement $\boldsymbol{\Lambda}_{loc}$ s'�crit de la mani�re suivante
:
\begin{equation}
\boldsymbol{\Lambda}_{loc} =
\left[
\begin{array}{cc}
a & 0 \\
0 & b
\end{array}
\right]
\end{equation}
Dans le cas o� l'on suppose un mod�le de frottement en module, on a :
\begin{equation}
a = F_1 \, |\boldsymbol{u}|
\end{equation}
\begin{equation}
b = F_2 \, |\boldsymbol{u}|
\end{equation}
o� $F_1$ et $F_2$ sont les coefficients de frottement perpendiculaire
et parall�le au faisceau respectivement.

\medskip

On montre que, dans le r�f�rentiel global, on a :
\begin{equation}
\boldsymbol{\Lambda} =
\left[
\begin{array}{cc}
a \, \cos^2\alpha + b \, \sin^2\alpha & (a-b) \, \cos\alpha \, \sin\alpha \\
(a-b) \, \cos\alpha \, \sin\alpha & a \, \sin^2\alpha + b \, \cos^2\alpha
\end{array}
\right]
\end{equation}

\medskip

Si l'on suppose que seules les forces de pression et de frottement
sont prises en compte, le bilan de quantit� de mouvement se r�duit �
l'�quation suivante~:
\begin{equation}
-\nabla P - \boldsymbol{\Lambda}\cdot \rho \, \boldsymbol{u} = 0
\end{equation}

On cherche alors une solution analytique pour laquelle le champ de
vitesse est dirig�e verticalement~:
\begin{equation}
\boldsymbol{u} = U_0 \, \boldsymbol{e}_y
\end{equation}
o� $U_0$ est une constante.

Avec cette hypoth�se, on montre facilement que la force de frottement
est la suivante~:
\begin{equation}
-\boldsymbol{\Lambda} \cdot \rho \, \boldsymbol{u} = - \rho \, U_0^2 \left[ \left( F_1 - F_2 \right) \cos\alpha \, \sin\alpha \right] \boldsymbol{e}_x - \rho \, U_0^2 \left[ F_1 \, \sin^2\alpha + F_2 \, \cos^2\alpha \right] \boldsymbol{e}_y
\end{equation}

On peut v�rifier que le champ de pression suivant est solution de
l'�quation de bilan de quantit� de mouvement~:
\begin{equation}
P(x,y) = - \rho \, U_0^2 \left\{ \left[ \left( F_1 - F_2 \right) \cos\alpha \, \sin\alpha \right] x + \left[ F_1 \, \sin^2\alpha + F_2 \, \cos^2\alpha \right] y \right\}
\end{equation}
