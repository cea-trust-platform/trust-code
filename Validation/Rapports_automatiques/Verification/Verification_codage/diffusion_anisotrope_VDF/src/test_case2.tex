The conductivity tensor for this case is anisotropic and non-uniform in $\Omega$, expressed as
\begin{equation}\label{tensor_c2}
\kappa(x,y)=\left(\begin{matrix} \eta x^2 +y^2 & 0 \\ 0 & x^2 + \eta y^2 \end{matrix}\right). 
\end{equation}
Here $\eta$ is a positive parameter characterizing the level of anisotropy. We note the anisotropy ratio as
\begin{equation}
A_r=\frac{1}{\eta}.
\end{equation}

Moreover, a thermal source term $\dot{q_v}$ is applied to the system in this test case. Its formulation is defined so that the analytical solution of the problem is 
\begin{equation}\label{sol_c2}
T_{exact}(x,y)=\sin^2(\pi x)\sin^2(\pi y).
\end{equation}
On the boundaries $\partial \Omega$, a Dirichlet boundary condition $T(x,y,t)=0$ is prescribed. 

In order to highlight the inflence of the anisotropy, four values of $A_r$ are investigated; respectively 1, 10, 100 and 1000. This leads not only to different values in the components of the $\kappa$ tensor, but also to different source term that verifies the exact solution \eqref{sol_c2}. In this work, we use the SYMPY library to derive the expression of $\dot{q_v}$ depending on the considered $A_r$. Recall that SYMPY is a library provided with python3 and TRUST. It allows manipulating mathematical operartions with symbolic variables; here the term $-\nabla.(\kappa \nabla T)=\dot{q_v}$ (at steady state). 

We denote by $\dot{q_v}^1$ the source term corresponding to $A_r=1$ ($\eta=1$). Similarly, $\dot{q_v}^{10}$ is referred to for the case of $A_r=10$ ($\eta=0.1$), $\dot{q_v}^{100}$ for $A_r=100$ ($\eta=0.01$) and $\dot{q_v}^{1000}$ for $A_r=1000$ ($\eta=0.001$). Following these notations, the imposed source terms are respectively expressed as

\begin{multline}
\dot{q_v}^1=- 4.0 \pi x \sin{\left(\pi x \right)} \sin^{2}{\left(\pi y \right)} \cos{\left(\pi x \right)} - 4.0 \pi y \sin^{2}{\left(\pi x \right)} \sin{\left(\pi y \right)} \cos{\left(\pi y \right)} + \\2 \pi^{2} \left(1.0 x^{2} + y^{2}\right) \sin^{2}{\left(\pi x \right)} \sin^{2}{\left(\pi y \right)} - 2 \pi^{2} \left(1.0 x^{2} + y^{2}\right) \sin^{2}{\left(\pi y \right)} \cos^{2}{\left(\pi x \right)} + \\2 \pi^{2} \left(x^{2} + 1.0 y^{2}\right) \sin^{2}{\left(\pi x \right)} \sin^{2}{\left(\pi y \right)} - 2 \pi^{2} \left(x^{2} + 1.0 y^{2}\right) \sin^{2}{\left(\pi x \right)} \cos^{2}{\left(\pi y \right)}
\end{multline}

\begin{multline}
\dot{q_v}^{10}=- 0.4 \pi x \sin{\left(\pi x \right)} \sin^{2}{\left(\pi y \right)} \cos{\left(\pi x \right)} - 0.4 \pi y \sin^{2}{\left(\pi x \right)} \sin{\left(\pi y \right)} \cos{\left(\pi y \right)} + \\2 \pi^{2} \left(0.1 x^{2} + y^{2}\right) \sin^{2}{\left(\pi x \right)} \sin^{2}{\left(\pi y \right)} - 2 \pi^{2} \left(0.1 x^{2} + y^{2}\right) \sin^{2}{\left(\pi y \right)} \cos^{2}{\left(\pi x \right)} + \\2 \pi^{2} \left(x^{2} + 0.1 y^{2}\right) \sin^{2}{\left(\pi x \right)} \sin^{2}{\left(\pi y \right)} - 2 \pi^{2} \left(x^{2} + 0.1 y^{2}\right) \sin^{2}{\left(\pi x \right)} \cos^{2}{\left(\pi y \right)}
\end{multline}

\begin{multline}
\dot{q_v}^{100}=- 0.04 \pi x \sin{\left(\pi x \right)} \sin^{2}{\left(\pi y \right)} \cos{\left(\pi x \right)} - 0.04 \pi y \sin^{2}{\left(\pi x \right)} \sin{\left(\pi y \right)} \cos{\left(\pi y \right)} + \\2 \pi^{2} \left(0.01 x^{2} + y^{2}\right) \sin^{2}{\left(\pi x \right)} \sin^{2}{\left(\pi y \right)} - 2 \pi^{2} \left(0.01 x^{2} + y^{2}\right) \sin^{2}{\left(\pi y \right)} \cos^{2}{\left(\pi x \right)} + \\2 \pi^{2} \left(x^{2} + 0.01 y^{2}\right) \sin^{2}{\left(\pi x \right)} \sin^{2}{\left(\pi y \right)} - 2 \pi^{2} \left(x^{2} + 0.01 y^{2}\right) \sin^{2}{\left(\pi x \right)} \cos^{2}{\left(\pi y \right)}
\end{multline}

\begin{multline}
\dot{q_v}^{1000}=- 0.004 \pi x \sin{\left(\pi x \right)} \sin^{2}{\left(\pi y \right)} \cos{\left(\pi x \right)} - 0.004 \pi y \sin^{2}{\left(\pi x \right)} \sin{\left(\pi y \right)} \cos{\left(\pi y \right)} +\\ 2 \pi^{2} \left(0.001 x^{2} + y^{2}\right) \sin^{2}{\left(\pi x \right)} \sin^{2}{\left(\pi y \right)} - 2 \pi^{2} \left(0.001 x^{2} + y^{2}\right) \sin^{2}{\left(\pi y \right)} \cos^{2}{\left(\pi x \right)} + \\2 \pi^{2} \left(x^{2} + 0.001 y^{2}\right) \sin^{2}{\left(\pi x \right)} \sin^{2}{\left(\pi y \right)} - 2 \pi^{2} \left(x^{2} + 0.001 y^{2}\right) \sin^{2}{\left(\pi x \right)} \cos^{2}{\left(\pi y \right)}
\end{multline}



Finally, to correctly investigate the influence of the anisotropy ratio on the quality of the numerical solution, three meshes are considered for this test case. They are respectively referred to by M1, M2 and M3. M1 is the coarsest one and consists of 51x51 squares. M2 is here an intermidiate resolution mesh and consists of 90x90 squares distributed over 2 MPI procs. The finest one is M3 with 180x180 squares splitted 4 MPI procs. We imphasize here that the idea is \textbf{not to perform a mesh convergence analysis}, but analyze the influence of the mesh by showing how a fine resolution is required when $A_r$ becomes important. Figures 3.1-3.3 depict the three meshes used for this test case. Numerical results, at steady state, are discussed and compared to the analytical solution \eqref{sol_c2} in sections 4 to 7.


