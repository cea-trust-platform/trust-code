For this test case, the source term is set to zero; $\dot{q_v}=0$. The heat conductivity tensor is anisotropic and discontinuous, defined $\forall (x,y)\in \Omega$ as 

\begin{equation}\label{tensor_c1}
\kappa(x,y)=
\left\{
\begin{aligned}
&\displaystyle\left(\begin{matrix} \kappa_l^{xx} & \kappa_l^{xy} \\ \kappa_l^{yx} & \kappa_l^{yy} \end{matrix}\right) \quad : \quad x\in[0,0.5], \\
&\displaystyle\left(\begin{matrix} \kappa_r^{xx} & \kappa_r^{xy} \\ \kappa_r^{yx} & \kappa_r^{yy} \end{matrix}    \right) \quad : \quad x\in[0.5,1].
\end{aligned}
\right.
\end{equation}

For such a $\kappa$ distribution, the analytical solution $T_{exact}$ is one-dimensional. With the Dirichlet boundary conditions defined as $T_{exact}(x=0)=0$ and $T_{exact}(x=1)=1$, the solution reads as
\begin{equation}\label{sol_c1}
T_{exact}(x)=\left\{
\begin{aligned}
&\displaystyle x\frac{2\kappa_r^{xx}}{\kappa_l^{xx}+\kappa_r^{xx}} \quad &: \quad x\in[0,0.5], \\
&\displaystyle\frac{\kappa_r^{xx}-\kappa_l^{xx}}{\kappa_r^{xx}+\kappa_l^{xx}} + x\frac{2\kappa_l^{xx}}{\kappa_l^{xx}+\kappa_r^{xx}}\quad &: \quad x\in[0.5,1].
\end{aligned}
\right.
\end{equation}

The mesh is triangular, generated by the MCADSurf algorith and consists of 2500 triangles (edge length between 0.02-0.03), thanks to the open source SALOME platform (see figure 2.1).  The boundary conditions at the left and at the right walls are set to $T=0$ and 1 K respectively. At the top and the bottom boundaries, a Dirichlet condition is prescribed satisfying the equation \eqref{sol_c1}.

For the numerical simulation, we assume $\kappa_l^{xx}=1$, $\kappa_l^{xy}= \kappa_l^{yx}=-1$, $\kappa_l^{yy}=4$, $\kappa_r^{xx}=10$, $\kappa_r^{xy}= \kappa_r^{yx}=-3$ and  $\kappa_r^{yy}=2$. The resulting components of the tensor are illustrated in figures 2.2-2.4 where the discontinuity of the fields is clearly noted. Figure 2.5 depict the temperature distribution at the steady state (physical time about 0.5 s). The error relative to the exact solution is considered in figure 2.6.  It is clear how the difference between the numerical and exact solution is very small. 


