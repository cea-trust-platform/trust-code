\section{Syntax to define a mathematical function\label{parser}}
In a mathematical function,
used for example in field definition, it's possible to use the predifined function (an object parser is used to
evaluate the functions) :

ABS\ \ \ \ : absolute value function

COS \ \ \ \ : cosine function

SIN\ \ \ \ : sine function

TAN\ \ \ \ : tangent function

ATAN\ \ : arctangent function

EXP\ \ \ \ : exponential function

LN\ \ \ \ : natural logarithm function

SQRT \ \ : square root function

INT\ \ \ \ : integer function

ERF\ \ \ \ : error function

RND(x)\ \ : random function (values between 0 and x)

COSH\ \ \ \ : hyperbolic cosine function

SINH\ \ \ \ : hyperbolic sine function

TANH\ \ \ \ : hyperbolic tangent function

ACOS\ \ \ \ : inverse cosine function

ASIN\ \ \ \ : inverse sine function

ATANH\ \ : inverse hyperbolic tangent function

NOT(x)\ \ : NOT x (returns 1 if x is false, 0 otherwise) 

SGN(x)\ \ : SGN x (returns 1 if x is positive, -1 if negative, 0 if zero) 

x\_AND\_y \ \ : boolean logical operation AND (returns 1 if both x and y are true, else 0)

x\_OR\_y\ \ : boolean logical operation OR (returns 1 if x or y is true, else 0)

x\_GT\_y\ \ : greater than (returns 1 if x{\textgreater}y, else 0)

x\_GE\_y\ \ : greater than or equal to (returns 1 if x{\textgreater}=y, else 0)

x\_LT\_y\ \ : less than (returns 1 if x{\textless}y, else 0)

x\_LE\_y\ \ : less than or equal to (returns 1 if x{\textless}=y, else 0)

x\_MIN\_y \ \ \ \ \ : returns the smallest of x and y

x\_MAX\_y \ \ \ \ : returns the largest of x and y

x\_MOD\_y \ \ \ \ : modular division of x per y

x\_EQ\_y \ \ \ \ \ \ \ \ : equal to (returns 1 if x==y, else 0)

x\_NEQ\_y \ \ \ \ \ : not equal to (returns 1 if x!=y, else 0) 


\bigskip

You can also use the following operations:

+\ \ : addition

{}- \ \ : subtraction

/ \ \ : division

*\ \ : multiplication

\%\ \ : modulo

\$\ \ : max

\^{} \ \ : power

{\textless}\ \ : less than

{\textgreater}\ \ : greater than

[\ \ : less than or equal to

]\ \ : greater than or equal to


\bigskip

You can also use the following constants:

Pi \ \ : pi value (3,1415{\dots})


\bigskip

The variables which can be used are:

x,y,z \ \ : coordinates 

t \ \ : time


\bigskip

{\bfseries
Examples:}

Champ\_front\_fonc\_txyz\index{Champ\_front\_fonc\_txyz} \ 2 \ cos(y+x\^{}2) \ t+ln(y)

Champ\_fonc\_xyz\index{xyz} dom 2 tanh(4*y)*(0.95+0.1*rnd(1)) 0.


\bigskip

{\bfseries
Possible errors:}

Error 1:

Champ\_fonc\_txyz 1 \ cos(10*t)*(1{\textless}x{\textless}2)*(1{\textless}y{\textless}2)

Previous line is wrong. It should be written as:

Champ\_fonc\_txyz 1 \ cos(10*t)*(1{\textless}x)*(x{\textless}2)*(1{\textless}y)*(y{\textless}2)

\bigskip
Error 2:

Champ\_front\_fonc\_xyz 1 \ 20*(x{\textless}-2)+10*(y]-5)+3*(z{\textgreater}0)

Previous line is wrong because negative values are not written between parentheses. It should be written as:

Champ\_front\_fonc\_xyz 1 \ 20*(x{\textless}(-2))+10*(y](-5))+3*(z{\textgreater}0)
